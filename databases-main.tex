% $Id: BASENAME-main.tex 23686 2019-02-20 20:00:48Z beringer $
% Main file for PDG review databases
%
% This file is included by the top-level file databases.tex and is where the
% text of your review should be included. If desired, you may split your review into multiple
% files that are included from this file using \input.
%
% Do NOT modify the top-level file databases.tex - it is generated from
% the PDG database and all manual changes WILL BE LOST!


% Review title
% ------------
% Please use \pdgtitle (rather than e.g. \chapter) to put the title of your review.
%
% To put the review title extracted from the PDG database, use \pdgtitle (no arguments).
% To override the default review title, you can use \pdgtitle[Some different title].
% In the latter case, please ask your overseer to update the review title in the database
\pdgtitle


% Table of contents
% -----------------
% If you want to include a table of contents at the start of your review,
% uncomment the line below. This should only be done for relatively long
% reviews.
%\tableofcontents


% Author information for this review
% ----------------------------------
% Please use one of the following forms:
%   \written{month year}
%   \revised{month year}
%   \customauthor{...}
% The first two, \written and \revised take the month and year when the review
% was written or revised as their only argument. The author list is generated
% from the PDG database. This is the preferred way of including author information.
% Only if really needed, you may use the third from, \customauthor, where you can
% specify the full text of the paragraph giving the review author information.
\revised{August 2019}


% Sectioning
% ----------
% This review is a regular review, please use \section for your top-level sectioning.
% For example:
%\section{Your first section title}


% Text of your review
% -------------------
% Please see the instructions at https://pdgdoc.lbl.gov/Pdg/ReviewTool on how to
% include figures, tables, and references. By default we include file examples.tex,
% which provides instructions and examples of how to include figures, tables, how
% to align equations, and more. It also produces an appendix showing all the standard
% PDG symbols and what they produce.
%
% To remove the examples from your review, comment out the following line when you
% start writing your review.
%\input{examples.tex}
\input{content.tex}


% References
% ----------
% The following line includes your bibliography using BibTeX. In case you do not
% yet use BibTex, you can put your bibliography below (using a series of \bibitem entries).
% Please  note that using BibTeX will become mandatory in the future.
\IfFileExists{databases.bib}{\putbib[databases]}{}



