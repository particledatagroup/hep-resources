%% file: mtexsis.tex (TeXsis version 2.17a)
%@(#) $Revision: 17.4 $ : $Date: 1999/04/07 01:05:56 $ : $Author: myers $
%======================================================================*
%  mTeXsis - mini version of the TeX Macros for Physicists
%
%  This is mtexsis.tex, a "mini" version of the TeXsis macro package
%  containing only the core TeXsis macros.  You can get the full
%  version of TeXsis, with complte documentation, from
%
%               ftp://lifshitz.ph.utexas.edu/texsis
%
%  If you receive a manuscript file prepared with TeXsis you can
%  (usually) print it by inserting the line
%
%       \input mtexsis
%
%  at the begining of the manuscript file (in place of \texsis, if it
%  is there) and then running it through Plain TeX.
%
%  Authors who prepare papers with TeXsis for submission to physics
%  e-print archives like hep-lat, hep-th or hep-ph should try their
%  manuscripts with Plain TeX and mtexsis.tex to be sure that it
%  works.  You can make your file automatically work with either
%  TeXsis or Plain and mtexsis by putting the following at the
%  begining of your manuscript file (in place of just \texsis):
%
%       \ifx\texsis\undefined\input mtexsis\else\texsis\fi
%
%  If you want to know how a macro works you will have to look in the
%  full TeXsis source files.  There are copious comments.
%
%  Comments, suggestions, or bug reports should be directed to 
%  texsis@lifshitz.ph.utexas.edu
%
%  (C) copyright 1988, 1992, 1994, 1997 by Eric Myers and Frank Paige.
%  Permission is granted to use this software by any member the academic
%  community in conjunction with basic research.  This software may not
%  be used or distributed for profit without permission from the authors.
%======================================================================*
\def\TeXsis{\TeX sis}%                          % the TeXsis logo
\catcode`@=11                                   % @ is a letter in what follows

% file: TXSfonts.tex (TeXsis version 2.17)
%%> {TeXsis fonts:}
\catcode`@=11
\newskip\ttglue
%++ 9 pt fonts:
\def\ninefonts{%
   \global\font\ninerm=cmr9
   \global\font\ninei=cmmi9
   \global\font\ninesy=cmsy9
   \global\font\nineex=cmex10
   \global\font\ninebf=cmbx9
   \global\font\ninesl=cmsl9
   \global\font\ninett=cmtt9
   \global\font\nineit=cmti9
   \skewchar\ninei='177
   \skewchar\ninesy='60
   \hyphenchar\ninett=-1
   \moreninefonts
   \gdef\ninefonts{\relax}}
\def\moreninefonts{\relax}%
%++ 10 pt fonts:                               
\font\tenss=cmss10
\def\moretenfonts{\relax}%
%++ 11 pt fonts:
\def\elevenfonts{%
   \global\font\elevenrm=cmr10 scaled \magstephalf
   \global\font\eleveni=cmmi10 scaled \magstephalf
   \global\font\elevensy=cmsy10 scaled \magstephalf
   \global\font\elevenex=cmex10
   \global\font\elevenbf=cmbx10 scaled \magstephalf
   \global\font\elevensl=cmsl10 scaled \magstephalf
   \global\font\eleventt=cmtt10 scaled \magstephalf
   \global\font\elevenit=cmti10 scaled \magstephalf
   \global\font\elevenss=cmss10 scaled \magstephalf
   \skewchar\eleveni='177%
   \skewchar\elevensy='60%
   \hyphenchar\eleventt=-1%
   \moreelevenfonts
   \gdef\elevenfonts{\relax}}%
\def\moreelevenfonts{\relax}%
%++ 12 pt fonts:
\def\twelvefonts{%
   \global\font\twelverm=cmr10 scaled \magstep1%
   \global\font\twelvei=cmmi10 scaled \magstep1%
   \global\font\twelvesy=cmsy10 scaled \magstep1%
   \global\font\twelveex=cmex10 scaled \magstep1%
   \global\font\twelvebf=cmbx10 scaled \magstep1%
   \global\font\twelvesl=cmsl10 scaled \magstep1%
   \global\font\twelvett=cmtt10 scaled \magstep1%
   \global\font\twelveit=cmti10 scaled \magstep1%
   \global\font\twelvess=cmss10 scaled \magstep1%
   \skewchar\twelvei='177%
   \skewchar\twelvesy='60%
   \hyphenchar\twelvett=-1%
   \moretwelvefonts
   \gdef\twelvefonts{\relax}}
\def\moretwelvefonts{\relax}%
%++ 14 pt fonts:
\def\fourteenfonts{%
   \global\font\fourteenrm=cmr10 scaled \magstep2%
   \global\font\fourteeni=cmmi10 scaled \magstep2%
   \global\font\fourteensy=cmsy10 scaled \magstep2%
   \global\font\fourteenex=cmex10 scaled \magstep2%
   \global\font\fourteenbf=cmbx10 scaled \magstep2%
   \global\font\fourteensl=cmsl10 scaled \magstep2%
   \global\font\fourteenit=cmti10 scaled \magstep2%
   \global\font\fourteenss=cmss10 scaled \magstep2%
   \skewchar\fourteeni='177%
   \skewchar\fourteensy='60%
   \morefourteenfonts
   \gdef\fourteenfonts{\relax}}
\def\morefourteenfonts{\relax}%
%++ 16 pt fonts:
\def\sixteenfonts{%
   \global\font\sixteenrm=cmr10 scaled \magstep3%
   \global\font\sixteeni=cmmi10 scaled \magstep3%
   \global\font\sixteensy=cmsy10 scaled \magstep3%
   \global\font\sixteenex=cmex10 scaled \magstep3%
   \global\font\sixteenbf=cmbx10 scaled \magstep3%
   \global\font\sixteensl=cmsl10 scaled \magstep3%
   \global\font\sixteenit=cmti10 scaled \magstep3%
   \skewchar\sixteeni='177%
   \skewchar\sixteensy='60%
   \moresixteenfonts
   \gdef\sixteenfonts{\relax}}
\def\moresixteenfonts{\relax}%
%++ 20 pt fonts:
\def\twentyfonts{%
   \global\font\twentyrm=cmr10 scaled \magstep4%
   \global\font\twentyi=cmmi10 scaled \magstep4%
   \global\font\twentysy=cmsy10 scaled \magstep4%
   \global\font\twentyex=cmex10 scaled \magstep4%
   \global\font\twentybf=cmbx10 scaled \magstep4%
   \global\font\twentysl=cmsl10 scaled \magstep4%
   \global\font\twentyit=cmti10 scaled \magstep4%
   \skewchar\twentyi='177%
   \skewchar\twentysy='60%
   \moretwentyfonts
   \gdef\twentyfonts{\relax}}
\def\moretwentyfonts{\relax}%
%++ 24 pt fonts:
\def\twentyfourfonts{%
   \global\font\twentyfourrm=cmr10 scaled \magstep5%
   \global\font\twentyfouri=cmmi10 scaled \magstep5%
   \global\font\twentyfoursy=cmsy10 scaled \magstep5%
   \global\font\twentyfourex=cmex10 scaled \magstep5%
   \global\font\twentyfourbf=cmbx10 scaled \magstep5%
   \global\font\twentyfoursl=cmsl10 scaled \magstep5%
   \global\font\twentyfourit=cmti10 scaled \magstep5%
   \skewchar\twentyfouri='177%
   \skewchar\twentyfoursy='60%
   \moretwentyfourfonts
   \gdef\twentyfourfonts{\relax}}
\def\moretwentyfourfonts{\relax}%
\def\tenmibfonts{%
   \global\font\tenmib=cmmib10
   \global\font\tenbsy=cmbsy10
   \skewchar\tenmib='177%
   \skewchar\tenbsy='60%
   \gdef\tenmibfonts{\relax}}
\def\elevenmibfonts{%
   \global\font\elevenmib=cmmib10 scaled \magstephalf
   \global\font\elevenbsy=cmbsy10 scaled \magstephalf
   \skewchar\elevenmib='177%
   \skewchar\elevenbsy='60%
   \gdef\elevenmibfonts{\relax}}
\def\twelvemibfonts{%
   \global\font\twelvemib=cmmib10 scaled \magstep1%
   \global\font\twelvebsy=cmbsy10 scaled \magstep1%
   \skewchar\twelvemib='177%
   \skewchar\twelvebsy='60%
   \gdef\twelvemibfonts{\relax}}
\def\fourteenmibfonts{%
   \global\font\fourteenmib=cmmib10 scaled \magstep2%
   \global\font\fourteenbsy=cmbsy10 scaled \magstep2%
   \skewchar\fourteenmib='177%
   \skewchar\fourteenbsy='60%
   \gdef\fourteenmibfonts{\relax}}
\def\sixteenmibfonts{%
   \global\font\sixteenmib=cmmib10 scaled \magstep3%
   \global\font\sixteenbsy=cmbsy10 scaled \magstep3%
   \skewchar\sixteenmib='177%
   \skewchar\sixteenbsy='60%
   \gdef\sixteenmibfonts{\relax}}
\def\twentymibfonts{%
   \global\font\twentymib=cmmib10 scaled \magstep4%
   \global\font\twentybsy=cmbsy10 scaled \magstep4%
   \skewchar\twentymib='177%
   \skewchar\twentybsy='60%
   \gdef\twentymibfonts{\relax}}
\def\twentyfourmibfonts{%
   \global\font\twentyfourmib=cmmib10 scaled \magstep5%
   \global\font\twentyfourbsy=cmbsy10 scaled \magstep5%
   \skewchar\twentyfourmib='177%
   \skewchar\twentyfourbsy='60%
   \gdef\twentyfourmibfonts{\relax}}
%%> {\string\mib,}
\def\mib{%
   \tenmibfonts
   \textfont0=\tenbf\scriptfont0=\sevenbf
   \scriptscriptfont0=\fivebf
   \textfont1=\tenmib\scriptfont1=\seveni
   \scriptscriptfont1=\fivei
   \textfont2=\tenbsy\scriptfont2=\sevensy
   \scriptscriptfont2=\fivesy}
%%> {\string\scr,}
\newfam\scrfam
\def\scr{\scrfonts\fam\scrfam\tenscr
   \global\textfont\scrfam=\tenscr\global\scriptfont\scrfam=\sevenscr
   \global\scriptscriptfont\scrfam=\fivescr}
\def\scrfonts{%
   \global\font\twentyfourscr=rsfs10  scaled \magstep5
   \global\font\twentyscr=rsfs10  scaled \magstep4
   \global\font\sixteenscr=rsfs10  scaled \magstep3
   \global\font\fourteenscr=rsfs10  scaled \magstep2
   \global\font\twelvescr=rsfs10  scaled \magstep1
   \global\font\elevenscr=rsfs10  scaled \magstephalf
   \global\font\tenscr=rsfs10
   \global\font\ninescr=rsfs7 scaled \magstep1
   \global\font\sevenscr=rsfs7
   \global\font\fivescr=rsfs5
   \global\skewchar\tenscr='177 \global\skewchar\sevenscr='177%
        \global\skewchar\fivescr='177%
   \global\textfont\scrfam=\tenscr\global\scriptfont\scrfam=\sevenscr
        \global\scriptscriptfont\scrfam=\fivescr
   \gdef\scrfonts{\relax}}%
%%> {9pt,}
\def\ninepoint{\ninefonts
   \def\rm{\fam0\ninerm}%
   \textfont0=\ninerm\scriptfont0=\sevenrm\scriptscriptfont0=\fiverm
   \textfont1=\ninei\scriptfont1=\seveni\scriptscriptfont1=\fivei
   \textfont2=\ninesy\scriptfont2=\sevensy\scriptscriptfont2=\fivesy
   \textfont3=\nineex\scriptfont3=\nineex\scriptscriptfont3=\nineex
   \textfont\itfam=\nineit\def\it{\fam\itfam\nineit}%
   \textfont\slfam=\ninesl\def\sl{\fam\slfam\ninesl}%
   \textfont\ttfam=\ninett\def\tt{\fam\ttfam\ninett}%
   \textfont\bffam=\ninebf
   \scriptfont\bffam=\sevenbf
   \scriptscriptfont\bffam=\fivebf\def\bf{\fam\bffam\ninebf}%
   \def\mib{\relax}%
   \def\scr{\relax}%
   \tt\ttglue=.5emplus.25emminus.15em
   \normalbaselineskip=11pt
   \setbox\strutbox=\hbox{\vrule height 8pt depth 3pt width 0pt}%
   \normalbaselines\rm\singlespaced}%
%%> {10pt,}
\def\tenpoint{%
   \def\rm{\fam0\tenrm}%
   \textfont0=\tenrm\scriptfont0=\sevenrm\scriptscriptfont0=\fiverm
   \textfont1=\teni\scriptfont1=\seveni\scriptscriptfont1=\fivei
   \textfont2=\tensy\scriptfont2=\sevensy\scriptscriptfont2=\fivesy
   \textfont3=\tenex\scriptfont3=\tenex\scriptscriptfont3=\tenex
   \textfont\itfam=\tenit\def\it{\fam\itfam\tenit}%
   \textfont\slfam=\tensl\def\sl{\fam\slfam\tensl}%
   \textfont\ttfam=\tentt\def\tt{\fam\ttfam\tentt}%
   \textfont\bffam=\tenbf
   \scriptfont\bffam=\sevenbf
   \scriptscriptfont\bffam=\fivebf\def\bf{\fam\bffam\tenbf}%
   \def\mib{%
      \tenmibfonts
      \textfont0=\tenbf\scriptfont0=\sevenbf
      \scriptscriptfont0=\fivebf
      \textfont1=\tenmib\scriptfont1=\seveni
      \scriptscriptfont1=\fivei
      \textfont2=\tenbsy\scriptfont2=\sevensy
      \scriptscriptfont2=\fivesy}%
   \def\scr{\scrfonts\fam\scrfam\tenscr
      \global\textfont\scrfam=\tenscr\global\scriptfont\scrfam=\sevenscr
      \global\scriptscriptfont\scrfam=\fivescr}%
   \tt\ttglue=.5emplus.25emminus.15em
   \normalbaselineskip=12pt
   \setbox\strutbox=\hbox{\vrule height 8.5pt depth 3.5pt width 0pt}%
   \normalbaselines\rm\singlespaced}%
%%> {11pt,}
\def\elevenpoint{\elevenfonts
   \def\rm{\fam0\elevenrm}%
   \textfont0=\elevenrm\scriptfont0=\sevenrm\scriptscriptfont0=\fiverm
   \textfont1=\eleveni\scriptfont1=\seveni\scriptscriptfont1=\fivei
   \textfont2=\elevensy\scriptfont2=\sevensy\scriptscriptfont2=\fivesy
   \textfont3=\elevenex\scriptfont3=\elevenex\scriptscriptfont3=\elevenex
   \textfont\itfam=\elevenit\def\it{\fam\itfam\elevenit}%
   \textfont\slfam=\elevensl\def\sl{\fam\slfam\elevensl}%
   \textfont\ttfam=\eleventt\def\tt{\fam\ttfam\eleventt}%
   \textfont\bffam=\elevenbf
   \scriptfont\bffam=\sevenbf
   \scriptscriptfont\bffam=\fivebf\def\bf{\fam\bffam\elevenbf}%
   \def\mib{%
      \elevenmibfonts
      \textfont0=\elevenbf\scriptfont0=\sevenbf
      \scriptscriptfont0=\fivebf
      \textfont1=\elevenmib\scriptfont1=\seveni
      \scriptscriptfont1=\fivei
      \textfont2=\elevenbsy\scriptfont2=\sevensy
      \scriptscriptfont2=\fivesy}%
   \def\scr{\scrfonts\fam\scrfam\elevenscr
      \global\textfont\scrfam=\elevenscr\global\scriptfont\scrfam=\sevenscr
      \global\scriptscriptfont\scrfam=\fivescr}%
   \tt\ttglue=.5emplus.25emminus.15em
   \normalbaselineskip=13pt
   \setbox\strutbox=\hbox{\vrule height 9pt depth 4pt width 0pt}%
   \normalbaselines\rm\singlespaced}%
%%> {12pt,}
\def\twelvepoint{\twelvefonts\ninefonts
   \def\rm{\fam0\twelverm}%
   \textfont0=\twelverm\scriptfont0=\ninerm\scriptscriptfont0=\sevenrm
   \textfont1=\twelvei\scriptfont1=\ninei\scriptscriptfont1=\seveni
   \textfont2=\twelvesy\scriptfont2=\ninesy\scriptscriptfont2=\sevensy
   \textfont3=\twelveex\scriptfont3=\twelveex\scriptscriptfont3=\twelveex
   \textfont\itfam=\twelveit\def\it{\fam\itfam\twelveit}%
   \textfont\slfam=\twelvesl\def\sl{\fam\slfam\twelvesl}%
   \textfont\ttfam=\twelvett\def\tt{\fam\ttfam\twelvett}%
   \textfont\bffam=\twelvebf
   \scriptfont\bffam=\ninebf
   \scriptscriptfont\bffam=\sevenbf\def\bf{\fam\bffam\twelvebf}%
   \def\mib{%
      \twelvemibfonts\tenmibfonts
      \textfont0=\twelvebf\scriptfont0=\ninebf
      \scriptscriptfont0=\sevenbf
      \textfont1=\twelvemib\scriptfont1=\ninei
      \scriptscriptfont1=\seveni
      \textfont2=\twelvebsy\scriptfont2=\ninesy
      \scriptscriptfont2=\sevensy}%
   \def\scr{\scrfonts\fam\scrfam\twelvescr
      \global\textfont\scrfam=\twelvescr\global\scriptfont\scrfam=\ninescr
      \global\scriptscriptfont\scrfam=\sevenscr}%
   \tt\ttglue=.5emplus.25emminus.15em
   \normalbaselineskip=14pt
   \setbox\strutbox=\hbox{\vrule height 10pt depth 4pt width 0pt}%
   \normalbaselines\rm\singlespaced}%
%%> {14pt,}
\def\fourteenpoint{\fourteenfonts\twelvefonts
   \def\rm{\fam0\fourteenrm}%
   \textfont0=\fourteenrm\scriptfont0=\twelverm\scriptscriptfont0=\tenrm
   \textfont1=\fourteeni\scriptfont1=\twelvei\scriptscriptfont1=\teni
   \textfont2=\fourteensy\scriptfont2=\twelvesy\scriptscriptfont2=\tensy
   \textfont3=\fourteenex\scriptfont3=\fourteenex
      \scriptscriptfont3=\fourteenex
   \textfont\itfam=\fourteenit\def\it{\fam\itfam\fourteenit}%
   \textfont\slfam=\fourteensl\def\sl{\fam\slfam\fourteensl}%
   \textfont\bffam=\fourteenbf
   \scriptfont\bffam=\twelvebf
   \scriptscriptfont\bffam=\tenbf\def\bf{\fam\bffam\fourteenbf}%
   \def\mib{%
      \fourteenmibfonts\twelvemibfonts\tenmibfonts
      \textfont0=\fourteenbf\scriptfont0=\twelvebf
        \scriptscriptfont0=\tenbf
      \textfont1=\fourteenmib\scriptfont1=\twelvemib
        \scriptscriptfont1=\tenmib
      \textfont2=\fourteenbsy\scriptfont2=\twelvebsy
        \scriptscriptfont2=\tenbsy}%
   \def\scr{\scrfonts\fam\scrfam\fourteenscr
      \global\textfont\scrfam=\fourteenscr\global\scriptfont\scrfam=\twelvescr
      \global\scriptscriptfont\scrfam=\tenscr}%
   \normalbaselineskip=17pt
   \setbox\strutbox=\hbox{\vrule height 12pt depth 5pt width 0pt}%
   \normalbaselines\rm\singlespaced}%
%%> {16pt,}
\def\sixteenpoint{\sixteenfonts\fourteenfonts\twelvefonts
   \def\rm{\fam0\sixteenrm}%
   \textfont0=\sixteenrm\scriptfont0=\fourteenrm\scriptscriptfont0=\twelverm
   \textfont1=\sixteeni\scriptfont1=\fourteeni\scriptscriptfont1=\twelvei
   \textfont2=\sixteensy\scriptfont2=\fourteensy\scriptscriptfont2=\twelvesy
   \textfont3=\sixteenex\scriptfont3=\sixteenex\scriptscriptfont3=\sixteenex
   \textfont\itfam=\sixteenit\def\it{\fam\itfam\sixteenit}%
   \textfont\slfam=\sixteensl\def\sl{\fam\slfam\sixteensl}%
   \textfont\bffam=\sixteenbf
   \scriptfont\bffam=\fourteenbf
   \scriptscriptfont\bffam=\twelvebf\def\bf{\fam\bffam\sixteenbf}%
   \def\mib{%
      \sixteenmibfonts\fourteenmibfonts\twelvemibfonts
      \textfont0=\sixteenbf\scriptfont0=\fourteenbf
        \scriptscriptfont0=\twelvebf
      \textfont1=\sixteenmib\scriptfont1=\fourteenmib
        \scriptscriptfont1=\twelvemib
      \textfont2=\sixteenbsy\scriptfont2=\fourteenbsy
         \scriptscriptfont2=\twelvebsy}%
   \def\scr{\scrfonts\fam\scrfam\sixteenscr
      \global\textfont\scrfam=\sixteenscr\global\scriptfont\scrfam=\fourteenscr
      \global\scriptscriptfont\scrfam=\twelvescr}%
   \normalbaselineskip=20pt
   \setbox\strutbox=\hbox{\vrule height 14pt depth 6pt width 0pt}%
   \normalbaselines\rm\singlespaced}%
%%> {20pt,}
\def\twentypoint{\twentyfonts\sixteenfonts\fourteenfonts
   \def\rm{\fam0\twentyrm}%
   \textfont0=\twentyrm\scriptfont0=\sixteenrm\scriptscriptfont0=\fourteenrm
   \textfont1=\twentyi\scriptfont1=\sixteeni\scriptscriptfont1=\fourteeni
   \textfont2=\twentysy\scriptfont2=\sixteensy\scriptscriptfont2=\fourteensy
   \textfont3=\twentyex\scriptfont3=\twentyex\scriptscriptfont3=\twentyex
   \textfont\itfam=\twentyit\def\it{\fam\itfam\twentyit}%
   \textfont\slfam=\twentysl\def\sl{\fam\slfam\twentysl}%
   \textfont\bffam=\twentybf
   \scriptfont\bffam=\sixteenbf
   \scriptscriptfont\bffam=\fourteenbf\def\bf{\fam\bffam\twentybf}%
   \def\mib{%
      \twentymibfonts\sixteenmibfonts\fourteenmibfonts
      \textfont0=\twentybf\scriptfont0=\sixteenbf
      \scriptscriptfont0=\fourteenbf
      \textfont1=\twentymib\scriptfont1=\sixteenmib
      \scriptscriptfont1=\fourteenmib
      \textfont2=\twentybsy\scriptfont2=\sixteenbsy
      \scriptscriptfont2=\fourteenbsy}%
   \def\scr{\scrfonts
      \global\textfont\scrfam=\twentyscr\fam\scrfam\twentyscr}%
   \normalbaselineskip=24pt
   \setbox\strutbox=\hbox{\vrule height 17pt depth 7pt width 0pt}%
   \normalbaselines\rm\singlespaced}%
%%> {24pt.}
\def\twentyfourpoint{\twentyfourfonts\twentyfonts\sixteenfonts
   \def\rm{\fam0\twentyfourrm}%
   \textfont0=\twentyfourrm\scriptfont0=\twentyrm\scriptscriptfont0=\sixteenrm
   \textfont1=\twentyfouri\scriptfont1=\twentyi\scriptscriptfont1=\sixteeni
   \textfont2=\twentyfoursy\scriptfont2=\twentysy\scriptscriptfont2=\sixteensy
   \textfont3=\twentyfourex\scriptfont3=\twentyfourex
      \scriptscriptfont3=\twentyfourex
   \textfont\itfam=\twentyfourit\def\it{\fam\itfam\twentyfourit}%
   \textfont\slfam=\twentyfoursl\def\sl{\fam\slfam\twentyfoursl}%
   \textfont\bffam=\twentyfourbf
   \scriptfont\bffam=\twentybf
   \scriptscriptfont\bffam=\sixteenbf\def\bf{\fam\bffam\twentyfourbf}%
   \def\mib{%
      \twentyfourmibfonts\twentymibfonts\sixteenmibfonts
      \textfont0=\twentyfourbf\scriptfont0=\twentybf
      \scriptscriptfont0=\sixteenbf
      \textfont1=\twentyfourmib\scriptfont1=\twentymib
      \scriptscriptfont1=\sixteenmib
      \textfont2=\twentyfourbsy\scriptfont2=\twentybsy
      \scriptscriptfont2=\sixteenbsy}%
   \def\scr{\scrfonts
      \global\textfont\scrfam=\twentyfourscr\fam\scrfam\twentyfourscr}%
   \normalbaselineskip=28pt
   \setbox\strutbox=\hbox{\vrule height 19pt depth 9pt width 0pt}%
   \normalbaselines\rm\singlespaced}%
\def\Tbf{\fourteenpoint\bf}
\def\tbf{\twelvepoint\bf}
\def\printfont{\autoload\printfont{printfont.txs}\printfont}

% file: TXSmacs.tex  (TeXsis version 2.17)
%%> {TeXsis main macros.}
\catcode`@=11
\let\XA=\expandafter
\let\NX=\noexpand
\def\dospecials{\do\ \do\\\do\{\do\}\do\$\do\&\do\"\do\(\do\)\do\[\do\]%
  \do\#\do\^\do\^^K\do\_\do\^^A\do\%\do\~}
\def\emsg#1{\relax
   \begingroup
     \def\@quote{"}%
     \def\TeX{TeX}\def\label##1{}\def\use{\string\use}%
     \def\ { }\def~{ }%
     \def\tt{}\def\bf{}\def\Tbf{}\def\tbf{}%
     \def\break{}\def\n{}\def\singlespaced{}\def\doublespaced{}%
     \immediate\write16{#1}%
   \endgroup}
\newif\ifmarkerrors     \markerrorsfalse
\def\@errmark#1{\ifmarkerrors
   \vadjust{\vbox to 0pt{%
   \kern-\baselineskip
   \line{\hfil\rlap{{\tt\ <-#1}}}%
   \vss}}\fi}%
\def\draft{%
   \def\draftline{{\tentt\DraftName\hfill
       -~\folio~- \hfill \runtime}}\footline={\draftline}%
   \eqnotracetrue
   \markerrorstrue
   \overfullrule=1em}%
\def\DraftName{Preliminary Draft}%
\def\setTableskip{\relax}%
\def\singlespaced{%
   \baselineskip=\normalbaselineskip
   \setRuledStrut
   \setTableskip}%
\def\singlespace{\singlespaced}%
\def\doublespaced{%
   \baselineskip=\normalbaselineskip
   \multiply\baselineskip by 150
   \divide\baselineskip by 100
   \setRuledStrut
   \setTableskip}%
\def\doublespace{\doublespaced}%
\def\TrueDoubleSpacing{%
   \baselineskip=\normalbaselineskip
   \multiply\baselineskip by 2
   \setRuledStrut
   \setTableskip}%
\def\truedoublespacing{\TrueDoubleSpacing}%
\def\triplespaced{%
   \baselineskip=\normalbaselineskip
   \multiply\baselineskip by 3
   \setRuledStrut
   \setTableskip}%
\def\widenspacing{%
   \multiply\baselineskip by 125
   \divide\baselineskip by 100
   \setRuledStrut
   \setTableskip}%
\def\whitespaced{\widenspacing}%
\def\whitespace{\widenspacing}%
\def\Footnote#1{%
   \let\@sf\empty
   \ifhmode\edef\@sf{\spacefactor\the\spacefactor}\/\fi
   ${}^{\scriptstyle\smash{#1}}$\@sf
   \Vfootnote{#1}}%
\def\Vfootnote#1{%
   \begingroup
     \def\@foot{\strut\egroup\endgroup}%
     \tenpoint
     \baselineskip=\normalbaselineskip
     \parskip=0pt
     \FootFont
     \vfootnote{${}^{\hbox{#1}}$}}%
\def\FootFont{\rm}%
\newcount\footnum \footnum=0
\let\footnotemark=\empty
\def\NFootnote{%
  \advance\footnum by 1
  \xdef\lab@l{\the\footnum}%
  \Footnote{\footnotemark\the\footnum}}
\def~{\ifmmode\phantom{0}\else\penalty10000\ \fi}%
\def\0{\phantom{0}}%
\def\,{\relax\ifmmode\mskip\the\thinmuskip\else\thinspace\fi}
\def\topspace{\hrule width \z@ height \z@ \vskip}
\def\n{\hfil\break}%
\def\nl{\hfil\break}%
\def\newpage{\vfill\eject}%
\def\Afour{\hsize=16.5cm \vsize=24.5cm\relax}%
\def\bye{%
    \endmode
    \par\vfill\supereject
    \checktags
    \end}
\def\endmode{\relax}%
\def\unobeylines{\catcode`\^^M=5}%
{\obeyspaces\gdef\unobeyspaces{\catcode`\ =10}}
\def\unraggedright{\rightskip=\z@\spaceskip=0pt\xspaceskip=0pt}
{\catcode`\^^M=\active\gdef\seeCR{\catcode`\^^M\active \let^^M\space}}
\catcode`\"=\active
\newcount\@quoteflag   \@quoteflag=\z@
\def"{\@quote}%
\def\@quote{%
   \ifnum\@quoteflag=\z@
     \@quoteflag=\@ne {``}%
   \else
     \@quoteflag=\z@ {''}%
   \fi}
\def\quoteon{\catcode`\"=\active\def"{\@quote}}%
\def\quoteoff{\catcode`\"=12}%
\def\@checkquote#1{\ifnum\@quoteflag=\@ne\message{#1}\fi}
\quoteoff
\def\checkquote{{\quoteoff\@checkquote{> Unbalanced "}}}%
\def\tightbox#1{\vbox{\hrule\hbox{\vrule\vbox{#1}\vrule}\hrule}}
\let\tightboxit=\tightbox
\def\loosebox#1{%
    \vbox{\vskip\jot
        \hbox{\hskip\jot #1\hskip\jot}%
        \vskip\jot}}
\def\eqnbox#1{\lower\jot\tightbox{\loosebox{\quad $#1$ \quad}}}
\def\undertext#1{\setbox0=\hbox{#1}\dimen0=\dp0
      \vtop{\box0 \vskip-\dimen0 \vskip 0.25ex \hrule}}
\def\theBlank#1{\nobreak\hbox{\lower\jot\vbox{\hrule width #1\relax}}}
\ifx\setRuledStrut\undefined\def\setRuledStrut{\relax}\fi
\def\Romannumeral#1{\uppercase\expandafter{\romannumeral #1}}
\def\monthname#1{\ifcase#1 \errmessage{0 is not a month}
    \or January\or February\or March\or April\or May\or June\or 
    July\or August\or September\or October\or November\or
    December\else \errmessage{#1 is not a month}\fi}
\def\jtem#1#2{\par\hangafter0\hangindent#1
              \noindent\llap{#2\enspace}\ignorespaces}
\def\leftpar#1{%
    \setbox\@capbox=\vbox{\normalbaselines
    \noindent #1\par
        \global\@caplines=\prevgraf}%
    \ifnum \@ne=\@caplines
        \leftline{#1}\else
        \hbox to\hsize{\hss\box\@capbox\hss}\fi}
\def\obsolete#1#2{\def#1{\@obsolete#1#2}}
\def\@obsolete#1#2{%
   \emsg{>=========================================================}%
   \emsg{> \string#1 is now obsolete! It may soon disappear!}%
   \emsg{> Please use \string#2 instead.  But I'll try to do it anyway...}%
   \emsg{>=========================================================}%
   \let#1=#2\relax
   #2}%
\def\ATlock{\catcode`@=12\relax}%
\def\ATunlock{\catcode`@=11\relax}%
\newhelp\AThelp{@: 
You've apparantly tried to use a macro which begins with ``@''.^^M
These macros are usually for internal TeXsis functions and should^^M
not be used casually.  If you really want to use the macro try first^^M
saying \string\ATunlock.  If you got this message by pure accident^^M
then something else is wrong.} 
\def\@{\begingroup
    \errhelp=\AThelp
    \newlinechar=`\^^M
    \errmessage{Are you tring to use an internal @-macro?}\relax
   \endgroup}
\long\def\comment#1/*#2*/{\relax}%
\long\def\Ignore#1\endIgnore{\relax}%
\def\endIgnore{\relax}%
{\catcode`\%=11 \gdef\@comment{% }}
\def\REV{\begingroup
   \def\endcomment{\endgroup}%
   \catcode`\|=12
   \catcode`(=12 \catcode`)=12
   \catcode`[=12 \catcode`]=12
   \comment}%
\def\begin#1{%
   \begingroup
     \let\end=\endbegin
     \expandafter\ifx\csname #1\endcsname\relax\relax
        \def\next{\beginerror{#1}}%
     \else
        \def\next{\csname #1\endcsname}%
     \fi\next}
\def\endbegin#1{%
   \endgroup
   \expandafter\ifx\csname end#1\endcsname\relax\relax
      \def\next{\begingroup\beginerror{end#1}}%
   \else
      \def\next{\csname end#1\endcsname}%
   \fi\next}
\newhelp\beginhelp{begin: 
    The \string\begin\space or \string\end\space marked above is for a
    non-existant^^M
    environment.  Check for spelling errors and such.}
\def\beginerror#1{%
   \endgroup
   \errhelp=\beginhelp
   \newlinechar=`\^^M
   \errmessage{Undefined environment for \string\begin\space or \string\end}}
\begingroup\seeCR%
\long\gdef\unexpandedwrite#1#2{\@CopyLine#1#2
\endlist}%
\long\gdef\@CopyLine#1#2
#3\endlist{\@unexpandedwrite#1{#2}%
\def\@arg{#3}\ifx\@arg\par\let\@arg=\empty\fi% ignore trailing ^^M
\ifx\@arg\empty\relax\let\@@next=\relax%
\else\def\@@next{\@CopyLine#1#3\endlist}%
\fi\@@next}%
\long\gdef\writeNX#1#2{\@CopyLineNX#1#2
\endlist}%
\long\gdef\@CopyLineNX#1#2
#3\endlist{\@writeNX#1{#2}%
\def\@arg{#3}\ifx\@arg\par\let\@arg=\empty\fi% ignore trailing ^^M
\ifx\@arg\empty\relax\let\@@next=\relax%
\else\def\@@next{\@CopyLineNX#1#3\endlist}%
\fi\@@next}%
\endgroup
\long\def\@unexpandedwrite#1#2{%
   \def\@finwrite{\immediate\write#1}%
   \begingroup
    \aftergroup\@finwrite
    \aftergroup{\relax
    \@NXstack#2\endNXstack
    \aftergroup}\relax
   \endgroup
 }
\long\def\@writeNX#1#2{%
   \def\@finwrite{\write#1}%
   \begingroup
    \aftergroup\@finwrite
    \aftergroup{\relax
    \@NXstack#2\endNXstack
    \aftergroup}\relax
   \endgroup}%
\def\@NXstack{\futurelet\next\@NXswitch} 
\def\\{\global\let\@stoken= }\\ 
\def\@NXswitch{%
    \ifx\next\endNXstack\relax
    \else\ifcat\noexpand\next\@stoken
        \aftergroup\space\let\next=\@eat
    \else\ifcat\noexpand\next\bgroup
        \aftergroup{\let\next=\@eat
    \else\ifcat\noexpand\next\egroup
        \aftergroup}\let\next=\@eat
     \else
        \let\next=\@copytoken
     \fi\fi\fi\fi 
     \next}%
\def\@eat{\afterassignment\@NXstack\let\next= } 
\long\def\@copytoken#1{%
    \ifcat\noexpand#1\relax
        \aftergroup\noexpand
    \else\ifcat\noexpand#1\noexpand~\relax
        \aftergroup\noexpand
    \fi\fi
    \aftergroup#1\relax
    \@NXstack}%
\def\endNXstack\endNXstack{}%

%%> {Checkpoint/Restart.}
\newwrite\checkpointout
\def\checkpoint#1{\emsg{\@comment\NX\checkpoint --> #1.chk}%
    \immediate\openout\checkpointout= #1.chk
    \@checkwrite{\pageno}   \@checkwrite{\chapternum}%
    \@checkwrite{\eqnum}    \@checkwrite{\corollarynum}%
    \@checkwrite{\fignum}   \@checkwrite{\definitionnum}%
    \@checkwrite{\lemmanum} \@checkwrite{\sectionnum}%
    \@checkwrite{\refnum}   \@checkwrite{\subsectionnum}%
    \@checkwrite{\tabnum}   \@checkwrite{\theoremnum}%
    \@checkwrite{\footnum}%
    \immediate\closeout\checkpointout}%
\def\@checkwrite#1{\edef\tnum{\the #1}%
     \immediate\write\checkpointout{\NX #1 = \tnum}}%
\def\restart#1{\relax
    \immediate\closeout\checkpointout
    \ATunlock
    \Input #1.chk \relax
    \@firstrefnum=\refnum
    \advance\@firstrefnum by \@ne
    \ATlock}%
\let\restore=\restart
\def\endstat{%
   \emsg{\@comment Last PAGE      number is \the\pageno.}%
   \emsg{\@comment Last CHAPTER   number is \the\chapternum.}%
   \emsg{\@comment Last EQUATION  number is \the\eqnum.}%
   \emsg{\@comment Last FIGURE    number is \the\fignum.}%
   \emsg{\@comment Last REFERENCE number is \the\refnum.}%
   \emsg{\@comment Last SECTION   number is \the\sectionnum.}%
   \emsg{\@comment Last TABLE     number is \the\tabnum.}%
   \tracingstats=1}%
\def\gloop#1\repeat{\gdef\body{#1}\iterate}%
\newif\iflastarg\lastargfalse
\def\car#1,#2;{\gdef\@arg{#1}\gdef\@args{#2}}
\def\@apply{%
    \iflastarg
    \else
        \XA\car\@args;
        \islastarg
        \XA\@fcn\XA{\@arg}%
        \@apply
    \fi}
\def\apply#1#2{%
    \gdef\@args{#2,}\let\@fcn#1
    \islastarg
    \@apply
    }
\def\islastarg{\ifx \@args\empty\lastargtrue\else\lastargfalse\fi}%
\def\setcnt#1#2{%
  \edef\th@value{\the#1}%
  \aftergroup\global\aftergroup#1
  \aftergroup=\relax
  \XA\@ftergroup\th@value\endafter
  \global#1=#2\relax}%
\def\@ftergroup{\futurelet\next\@ftertoken} 
\long\def\@ftertoken#1{
   \ifx\next\endafter\relax
     \let\next=\relax
   \else\aftergroup#1\relax
     \let\next=\@ftergroup
   \fi\next}%
\let\DUMP=\dump
\def\@seppuku{\errmessage{Interwoven alignment preambles are not allowed.}\end}

% file: TXSinit.tex  (TeXsis version 2.17)
%%> {Initialization.}
\catcode`@=11
\long\def\texsis{%
    \quoteon
    \Contentsfalse
    \autoparens
    \ATlock
    \resetcounters
    \pageno=1
    \colwidth=\hsize
    \headline={\HeadLine}\headlineoffset=0.5cm
    \footline={\FootLine}\footlineoffset=0.5cm
    \twelvepoint
    \doublespaced
    \newlinechar=`\^^M
    \uchyph=\@ne
    \brokenpenalty=\@M
    \widowpenalty=\@M
    \clubpenalty=\@M
}
\obsolete\inittexsis\texsis     \obsolete\texsisinit\texsis    
\obsolete\initexsis\texsis      \obsolete\initTeXsis\texsis    
\def\LaTeXwarning{\emsg{> }%
   \emsg{> Whoops! This seems to be a LaTeX file.}%
   \emsg{> Try saying `latex \jobname` instead.}%
   \emsg{> }\end}
\def\documentstyle{\LaTeXwarning}
\def\@writefile{\LaTeXwarning}
\def\today{\number\day\ 
    \ifcase\month\or 
    January\or February\or March\or April\or May\or June\or
    July\or August\or September\or October\or November\or December\fi\
    \number\year}
\let\@today=\today
\def\dated#1{\xdef\today{#1}}
\def\SetDate{%
  \xdef\adate{\monthname{\the\month}~\number\day, \number\year}%
  \xdef\edate{\number\day~\monthname{\the\month}~\number\year}%
  \count255=\time\divide\count255 by 60
  \edef\hour{\the\count255}%
  \multiply\count255 by -60 \advance\count255 by\time
  \edef\minutes{\ifnum 10>\count255 {0}\fi\the\count255}%
  \edef\runtime{\the\year/\the\month/\the\day\space\hour:\minutes}}
\def\gzero#1{\ifx#1\undefined\relax\else\global#1=\z@\fi}
\def\resetcounters{%
  \gzero\chapternum     \gzero\sectionnum       \gzero\subsectionnum  
  \gzero\theoremnum     \gzero\lemmanum         \gzero\subsubsectionnum 
  \gzero\tabnum         \gzero\fignum           \gzero\definitionnum    
  \gzero\@BadRefs       \gzero\@BadTags         \gzero\@quoteflag  
  \gzero\@envDepth      \gzero\enumDepth        \gzero\enumcnt        
  \gzero\refnum         \gzero\eqnum            \gzero\corollarynum   
  \global\@firstrefnum=1\global\@lastrefnum=1                   
}
\def\@FileInit#1=#2[#3]{%
   \immediate\openout#1=#2 \relax
   \immediate\write#1{\@comment #3 for job \jobname\space - created: \runtime}%
   \immediate\write#1{\@comment ====================================}}
\newread\txsfile
\let\patchfile=\txsfile
\def\LoadSiteFile{%
  \immediate\openin\patchfile=TXSsite.tex
  \ifeof\patchfile
     \emsg{> No TXSsite.tex file found.}%
     \immediate\closein\patchfile
  \else
     \emsg{> Trying to read in TXSsite.tex...}%
     \immediate\closein\patchfile
     \input TXSsite.tex \relax
  \fi}
\def\ReadPatches{%
    \immediate\openin\patchfile=\TXSpatches.tex
    \ifeof\patchfile
         \closein\patchfile
    \else\immediate\closein\patchfile
       \input\TXSpatches.tex \relax
    \fi
    \immediate\openin\patchfile=\TXSmods.tex \relax
    \ifeof\patchfile
       \closein\patchfile
    \else\immediate\closein\patchfile
       \input\TXSmods.tex \relax
    \fi}
\newinsert\botins 
\skip\botins=\bigskipamount
\count\botins=1000
\dimen\botins=\maxdimen
\newif\if@bot
\def\topinsert{\@midfalse\p@gefalse\@botfalse\@ins}
\def\pageinsert{\@midfalse\p@getrue\@botfalse\@ins}
\def\midinsert{\@midtrue\p@gefalse\@botfalse\@ins\topspace\bigskipamount}
\def\heavyinsert{\@midtrue\p@gefalse\@bottrue\@ins\topspace\bigskipamount}
\def\bottominsert{\@midfalse\p@gefalse\@bottrue\@ins\topspace\bigskipamount}
\def\endinsert{%
  \egroup
  \if@mid \dimen@\ht\z@ \advance\dimen@\dp\z@ 
    \advance\dimen@12\p@ \advance\dimen@\pagetotal
    \ifdim\dimen@>\pagegoal\@midfalse\p@gefalse\fi\fi
  \if@mid \bigskip\box\z@\bigbreak
  \else\if@bot\@insert\botins \else\@insert\topins \fi
  \fi
  \endgroup}
\def\@insert#1{%
  \insert#1{\penalty100
  \splittopskip\z@skip
  \splitmaxdepth\maxdimen \floatingpenalty\z@
  \ifp@ge \dimen@=\dp\z@
    \vbox to\vsize{\unvbox\z@\kern-\dimen@}%
  \else \box\z@ \nobreak
    \ifx #1\topins \ifp@ge\else\bigbreak\fi\fi
  \fi}}
\def\pagecontents{%
  \ifvoid\topins\else\unvbox\topins
      \vskip\skip\topins\fi
  \dimen@=\dp\@cclv \unvbox\@cclv
  \ifvoid\footins\else
    \vskip\skip\footins
    \footnoterule
    \unvbox\footins\fi
  \ifvoid\botins\else\vskip\skip\botins
        \unvbox\botins\fi
  \ifr@ggedbottom \kern-\dimen@ \vfil \fi}
\def\loadstyle#1#2{%
   \def#1{\@loaderr{#1}}%
   \ATunlock
   \immediate\openin\txsfile=#2
   \ifeof\txsfile
      \emsg{> Trying to load the style file #2...}%
   \fi
   \closein\txsfile
   \input #2 \relax
   \ATlock
   #1}%
\newhelp\@utohelp{%
loadstyle: The macro named above was supposed to be defined^^M
In the style file that was just read, but I couldn't find^^M
the definition in that file.  Maybe you can learn something^^M
from the comments in that style file, or find someone who knows^^M
something about it.}
\def\@loaderr#1{%
   \newlinechar=`\^^M
   \errhelp=\@utohelp
   \errmessage{No definition of \string#1 in the style file.}}
\def\autoload#1#2{%
   \def#1{\loadstyle#1{#2}}}
\autoload\PhysRev{PhysRev.txs}%
\autoload\PhysRevLett{PhysRev.txs}%
\autoload\PhysRevManuscript{PhysRev.txs}%
\autoload\nuclproc{nuclproc.txs}%
\autoload\NorthHolland{Elsevier.txs}%
\autoload\NorthHollandTwo{Elsevier.txs}%
\autoload\WorldScientific{WorldSci.txs}%
\autoload\IEEEproceedings{IEEE.txs}%
\autoload\IEEEreduced{IEEE.txs}%
\autoload\AIPproceedings{AIP.txs}%
\autoload\CVformat{CVformat.txs}%
\autoload\idx{index.tex}\autoload\index{index.tex}\autoload\theindex{index.tex}
\autoload\markindexfalse{index.tex}\autoload\markindextrue{index.tex}
\autoload\makeindexfalse{index.tex}\autoload\makeindextrue{index.tex}
\autoload\spine{spine.txs}

% file: TXShead.tex (TeXsis version 2.17)
%%> {Running Headlines.}
\newdimen\headlineoffset        \headlineoffset=0.0cm
\newdimen\footlineoffset        \footlineoffset=0.0cm
\newif\ifRunningHeads           \RunningHeadsfalse
\newif\ifbookpagenumbers        \bookpagenumbersfalse
\newif\ifrightn@m               \rightn@mtrue
\def\makeheadline{\vbox to 0pt{\vskip-22.5pt
   \vskip-\headlineoffset
   \line{\vbox to 8.5pt{}\the\headline}\vss}\nointerlineskip}
\def\makefootline{\baselineskip=24pt
   \vskip\footlineoffset
   \line{\the\footline}}
\def\HeadLine{%
   \edef\firstm{{\XA\iffalse\firstmark\fi}}%
   \edef\topm{{\XA\iffalse\topmark\fi}}%
   \ifRunningHeads
     \def\He@dText{{\HeadFont \HeadText}}%
   \else\def\He@dText{\relax}\fi
   \ifbookpagenumbers
      \ifodd\pageno\rightn@mtrue
      \else\rightn@mfalse\fi
   \else\rightn@mtrue\fi
   \tenrm
   \ifx\topm\firstm
     \ifrightn@m
        {\hss\He@dText\hss\llap{\rm\PageNumber}}%
     \else
        {\rlap{\rm\PageNumber}\hss\He@dText\hss}%
      \fi 
   \else \hfill \fi}%
\def\HeadText{\hfill}
\def\FootLine{%
   \edef\firstm{%
      {\expandafter\iffalse\firstmark\fi}}%
   \edef\topm{%
      {\expandafter\iffalse\topmark\fi}}%
   \ifx\topm\firstm \hss
    \else {\hss\HeadFont \FootText \hss} \fi}%
\def\FootText{\hfill}%
\def\HeadFont{\tenit}%
\begingroup
  \catcode`<=12 \catcode`>=12 \catcode`\"=12 
  \gdef\PageLinkto#1{%
        \html{<a href="\hash sect.TOC">}%
        \html{<a NAME="page.\the\pageno">}%
        {#1}\html{</a>}%
        \html{</a>}%
   }%
\endgroup
\def\PageNumber{\PageLinkto{\folio}}%
\def\nopagenumbers{\headline={\hfil}\footline={\hfil}}%
\def\pagenumbers{\headline={\HeadLine}\footline={\FootLine}}
\def\bottompagenumbers{\footline={\hfill{\rm\PageNumber}\hfill}%
                \headline={\hfill}}
\def\bookpagenumbers{\bookpagenumberstrue}
\def\plainoutput{%
  \makeBindingMargin
  \shipout\vbox{\makeheadline\pagebody\makefootline}%
  \advancepageno
  \ifnum\outputpenalty>-\@MM \else\dosupereject\fi}
\newdimen\BindingMargin \BindingMargin=0pt
\def\makeBindingMargin{%
   \ifdim\BindingMargin>0pt
   \ifodd\pageno\hoffset=\BindingMargin\else
   \hoffset=-\BindingMargin\fi\fi}

% file: TXSeqns.tex  (TeXsis version 2.17)
%%> {Equation numbering.}
\newcount\eqnum         \eqnum=\z@
\def\@chaptID{}         \def\@sectID{}%
\newif\ifeqnotrace      \eqnotracefalse
\def\EQN{%
   \begingroup
   \quoteoff\offparens
   \@EQN}%
\def\@EQN#1$${%
   \endgroup
   \if ?#1? \EQNOparse *;;\endlist
   \else \EQNOparse#1;;\endlist\fi
   $$}%
\def\EQNOparse#1;#2;#3\endlist{%
  \if ?#3?\relax
    \global\advance\eqnum by\@ne
    \edef\tnum{\@chaptID\@sectID\the\eqnum}%
    \Eqtag{#1}{\tnum}%
    \@EQNOdisplay{#1}%
  \else\stripblanks #2\endlist
    \edef\p@rt{\tok}%
    \if a\p@rt\relax
      \global\advance\eqnum by\@ne\fi
    \edef\tnum{\@chaptID\@sectID\the\eqnum}%
    \Eqtag{#1}{\tnum}%
    \edef\tnum{\@chaptID\@sectID\the\eqnum\p@rt}%
    \Eqtag{#1;\p@rt}{\tnum}%
    \@EQNOdisplay{#1;#2}%
  \fi
  \global\let\?=\tnum
  \relax}%
\def\Eqtag#1#2{\tag{Eq.#1}{#2}}
\def\@EQNOdisplay#1{%
   \@eqno
   \ifeqnotrace
     \rlap{\phantom{(\tnum)}%
        \quad{\tenpoint\tt["#1"]}}\fi
     \linkname{Eq.#1}{(\tnum)}%
   }
\let\@eqno=\eqno
\def\endlist{\endlist}%
\def\Eq#1{\linkto{Eq.#1}{Eq.~($\use{Eq.#1}$)}}%
\def\Eqs#1{\linkto{Eq.#1}{Eqs.~($\use{Eq.#1}$)}}%
\def\Ep#1{\linkto{Eq.#1}{($\use{Eq.#1}$)}}%
\def\EQNdisplaylines#1{%
   \@EQNcr
   \displ@y
   \halign{%
      \hbox to\displaywidth{%
      $\@lign\hfil\displaystyle##\hfil$}%
      &\llap{$\@lign\@@EQN{##}$}\crcr
   #1\crcr}%
   \@EQNuncr}%
\long\def\EQNalign#1{%
   \@EQNcr
   \displ@y
     \tabskip=\centering
   \halign to\displaywidth{%
   \hfil$\relax\displaystyle{##}$
     \tabskip=0pt
   &$\relax\displaystyle{{}##}$\hfil
     \tabskip=\centering
   &\llap{$\relax\@@EQN{##}$}%
     \tabskip=0pt\crcr
    #1\crcr}%
   }
\def\@@EQN#1{\if ?#1? \EQNOparse *;;\endlist
         \else \EQNOparse#1;;\endlist\fi}%
\def\@EQNcr{%
   \let\EQN=&
   \let\@eqno=\relax}%
\def\@EQNuncr{%
   \let\EQN=\@EQN
   \let\@eqno=\eqno}%
\def\EQNdoublealign#1{%
   \@EQNcr
   \displ@y
   \tabskip=\centering
   \halign to\displaywidth{%
      \hfil$\relax\displaystyle{##}$
      \tabskip=0pt
   &$\relax\displaystyle{{}##}$\hfil
      \tabskip=0pt
   &$\relax\displaystyle{{}##}$\hfil
      \tabskip=\centering
   &\llap{$\relax\@@EQN{##}$}%
      \tabskip=0pt\crcr
   #1\crcr}%
   \@EQNuncr}%
%\def\displaylines#1{%
%   \displ@y                            
%   \halign{\hbox to \displaywidth{$\hfil\displaystyle##\hfil$}%
%           &\llap{$##$}\crcr           
%    #1\crcr}}                          
\def\eqn#1$${\edef\tok\string#1
   \xdef#1{\NX\use{Eq.\tok}}%
   \EQNOparse \tok;;\endlist $$}%
\def\eqnmarker{\triangleright}%
\def\eqnmark{\quoteoff\offparens\@eqnmark}
\def\@eqnmark#1$${\@@eqnmark#1\eqno\eqno\endlist}
\def\@@eqnmark#1\eqno#2\eqno#3\endlist{\def\EQN{\relax}%
   \if ?#3? \@EQNmark#1\EQN\EQN\endlist
   \else\displaylines{\hbox to 0pt{$\eqnmarker$\hss}\hfill{#1}\hfill
                      \hbox to 0pt{\hss$#2$}}\fi$$}
\def\@EQNmark#1\EQN#2\EQN#3\endlist{%
   \if ?#3?\displaylines{\hbox to 0pt{$\eqnmarker$\hss}\hfill{#1}\hfill}%
   \else \let\@eqno=\relax
      \EQNdisplaylines{\hbox to 0pt{$\eqnmarker$\hss}\hfill{#1}\hfill
                \hbox to 0pt{\hss$\EQNOparse#2;;\endlist$}}\fi}

% file: TXSprns.tex (TeXsis version 2.17)
%%> {Auto-sizing of Parentheses.}
\catcode`@=11
\ifx\@left\undefined
 \let\@left=\left       \let\@right=\right
 \let\lparen=(          \let\rparen=)
 \let\lbrack=[          \let\rbrack=]
 \let\@vert=\vert
\fi
\begingroup
\catcode`\(=\active \catcode`\)=\active
\catcode`\[=\active \catcode`\]=\active
\gdef({\relax
   \ifmmode \push@delim{P}%
    \@left\lparen
   \else\lparen
   \fi}
\global\let\@lparen=(
\gdef){\relax
   \ifmmode\@right\rparen
     \pop@delim\@delim
     \if P\@delim \relax \else
       \if B\@delim\emsg{> Expecting \string] but got \string).}%
                   \@errmark{PAREN}%
       \else\emsg{> Unmatched \string).}\@errmark{PAREN}%
     \fi\fi
   \else\rparen
   \fi}
\gdef[{\relax
   \ifmmode \push@delim{B}%
     \@left\lbrack
   \else\lbrack
   \fi}
\global\let\@lbrack=[
\gdef]{\relax
   \ifmmode\@right\rbrack
     \pop@delim\@delim
     \if B\@delim \relax \else
       \if P\@delim\emsg{> Expecting \string) but got \string].}%
                   \@errmark{BRACK}%
       \else\emsg{> Unmatched \string].}\@errmark{BRACK}%
     \fi\fi
   \else\rbrack
   \fi}
\gdef\EZYleft{\futurelet\nexttok\@EZYleft}%
\gdef\@EZYleft#1{%
   \ifx\nexttok(  \let\nexttok=\lparen
   \else
   \ifx\nexttok[  \let\nexttok=\lbrack
   \fi\fi
   \@left\nexttok}%
\gdef\EZYright{\futurelet\nexttok\@EZYright}%
\gdef\@EZYright#1{%
   \ifx\nexttok)  \let\nexttok=\rparen
   \else
   \ifx\nexttok]  \let\nexttok=\rbrack
   \fi\fi
   \@right\nexttok}%
\endgroup
\toksdef\@CAR=0  \toksdef\@CDR=2
\def\push@delim#1{\@CAR={{#1}}%
     \@CDR=\XA{\@delimlist}%
    \edef\@delimlist{\the\@CAR\the\@CDR}}%
\def\pop@delim#1{\XA\pop@delimlist\@delimlist\endlist#1}%
\def\pop@delimlist#1#2\endlist#3{\def\@delimlist{#2}\def#3{#1}}    
\def\@delimlist{}%
\newif\ifEZparens   \EZparensfalse
\def\autoparens{\EZparenstrue
   \everydisplay={\@onParens}%
   }
\def\@onParens{%
   \ifEZparens
    \def\@delimlist{}%
    \let\left=\EZYleft
    \let\right=\EZYright
    \catcode`\(=\active \catcode`\)=\active
    \catcode`\[=\active \catcode`\]=\active
   \fi}
\def\offparens{%
   \EZparensfalse\@offParens
   \everymath={}\everydisplay={}}%
\def\@offParens{%
   \let\left=\@left
   \let\right=\@right
   \catcode`(=12 \catcode`)=12
   \catcode`[=12 \catcode`]=12
   }
\offparens
\def\onparens{%
   \EZparenstrue
   \everymath={\@onMathParens}%
   \everydisplay={\@onParens}%
   }
\def\easyparenson{\onparens}%
\def\@onMathParens#1{%
   \@SetRemainder#1\endlist
   \ifx#1\lparen\let\@remainder=\@lparen\fi
   \ifx#1\lbrack\let\@remainder=\@lbrack\fi
   \@onParens
   \@remainder}%
\def\@SetRemainder#1#2\endlist{%
   \ifx @#2@ \def\@remainder{#1}%
   \else  \def\@remainder{{#1#2}}%
   \fi}
\def\easyparensoff{\offparens}%
\def\pmatrix#1{\@left\lparen\matrix{#1}\@right\rparen}
\def\bordermatrix#1{\begingroup \m@th
  \setbox\z@\vbox{\def\cr{\crcr\noalign{\kern2\p@\global\let\cr\endline}}%
    \ialign{$##$\hfil\kern2\p@\kern\p@renwd&\thinspace\hfil$##$\hfil
      &&\quad\hfil$##$\hfil\crcr
      \omit\strut\hfil\crcr\noalign{\kern-\baselineskip}%
      #1\crcr\omit\strut\cr}}%
  \setbox\tw@\vbox{\unvcopy\z@\global\setbox\@ne\lastbox}%
  \setbox\tw@\hbox{\unhbox\@ne\unskip\global\setbox\@ne\lastbox}%
  \setbox\tw@\hbox{$\kern\wd\@ne\kern-\p@renwd\@left\lparen\kern-\wd\@ne
    \global\setbox\@ne\vbox{\box\@ne\kern2\p@}%
    \vcenter{\kern-\ht\@ne\unvbox\z@\kern-\baselineskip}\,\right\rparen$}%
  \;\vbox{\kern\ht\@ne\box\tw@}\endgroup}
\def\partitionmatrix#1{\,\vcenter{\offinterlineskip\m@th
   \def\tablerule{\noalign{\hrule}}
   \halign{\hfil\loosebox{$\mathstrut ##$}\hfil&&\quad\vrule##\quad&
      \hfil\loosebox{$##$}\hfil\crcr
   #1\crcr}}\,}

% file: TXSrefs.tex (TeXsis version 2.17)
%%> {References and Citations.}
\catcode`@=11
\catcode`\"=12 \catcode`\(=12 \catcode`\)=12
\newcount\refnum        \refnum=\z@
\newcount\@firstrefnum  \@firstrefnum=1
\newcount\@lastrefnum   \@lastrefnum=1
\newcount\@BadRefs      \@BadRefs=0
\newif\ifrefswitch      \refswitchtrue
\newif\ifbreakrefs      \breakrefstrue
\newif\ifrefpunct       \refpuncttrue
\newif\ifmarkit         \markittrue
\newif\ifnullname
\newif\iftagit
\newif\ifreffollows
\def\refterminator{}
\def\RefLabel{}
\newdimen\refindent     \refindent=2em
\newdimen\refpar        \refpar=20pt
\newbox\tempbox
{\catcode`\%=11 \gdef\@comment{% }}
\newcount\CiteType     \CiteType=1
%\newif\ifsuperrefs    \superrefstrue  
\def\superrefstrue{\CiteType=1}%
\def\superrefsfalse{\CiteType=2}%
\def\NamedCitations{\CiteType=3}
\def\FootnoteCitations{\CiteType=4}
\newwrite\reflistout
\newread\reflistin
\def\@refinit{%
  \immediate\closeout\reflistout
  \ifrefswitch
    \@FileInit\reflistout=\jobname.ref[List of References]% yes: initialize
  \else
    \let\@refwrite=\@refwrong \let\@refNXwrite=\@refwrong  
  \fi
  \gdef\refinit{\relax}}%
\def\refReset{%
   \global\refnum=\z@
   \global\@firstrefnum=1
   \global\@lastrefnum=1
   \global\@BadRefs=0
   \gdef\refinit{\@refinit}}%
\refReset
\def\@refwrite#1{\refinit\immediate\write\reflistout{#1}}
\def\@refNXwrite#1{\refinit\unexpandedwrite\reflistout{#1}} 
\def\@refwrong#1{}%
\long\def\reference#1{%
  \markittrue
  \@tagref{#1}%
  \@GetRefText{#1}}%
\long\def\addreference#1{%
  \markitfalse
  \@tagref{#1}%
  \@GetRefText{#1}}%
\def\hiddenreference{\addreference}%
\def\@tagref#1{%
  \stripblanks #1\endlist
  \XA\ifstar\tok*\relax
  \ifnullname\relax\else
    \def\RefLabel{#1}%
    \global\advance\refnum by \@ne
    \@lastrefnum=\refnum
    \edef\rnum{\the\refnum}%
    \tag{Ref.#1}{\rnum}%
    \ifnum\CiteType>0
       \immediate\write16{(\the\refnum)% write comment to .LIS and terminal
          First reference to "#1" on page \the\pageno.}\fi
  \fi}%
\def\ifstar#1#2\relax{\ifx*#1\relax\nullnametrue\else\nullnamefalse\fi}
\def\@GetRefText#1{%
  \ifnum\CiteType<3
    \ifnullname
      \p@nctwrite;\relax
      \@refwrite{\@comment ... Reference text for%
      "#1" defined on page \number\pageno.}%
    \else
      \ifnum\refnum>1\p@nctwrite.\fi
      \@refwrite{\@comment }%
      \@refwrite{\@comment (\the\refnum) Reference text for%
                "#1" defined on page \number\pageno.}%
      \@refwrite{\string\@refitem{\the\refnum}{#1}}%
  \fi\fi
  \begingroup
    \def\endreference{\NX\endreference}%
    \def\reference{\NX\reference}\def\ref{\NX\ref}%
    \seeCR\newlinechar=`\^^M
    \@copyref}%
\def\@copyref#1#2\endreference{%
  \endgroup
  \ifnum\CiteType=4
    \ifx#1\par\def\arg{#2}\else\def\arg{#1#2}\fi
    \Vfootnote{\the\refnum}%          
        {\hangindent=\parindent\hangafter=1\seeCR\arg}%
  \else
    \ifx#1\par\@refNXwrite{#2\@endrefitem}%
    \else\@refNXwrite{#1#2\@endrefitem}\fi
  \fi
  \@endreference}%
\def\@endrefitem#1{#1}%
\long\def\@endreference#1{%
  \reffollowsfalse
  \ifx#1\cite\reffollowstrue\fi
  \ifx#1\citerange\reffollowstrue\fi
  \ifx#1\refrange\reffollowstrue\fi
  \ifx#1\ref\reffollowstrue\fi
  \ifx#1\reference\reffollowstrue
     \ifnum\CiteType=3
        \xdef\@refmark{\linkto{Ref.\RefLabel}{\RefLabel}}\add@refmark\fi 
     \ifnum\CiteType=6
        \xdef\@refmark{\linkto{Ref.\RefLabel}{\RefLabel}}\add@refmark\fi
  \else
     \ifnum\@firstrefnum>\@lastrefnum\relax
     \else
       \ifnum\CiteType=3
          \xdef\@refmark{\linkto{Ref.\RefLabel}{\RefLabel}}%
       \else\ifnum\CiteType=6
          \xdef\@refmark{\linkto{Ref.\RefLabel}{\RefLabel}}%
       \else
         \ifnum\@firstrefnum=\@lastrefnum
           \xdef\@refmark{\linkto{Ref.\the\@lastrefnum}{\the\@lastrefnum}}%
         \else
            \xdef\@refmark{\linkto{Ref.\the\@firstrefnum}{\the\@firstrefnum}-
                        \linkto{Ref.\the\@lastrefnum}{\the\@lastrefnum}}%
         \fi
       \fi\fi
       \global\@firstrefnum=\refnum
       \global\advance\@firstrefnum by \@ne
       \add@refmark
     \fi
  \fi
  \flush@reflist{#1}}%
\def\endreference{%
  \emsg{>  Whoops! \string\endreference was called without
                first calling \string\reference.}\@errmark{REF?}%
  \emsg{>  I'll just ignore it.}%
  }%
\def\@refspace{\ }
\def\citemark#1{%
   \relax\let\@sf\empty
   \ifhmode\edef\@sf{\spacefactor\the\spacefactor}\/\fi
   \ifcase\CiteType\relax
   \or $\relax{}^{\hbox{$\citestyle
           #1\refterminator$}}$\relax
   \or {}~[{#1}]\relax
   \or {}~[{#1}]\relax
   \or $\relax{}^{\hbox{$\citestyle
          #1\refterminator$}}$\relax
   \or {}~({#1})\relax
   \or {}~({#1})\relax
   \else\relax\fi
   \@sf}%
\def\citestyle{\scriptstyle}%
\def\referencelist{%
   \ifnum\CiteType=4
        \emsg{> Warning: \string\referencelist is not compatible with%
                footnoted reference citations.}\fi
   \begingroup
       \pageno=0\CiteType=0}%
\def\endreferencelist{%
   \endgroup}%
\long\def\cite#1#2{%
  \def\RefLabel{#1}%
  \markittrue
  \reffollowsfalse
  \ifx#2\cite\reffollowstrue\fi
  \ifx#2\citerange\reffollowstrue\fi
  \ifx#2\refrange\reffollowstrue\fi
  \ifx#2\ref\reffollowstrue\fi
  \ifx#2\reference\reffollowstrue\fi
  \auxwritenow{\string\citation\string{#1\string}}%
  \make@refmark{#1}%
  \add@refmark
  \flush@reflist{#2}}%
\let\ref=\cite
\def\@refmarklist{}%
\def\nocite#1{%
  \auxwritenow{\string\citation\string{#1\string}}}%
\def\make@refmark#1{%
  \testtag{Ref.#1}\ifundefined
    \emsg{> UNDEFINED REFERENCE #1 ON PAGE \number\pageno.}%
    \global\advance\@BadRefs by 1
    \xdef\@refmark{{\tenbf #1}}%
    \@errmark{REF?}%
  \else
    \ifnum\CiteType=3
      \xdef\@refmark{\linkto{Ref.#1}{#1}}%
    \else
   \xdef\@refmark{\linkto{Ref.\csname\tok\endcsname}{\csname\tok\endcsname}}%
  \fi\fi}%
\def\add@refmark{%
  \ifmarkit
  \ifx\@refmarklist\empty\relax
     \xdef\@refmarklist{\@refmark}%
  \else
    \ifnum\CiteType=3
      \xdef\@refmarklist{\@refmarklist; \@refmark}%
    \else
      \xdef\@refmarklist{\@refmarklist,\@refmark}%
  \fi\fi\fi}%
\long\def\flush@reflist#1{%
  \ifmarkit
  \ifreffollows\else
    \citemark{\@refmarklist}%
    \gdef\@refmarklist{}%
    \ifx#1\par\else\space@head{#1}\fi
  \fi\fi
  \def\@next{#1}\ifcat.\NX#1\def\@next{#1 }\fi
  \@next}%
{\quoteon
\gdef\space@head#1{\def\next{\space}%
    \ifcat.\NX#1\relax\def\next{\relax}\fi
    \ifx)#1\def\next{\relax}\fi
    \ifx]#1\def\next{\relax}\fi
    \ifx"#1\def\next{\relax}\fi
   \next}}%
\def\Ref#1{%
   \ifnum\CiteType=3 \citemark{\linkto{Ref.#1}{\use{Ref.#1}}}%
   \else 
     \testtag{Ref.#1}\ifundefined
       Ref.~\use{Ref.#1}%
     \else 
       \linkto{Ref.\csname\tok\endcsname}{Ref.~\csname\tok\endcsname}%
   \fi\fi}
\long\def\refrange#1#2#3{%
  \ifnum\CiteType=3\emsg{> WARNING: \string\refrange\space%
                doesn't work with named citations.}\@errmark{REF?}\fi 
  \reffollowsfalse
  \ifx#3\cite\reffollowstrue\fi
  \ifx#3\ref\reffollowstrue\fi
  \ifx#3\reference\reffollowstrue\fi
  \ifx#3\refrange\reffollowstrue\fi
  \make@refmark{#2}%
  \xdef\@refmarktwo{\@refmark}%
  \make@refmark{#1}%
  \xdef\@refmark{\@refmark\hbox{--}\@refmarktwo}%
  \add@refmark
  \flush@reflist{#3}}%
\let\citerange=\refrange
\def\vol#1{\undertext{#1}}
\def\booktitle#1{{\sl #1}}
\newif\ifShowArticleTitle  \ShowArticleTitlefalse
\def\ArticleTitle#1{\ifShowArticleTitle{\sl #1},\fi}
\def\etal{{\it et al.}} \def\ie{{\it i.e.}}
\def\cf{{\it cf.}}      \def\ibid{{\it ibid.}}
\def\ListReferences{%
  \ifnum\CiteType=1\@ListReferences\fi
  \ifnum\CiteType=2\@ListReferences\fi}
\def\@ListReferences{\emsg{Reference List}%
  \ifnum\refnum>\z@ \p@nctwrite{.}%
    \@refwrite{\@comment>>> EOF \jobname.ref <<<}% and EOF marker comment
    \immediate\closeout\reflistout
  \fi
  \ifnum\@BadRefs>\z@
    \emsg{>}\emsg{> There were \the\@BadRefs\ undefined references.}%
    \emsg{> See the file \jobname.log for the citations, or try running}%
    \emsg{> TeXsis again to resolve forward references.}\emsg{>}%
  \fi
  \begingroup
    \offparens
    \immediate\openin\reflistin=\jobname.ref
    \ifeof\reflistin
       \closein\reflistin
       \emsg{> \string\ListReferences: no references in \jobname.ref}%
    \else
       \catcode`@=11
       \catcode`\^^M=10
       \setbox\tempbox\hbox{\the\refnum.\quad}%
       \refindent=\wd\tempbox
       \leftskip=\refindent
       \parindent=\z@
       \def\reference{\@noendref}%
       \refFormat
       \Input\jobname.ref  \relax
       \vskip 0pt
    \fi
  \endgroup
  \refReset
  }%
\def\References{\ListReferences}%
\def\refFormat{\relax}%
\def\@noendref#1{%
   \emsg{>  Whoops! \string\reference{#1} was given before the}%
   \emsg{>  \string\endreference for the previous \string\reference.}%
   \emsg{>  I'll just ignore it and run the two together.}%
   }%
\def\@refitem#1#2#3{\message{#1.}%
   \auxwritenow{\string\bibcite\string{#2\string}\string{#1\string}}%
   \refskip\noindent\hskip-\refindent
   \hbox to \refindent {\hss\linkname{Ref.#1}{#1.}\quad}%
   #3}
\def\refskip{\smallskip}%
\def\@refpunct#1{\unskip#1}%
\def\p@nctwrite#1{%
   \ifrefpunct
      \@refwrite{\NX\@refpunct#1\NX\@refbreak}%
   \else
      \@refwrite{\NX\@refbreak}%
   \fi}
\def\@refbreak{\ifbreakrefs\par\fi}
\newif\ifEurostyle     \Eurostylefalse
\offparens
{\catcode`\.=\active \gdef.{\hbox{\p@riod\null}}}%
\def\p@riod{.}%
\def\journal{%
  \bgroup
   \catcode`\.=\active
   \offparens
   \j@urnal}%
 \def\j@urnal#1;#2,#3(#4){%
   \ifEurostyle
      {#1} {\vol{#2}} (\@fullyear{#4}) #3\relax
   \else
      {#1} {\vol{#2}}, #3 (\@fullyear{#4})\relax
   \fi
  \egroup}%
\def\@fullyear#1{%
  \begingroup
   \count255=\year
      \divide \count255 by 100 \multiply \count255 by 100
   \count254=\year
      \advance \count254 by -\count255 \advance \count254 by 1
   \count253=#1\relax
   \ifnum\count253<100
     \ifnum \count253>\count254
       \advance \count253 by -100\fi
      \advance \count253 by \count255
   \fi
   \number\count253
  \endgroup}%
\def\NP{Nucl.\ Phys.}   \def\PL{Phys.\ Lett.}
\def\PR{Phys.\ Rev.}    \def\PRL{Phys.\ Rev.\ Lett.}
\def\ao{Appl.\  Opt.\ }         \def\ap{Appl.\  Phys.\ }
\def\apl{Appl.\ Phys.\ Lett.\ } \def\apj{Astrophys.\ J.\ }
\def\jcp{J.\ Chem.\ Phys.\ }    \def\jmo{J.\ Mod.\ Opt.\ }
\def\josa{J.\ Opt.\ Soc.\ Am.\ }\def\josaa{J.\ Opt.\ Soc.\ Am.\ A }
\def\jpp{J.\ Phys.\ (Paris) }   \def\nat{Nature (London) }
\def\oc{Opt.\ Commun.\ }        \def\ol{Opt.\ Lett.\ }
\def\pl{Phys.\ Lett.\ }         \def\pra{Phys.\ Rev.\ A }
\def\prb{Phys.\ Rev.\ B }       \def\prc{Phys.\ Rev.\ C }
\def\prd{Phys.\ Rev.\ D }       \def\pre{Phys.\ Rev.\ E }
\def\prl{Phys.\ Rev.\ Lett.\ }  \def\rmp{Rev.\ Mod.\ Phys.\ }
\def\bell{Bell Syst.\ Tech.\ J.\ }
\def\jqe{IEEE J.\ Quantum Electron.\ }
\def\assp{IEEE Trans.\ Acoust.\ Speech Signal Process.\ }
\def\aprop{IEEE Trans.\ Antennas Propag.\ }
\def\mtt{IEEE Trans.\ Microwave Theory Tech.\ }
\def\iovs{Invest.\ Ophthalmol.\ Vis.\ Sci.\ }
\def\josab{J.\ Opt.\ Soc.\ Am.\ B }
\def\pspie{Proc.\ Soc.\ Photo-Opt.\ Instrum.\ Eng.\ }
\def\sjqe{Sov.\ J.\ Quantum Electron.\ }
\def\jhep#1,#2(#3){{\rm JHEP\ }{\bf #1}, {\rm#2} {\rm(#3)}}
%%> {BIBTeX support for TeXsis (macros and patches) v 0.92}
\def\citation#1{\relax} \def\bibdata#1{\relax}
\def\bibstyle#1{\relax} \def\bibcite#1#2{\relax}
\def\emdash{--}
\def\ReferenceStyle#1{\auxwritenow{\string\bibstyle\string{#1\string}}}
\let\bibliographystyle=\ReferenceStyle
\def\ReferenceFiles#1{%
    \auxwritenow{\string\bibdata\string{#1\string}}%
    \immediate\openin\reflistin=\jobname.bbl
    \ifeof\reflistin
         \closein\reflistin
    \else\immediate\closein\reflistin
       \input\jobname.bbl \relax
    \fi}
\let\bibliography=\ReferenceFiles

% file: TXSsects.tex  (TeXsis version 2.17)
%%> {Section and Chapter divisions.}
\catcode`@=11
\newcount\chapternum            \chapternum=\z@
\newcount\sectionnum            \sectionnum=\z@
\newcount\subsectionnum         \subsectionnum=\z@
\newcount\subsubsectionnum      \subsubsectionnum=\z@
\newif\ifshowsectID             \showsectIDtrue
\def\@sectID{}%
\newif\ifshowchaptID            \showchaptIDtrue
\def\@chaptID{}%
\newskip\sectionskip            \sectionskip=2\baselineskip
\newskip\subsectionskip         \subsectionskip=1.5\baselineskip
\newdimen\sectionminspace       \sectionminspace = 0.20\vsize
\long\def\chapter#1#2 {%
  \def\@aftersect{#2}%
  \ifx\@aftersect\empty\let\@aftersect=\@eatpar
  \else\def\@aftersect{\@eatpar #2 }\fi
  \vfill\supereject
  \global\advance\chapternum by \@ne
  \global\sectionnum=\z@
  \global\def\@sectID{}%
  \edef\lab@l{\ChapterStyle{\the\chapternum}}%
  \ifshowchaptID
    \global\edef\@chaptID{\lab@l.}%
    \r@set
  \else\edef\@chaptID{}\fi
  \everychapter
  \ifx\Tbf\undefined\def\Tbf{\bf}\fi
  \ifshowchaptID
    \leftline{\Tbf{Chapter\ \@chaptID}}%
    \nobreak\smallskip\fi
  \begingroup
    \raggedright\pretolerance=2000\hyphenpenalty=2000
    \parindent=\z@ {\Tbf{#1}\bigskip}%
  \endgroup
  \nobreak\bigskip
  \begingroup
    \def\label##1{}%
    \xdef\ChapterTitle{#1}%
    \def\n{}\def\nl{}\def\mib{}%
    \setHeadline{#1}%
    \emsg{\@chaptID\space #1}%
    \def\@quote{\string\@quote\relax}%
    \addTOC{0}{\TOCcID{\lab@l.}#1}{\folio}%
  \endgroup
  \@Mark{#1}%
  \s@ction
  \afterchapter\@aftersect}%
\def\everychapter{\relax}%
\def\afterchapter{\relax}%
\def\ChapterStyle#1{#1}%
\def\setChapterID#1{\edef\@chaptID{#1.}}%
\def\r@set{%
  \global\subsectionnum=\z@
  \global\subsubsectionnum=\z@
  \ifx\eqnum\undefined\relax
    \else\global\eqnum=\z@\fi
  \ifx\theoremnum\undefined\relax
  \else
    \global\theoremnum=\z@    \global\lemmanum=\z@                
    \global\corollarynum=\z@  \global\definitionnum=\z@
    \global\fignum=\z@       
    \ifRomanTables\relax     
    \else\global\tabnum=\z@\fi
  \fi}
\long\def\s@ction{%
  \checkquote
  \checkenv
  \vskip -\parskip
  \nobreak\noindent}
\def\@aftersect{}
\def\@Mark#1{%
   \begingroup
     \def\label##1{}%
     \def\goodbreak{}%
     \def\mib{}\def\n{}%
     \mark{#1\NX\else\lab@l}%
   \endgroup}%
\def\@noMark#1{\relax}%
\def\setHeadline#1{\@setHeadline#1\n\endlist}%
\def\@setHeadline#1\n#2\endlist{%
   \def\@arg{#2}\ifx\@arg\empty
      \global\edef\HeadText{#1}%
   \else
      \global\edef\HeadText{#1\dots}%
   \fi
}
\long\def\section#1#2 {%
  \def\@aftersect{#2}%
  \ifx\@aftersect\empty\let\@aftersect=\@eatpar
  \else\def\@aftersect{\@eatpar #2 }\fi
  \vskip\parskip\vskip\sectionskip
  \goodbreak\pagecheck\sectionminspace
  \global\advance\sectionnum by \@ne
  \edef\lab@l{\@chaptID\SectionStyle{\the\sectionnum}}%
  \ifshowsectID
    \global\edef\@sectID{\SectionStyle{\the\sectionnum}.}%
    \global\edef\@fullID{\lab@l.\space\space}%
    \r@set
  \else\gdef\@fullID{}\def\@sectID{}\fi
  \everysection
  \ifx\tbf\undefined\def\tbf{\bf}\fi
  \vbox{%
     \raggedright\pretolerance=2000\hyphenpenalty=2000
     \setbox0=\hbox{\noindent\tbf\@fullID}%
     \hangindent=\wd0 \hangafter=1
     \noindent\unhbox0{\tbf{#1}\medskip}}%
   \nobreak
   \begingroup
     \def\label##1{}%
     \global\edef\SectionTitle{#1}%
     \def\n{}\def\nl{}\def\mib{}%
     \ifnum\chapternum=0\setHeadline{#1}\fi
     \emsg{\@fullID #1}%
     \def\@quote{\string\@quote\relax}%
     \addTOC{1}{\TOCsID{\lab@l.}#1}{\folio}%
   \endgroup
   \s@ction
   \aftersection\@aftersect}%
\def\everysection{\relax}%
\def\aftersection{\relax}%
\def\setSectionID#1{\edef\@sectID{#1.}}%
\def\SectionStyle#1{#1}%
\long\def\subsection#1#2 {%
  \def\@aftersect{#2}%
  \ifx\@aftersect\empty\let\@aftersect=\@eatpar
  \else\def\@aftersect{\@eatpar #2 }\relax\fi
  \vskip\parskip\vskip\subsectionskip
  \goodbreak\pagecheck\sectionminspace
  \global\advance\subsectionnum by \@ne
  \subsubsectionnum=\z@
  \edef\lab@l{\@chaptID\@sectID\SubsectionStyle{\the\subsectionnum}}%
  \ifshowsectID
     \global\edef\@fullID{\lab@l.\space}%
  \else\gdef\@fullID{}\fi
  \everysubsection
  \vbox{%
    {\raggedright\pretolerance=2000\hyphenpenalty=2000
    \setbox0=\hbox{\noindent\bf\@fullID}%
    \hangindent=\wd0 \hangafter=1
    \noindent\unhbox0{\bf{#1}\nobreak\medskip}}}%
  \begingroup
    \def\label##1{}%
    \global\edef\SubsectionTitle{#1}%
    \def\n{}\def\nl{}\def\mib{}%
   \emsg{\@fullID #1}%
    \def\@quote{\string\@quote\relax}%
    \addTOC{2}{\TOCsID{\lab@l.}#1}{\folio}%
  \endgroup
  \s@ction
  \aftersubsection\@aftersect}%
\def\everysubsection{\relax}%
\def\aftersubsection{\relax}%
\def\SubsectionStyle#1{#1}%
\long\def\subsubsection#1#2 {%
  \def\@aftersect{#2}%
  \ifx\@aftersect\empty\let\@aftersect=\@eatpar
  \else\def\@aftersect{\@eatpar #2 }\fi
  \vskip\parskip\vskip\subsectionskip
  \goodbreak\pagecheck\sectionminspace
  \global\advance\subsubsectionnum by \@ne
   \edef\lab@l{\@chaptID\@sectID\SubsectionStyle{\the\subsectionnum}.%
           \SubsubsectionStyle{\the\subsubsectionnum}}%
   \ifshowsectID
     \global\edef\@fullID{\lab@l.\space\space}%
   \else\gdef\@fullID{}\fi
   \everysubsubsection
   \vbox{%
     {\raggedright\bf
     \setbox0=\hbox{\noindent\@fullID}%
     \hangindent=\wd0 \hangafter=1
     \noindent\@fullID\relax
     #1\nobreak\medskip}}%
   \begingroup
     \def\label##1{}%
     \global\edef\SubsectionTitle{#1}%
     \def\n{}\def\nl{}\def\mib{}%
     \emsg{\@fullID #1}%
     \def\@quote{\string\@quote\relax}%
     \addTOC{3}{\TOCsID{\lab@l.}#1}{\folio}%
   \endgroup
   \s@ction
   \aftersubsubsection\@aftersect}%
\def\everysubsubsection{\relax}%
\def\aftersubsubsection{\relax}%
\def\SubsubsectionStyle#1{#1}%
\long\def\Appendix#1#2#3 {%
  \def\@aftersect{#3}%
  \ifx\@aftersect\empty\let\@aftersect=\@eatpar
  \else\def\@aftersect{\@eatpar #3 }\fi
  \def\@arg{#1}%
  \vfill\supereject
  \global\sectionnum=\z@
  \edef\lab@l{#1}%
  \gdef\@sectID{}%
  \ifshowchaptID
    \ifx\@arg\empty\else
      \global\edef\@chaptID{\lab@l.}\fi
    \r@set
  \else\def\@chaptID{}\fi
  \everychapter
  \ifx\Tbf\undefined\def\Tbf{\bf}\fi
  \leftline{\Tbf{Appendix\ \@chaptID}}%
  \begingroup
    \nobreak\smallskip
    \parindent=\z@\raggedright
    {\Tbf{#2}\bigskip}%
  \endgroup
  \nobreak\bigskip
  \begingroup
    \def\label##1{}%
    \global\edef\ChapterTitle{#2}%
    \def\n{}\def\nl{}\def\mib{}%
    \setHeadline{#2}%
    \emsg{Appendix \@chaptID\space #2}%
    \def\@quote{\string\@quote\relax}%
    \addTOC{0}{\TOCcID{\lab@l.}#2}{\folio}%
  \endgroup
  \@Mark{#2}%
  \s@ction
  \afterchapter\@aftersect}%
\long\def\appendix#1#2#3 {%
  \def\@aftersect{#3}%
  \ifx\@aftersect\empty\let\@aftersect=\@eatpar
  \else\def\@aftersect{\@eatpar #3 }\fi
   \vskip\parskip\vskip\sectionskip
   \goodbreak\pagecheck\sectionminspace
           \global\advance\sectionnum by \@ne
   \def\@arg{#1}%
   \gdef\@sectID{}\gdef\@fullID{}%
   \edef\lab@l{#1}%
   \ifshowsectID
     \r@set
     \ifx\@arg\empty\else
       \global\edef\@sectID{\lab@l.}%
       \global\edef\@fullID{\lab@l.\space\space}\fi
   \fi
   \everysection
   \ifx\tbf\undefined\def\tbf{\bf}\fi
   \vbox{%
     {\raggedright\tbf
     \setbox0=\hbox{\tbf\@fullID}%
     \hangindent=\wd0 \hangafter=1
     \noindent\@fullID
     {#2}\nobreak\medskip}}%
   \begingroup
     \def\label##1{}%
     \global\edef\SectionTitle{#2}%
     \def\n{}\def\nl{}\def\mib{}%
     \ifnum\chapternum=0\setHeadline{#2}\fi
     \emsg{appendix \@fullID #2}%
     \def\@quote{\string\@quote\relax}%
     \addTOC{1}{\TOCsID{\lab@l.}#2}{\folio}%
   \endgroup
   \s@ction
   \aftersection\@aftersect}%
\def\pagecheck#1{%
   \dimen@=\pagegoal
   \advance\dimen@ by -\pagetotal
   \ifdim\dimen@>0pt
   \ifdim\dimen@< #1\relax
      \vfil\break \fi\fi
   }
\def\nosechead#1{%
   \vskip\subsectionskip
   \goodbreak\pagecheck\sectionminspace
   \checkquote\checkenv
   \vbox{%
     {\raggedright\bf\noindent
     {#1}%
     \nobreak\medskip}}%
   }
\def\checkenv{%
   \ifx\@envdepth\undefined\relax
   \else\ifnum\@envdepth=\z@\relax
      \else\emsg{> Unclosed environment \@envname in the last section!}\fi 
   \fi}%

% file: TXStags.tex  (TeXsis version 2.17)
%%> {Labels and tags.}
\newread\auxfilein
\newwrite\auxfileout
\newif\ifauxswitch      \auxswitchtrue
\let\XA=\expandafter    \let\NX=\noexpand
\catcode`"=12
\catcode`@=11
\newcount\@BadTags   \@BadTags= 0
\def\auxinit{%
  \ifauxswitch
    \@FileInit\auxfileout=\jobname.aux[Auxiliary File]%
  \else \gdef\auxwritenow##1{}\gdef\auxwrite##1{} \fi
  \gdef\auxinit{\relax}}%
\def\auxwritenow#1{\auxinit
   \immediate\write\auxfileout{#1}}
\def\auxwrite#1{\auxinit\write\auxfileout{#1}}%
\def\auxoutnow#1#2{\auxwritenow{\string\newlabel{#1}{{#2}{\folio}}}}
\def\auxout#1#2{\auxwrite{\string\newlabel{#1}{{#2}{\folio}}}}
\def\ReadAUX{%
   \openin\auxfilein=\jobname.aux
   \ifeof\auxfilein\closein\auxfilein
   \else\closein\auxfilein
     \begingroup
        \def\@tag##1##2{\endgroup
           \edef\@@temp{##2}%
           \testtag{##1}\XA\xdef\csname\tok\endcsname{\@@temp}}%
       \unSpecial\ATunlock
       \input\jobname.aux \relax
     \endgroup
   \fi}%
\def\tag{%
   \begingroup\unSpecial
    \@tag}%
\def\@tag#1#2{%
   \endgroup
   \ifx\folio#2
     \auxout{#1}{#2}%
   \else
     \edef\@@temp{#2}%
     \stripblanks @#1@\endlist
     \XA\xdef\csname\tok\endcsname{\@@temp}%
     \auxoutnow{#1}{\@@temp}%
   \fi}
\def\label{\begingroup\unSpecial\@label}
\def\@label#1{\endgroup\tag{#1}{\lab@l}}
\def\lab@l{\relax}%
\def\newlabel{\begingroup\unSpecial\@newlabel}
\def\@newlabel#1#2{\endgroup\do@label#2\label{#1}}
\def\do@label#1#2{\def\lab@l{#1}\def\lab@lpage{#2}}
\def\use{%
   \begingroup\unSpecial\@use}          
\def\@use#1{\endgroup
   \stripblanks @#1@\endlist
   \XA\ifx\csname\tok\endcsname\relax\relax
     \emsg{> UNDEFINED TAG #1 ON PAGE \folio.}%
     \global\advance\@BadTags by 1
     \@errmark{UNDEF}%
     \edef\tok{{\bf\tok}}%
   \else
     \edef\tok{\csname\tok\endcsname}%
   \fi
   \tok}%
\def\unSpecial{%
     \catcode`@=12 \catcode`"=12 \catcode``=12  \catcode`'=12
     \catcode`[=12 \catcode`]=12 \catcode`(=12  \catcode`)=12
     \catcode`<=12 \catcode`>=12 \catcode`\&=12 \catcode`\#=12 
     \catcode`/=12}
\def\stripblanks{%
   \let\tok=\empty\@stripblanks}
\def\@stripblanks#1{\def\next{#1}\@striplist}
\def\@striplist{%
   \ifx\next\stripblanks\message{>\NX\@striplist: Oops!}\next=\endlist\fi
   \ifx\next\endlist\let\next=\relax
   \else\@stripspace\let\next=\@stripblanks\fi
   \next}
\def\@stripspace{\XA\if\space\next\else\edef\tok{\tok\next}\fi}
\def\endlist{\endlist}%
\newif\ifundefined      \undefinedfalse
\def\testtag#1{\stripblanks @#1@\endlist 
   \XA\ifx\csname\tok\endcsname\relax\undefinedtrue
      \else\undefinedfalse\fi}
\def\checktags{%
  \ifnum\@BadTags>\z@
    \emsg{>}\emsg{> There were \the\@BadTags\ references to undefined tags.}%
    \emsg{> See the file \jobname.log for the citations, or try running}%
    \emsg{> TeXsis again to resolve forward references.}\emsg{>}%
  \fi}
\def\LabelParse#1;#2;#3\endlist{%
  \def\@TagName{\@prefix#1}%
  \if ?#3?\relax
    \global\advance\@count by\@ne
  \else
    \stripblanks #2\endlist
    \edef\@arg{\tok}\if a\@arg\relax
      \global\advance\@count by\@ne\fi
    \xdef\@ID{\@chaptID\@sectID\the\@count\@arg}%
    \tag{\@prefix#1;\@arg}{\@ID}%
  \fi
  \xdef\@ID{\@chaptID\@sectID\the\@count}%
  \tag{\@prefix#1}{\@ID}%
}%
\def\@ID{}%
\newif\ifhtml   \htmlfalse
\def\html{\begingroup\htmlChar\@html}
\def\linkto{\begingroup\htmlChar\@linkto}
\def\linkname{\begingroup\htmlChar\@linkname}
\def\href{\begingroup\htmlChar\@href}
\def\URL{\begingroup\htmlChar\@URL}
\def\xxxcite{\begingroup\htmlChar\@xxxcite}
\def\notie{\def~{\Tilde}}
\def\urlChar{\def\/{\discretionary{}{/}{/}}}
\def\@htmlChar{\def\/{/}}
\begingroup
  \catcode`\~=12  \catcode`"=12     \catcode`\/=12
  \catcode`<=12   \catcode`>=12  
  \begingroup
     \catcode`\%=12 \catcode`\#=12 
     \gdef\htmlChar{\notie
        \catcode`@=12 \catcode`"=12  \catcode``=12  \catcode`'=12
        \catcode`[=12 \catcode`]=12  \catcode`(=12  \catcode`)=12
        \catcode`<=12 \catcode`>=12  \catcode`_=12  \catcode`^=12  
        \catcode`$=12 \catcode`\&=12 \catcode`\#=12 \catcode`%=12 
        \catcode`~=12 \catcode`/=12  \catcode`/=12  \@htmlChar}
     \gdef\hash{#}\gdef\Tilde{~}
  \endgroup
  \gdef\@html#1{\ifhtml\special{html:#1}\fi\endgroup}%
  \gdef\@linkto#1{\endgroup\@@linkto{#1}}%
  \gdef\@@linkto#1#2{\html{<a href="\hash#1">}{#2}\html{</a>}}
  \gdef\@linkname#1{\endgroup\@@linkname{#1}}
  \gdef\@@linkname#1#2{\html{<a name="#1">}{#2}\html{</a>}}
  \gdef\@href#1{\endgroup\@@href{#1}}%
  \gdef\@@href#1#2{\html{<a href="#1">}\urlChar{#2}\html{</a>}}%
  \gdef\@URL#1{\html{<a href="#1">}\urlChar{\tt #1}\html{</a>}\endgroup}%
  \gdef\@xxxcite#1{\href{http://xxx.lanl.gov/abs/#1}%
        \urlChar{#1}\relax}
\endgroup
\let\hypertarget=\linkname  \let\hname=\linkname

% file: TXStitle.tex  (TeXsis version 2.17)
%%> {Title Page macros.}
\catcode`@=11
\def\pubcode#1{\gdef\@DOCcode{#1}}
\def\PUBcode#1{\gdef\@DOCcode{#1}}%
\def\DOCcode#1{\PUBcode{#1}}%
\def\BNLcode#1{\PUBcode{#1}\banner}%
\def\@DOCcode{\TeXsis~\fmtversion}%
\def\pubdate#1{\gdef\@PUBdate{#1}}
\def\PUBdate#1{\gdef\@PUBdate{#1}}%
\def\@PUBdate{\monthname{\month},~\number\year}%
\def\ORGANIZATION{}%
\def\banner{%
   \line{\hfil
      \vbox to 0pt{\vss \hbox{\twelvess \ORGANIZATION}}%
      \hfil}%
   \vskip 12pt
   \hrule height 0.6pt \vskip 1pt \hrule height 0.6pt
   \vskip 4pt \relax
   \line{\twelvepoint\rm\@PUBdate \hfil \@DOCcode}%
   \vskip 3pt
   \hrule height 0.6pt \vskip 1pt \hrule height 0.6pt
   \vskip 0pt plus 1fil
   \vskip 1.0cm minus 1.0cm
   \relax}
\def\titlepage{%
   \bgroup
   \pageno=1
   \hbox{\space}%
   \let\title=\Title
   \let\endmode=\relax
   }
\def\endtitlepage{%
   \endmode
   \vfil\eject
   \egroup}%
\def\title{%
   \endmode
   \vskip 0pt
   \mark{Title Page\NX\else Title Page}% mark page so no \HeadLine
   \bgroup
   \let\endmode=\endTitle
   \center\Tbf}%
\let\Title=\title
\def\endtitle{%
   \endcenter
   \bigskip
   \gdef\title{%
      \emsg{> Please use \NX\booktitle instead of \NX\title.}%
      \@errmark{OLD!}%
      \booktitle}%
   \egroup}%
\def\endTitle{\endtitle}%
\def\Tbf{\sixteenpoint\bf}%
\def\author{%
  \endmode
  \bgroup
   \let\endmode=\endauthor
   \singlespaced\parskip=0pt
   \obeylines\def\\{\par}%
   \@getauthor}%
{\obeylines\gdef\@getauthor#1
  #2
  {#1\bigskip\def\n{\egroup\centerline\bgroup\bf}%
   \centerline{\bf #2}%
   \medskip\center}%
}
\def\endauthor{\endcenter\egroup\bigskip}
\def\authors{%
   \endmode
   \bigskip
   \bgroup
    \let\endmode=\endauthors
    \let\@uthorskip=\medskip
    \raggedcenter\singlespaced}%
\def\endauthors{%
   \endraggedcenter
   \egroup
   \bigskip}%
\def\note#1#2{%
  ${}^{\hbox{#1}}\ $
  \space@head#2
  #2}%
\def\institution#1#2{%
   \@uthorskip\let\@uthorskip=\relax
   \raggedcenter
      ${}^{\rm #1}$\space #2%                  
   \endraggedcenter
   }
\let\@uthorskip=\medskip
\long\def\titlenote#1#2{%
   \footnote{}{%
   \llap{\hbox to \parindent{\hfil
   ${}^{\rm #1}$\space}}#2}}%
\def\and{\centerline{and}\medskip}
\def\AbstractName{ABSTRACT}%
\def\abstract{%
   \endmode
   \bigskip\bigskip
    \centerline{\AbstractName}%
    \medskip
    \bgroup
    \let\endmode=\endabstract
    \narrower\narrower
    \singlespaced
    \everyabstract}%
\def\everyabstract{}%
\def\endabstract{\smallskip\egroup}
\def\pacs#1{\medskip\centerline{PACS numbers: #1}\smallskip}
\def\submit#1{\bigskip\centerline{Submitted to {\sl #1}}}
\def\submitted#1{\submit{#1}}%
\def\toappear#1{\bigskip\raggedcenter
     To appear in {\sl #1}
     \endraggedcenter}
\def\disclaimer#1{\footnote{}\bgroup\tenrm\singlespaced
   This manuscript has been authored under contract number #1
   \@disclaimer\par}
\def\disclaimers#1{\footnote{}\bgroup\tenrm\singlespaced
   This manuscript has been authored under contract numbers #1
   \@disclaimer\par}
\def\@disclaimer{%
with the U.S. Department of Energy.  Accordingly, the U.S.
Government retains a non-exclusive, royalty-free license to publish
or reproduce the published form of this contribution,
or allow others to do so, for U.S. Government purposes.
\egroup}

% file: TXSenvmt.tex  (TeXsis version 2.17)
%%> {Environments.}
\catcode`@=11
\chardef\other=12
\def\center{%
   \flushenv
   \advance\leftskip \z@ plus 1fil
   \advance\rightskip \z@ plus 1fil
   \obeylines\@eatpar}%
\def\flushright{%
    \flushenv
    \advance\leftskip \z@ plus 1fil
    \obeylines\@eatpar}%
\def\flushleft{%
   \flushenv
   \advance\rightskip \z@ plus 1fil
   \obeylines\@eatpar}%
\def\flushenv{%
    \vskip \z@
    \bgroup
     \def\flushhmode{F}%
     \parindent=\z@  \parfillskip=\z@}%
\def\endcenter{\endflushenv}
\def\endflushleft{\endflushenv}
\def\endflushright{\endflushenv}
\def\@eatpar{\futurelet\next\@testpar}
\def\@testpar{\ifx\next\par\let\@next=\@@eatpar\else\let\@next=\relax\fi\@next}
\long\def\@@eatpar#1{\relax}
\def\raggedcenter{%
    \flushenv
    \advance\leftskip\z@ plus4em
    \advance\rightskip\z@ plus 4em
    \spaceskip=.3333em \xspaceskip=.5em
    \pretolerance=9999 \tolerance=9999
    \hyphenpenalty=9999 \exhyphenpenalty=9999
    \@eatpar}%
\def\endraggedcenter{\endflushenv}%
\def\hcenter{\hflushenv
   \advance\leftskip \z@ plus 1fil
   \advance\rightskip \z@ plus 1fil
   \obeylines\@eatpar}%
\def\hflushright{\hflushenv
    \advance\leftskip \z@ plus 1fil
    \obeylines\@eatpar}%
\def\hflushleft{\hflushenv
    \advance\rightskip \z@ plus 1fil
    \obeylines\@eatpar}%
\def\hflushenv{%
   \def\par{\endgraf\indent}%
   \hbox to \z@ \bgroup\hss\vtop
   \flushenv\def\flushhmode{T}}%
\def\endflushenv{%
   \ifhmode\endgraf\fi
   \if T\flushhmode \egroup\hss\fi
   \egroup}%
\def\flushhmode{U}     
\def\endhcenter{\endflushenv}
\def\endhflushleft{\endflushenv}
\def\endhflushright{\endflushenv}
\newskip\EnvTopskip     \EnvTopskip=\medskipamount
\newskip\EnvBottomskip  \EnvBottomskip=\medskipamount
\newskip\EnvLeftskip    \EnvLeftskip=2\parindent
\newskip\EnvRightskip   \EnvRightskip=\parindent
\newskip\EnvDelt@skip   \EnvDelt@skip=0pt
\newcount\@envDepth     \@envDepth=\z@
\def\beginEnv#1{%
   \begingroup
     \def\@envname{#1}%
     \ifvmode\def\@isVmode{T}%
     \else\def\@isVmode{F}\vskip 0pt\fi
     \ifnum\@envDepth=\@ne\parindent=\z@\fi
     \advance\@envDepth by \@ne
     \EnvDelt@skip=\baselineskip
     \advance\EnvDelt@skip by-\normalbaselineskip
     \@setenvmargins\EnvLeftskip\EnvRightskip
     \setenvskip{\EnvTopskip}%
     \vskip\skip@\penalty-500
   }
\def\endEnv#1{%
   \ifnum\@envDepth<1
      \emsg{> Tried to close ``#1'' environment, but no environment open!}%
      \begingroup
   \else
      \def\test{#1}%
      \ifx\test\@envname\else
         \emsg{> Miss-matched environments!}%
         \emsg{> Should be closing ``\@envname'' instead of ``\test''}%
      \fi
   \fi
   \vskip 0pt
   \setenvskip\EnvBottomskip
   \vskip\skip@\penalty-500
   \xdef\@envtemp{\@isVmode}%
   \endgroup
   \if F\@envtemp\vskip-\parskip\par\noindent\fi
   }
\def\setenvskip#1{\skip@=#1 \divide\skip@ by \@envDepth}
\def\@setenvmargins#1#2{%
   \advance \leftskip  by #1    \advance \displaywidth by -#1
   \advance \rightskip by #2    \advance \displaywidth by -#2
   \advance \displayindent by #1}%
\def\itemize{\beginEnv{itemize}%
   \let\itm=\itemizeitem
%   \if F\@isVmode 
      \vskip-\parskip
%   \fi    
   }
\def\itemizeitem{%
   \par\noindent
   \hbox to 0pt{\hss\itemmark\space}}%
\def\enditemize{\endEnv{itemize}}%
\def\itemmark{$\bullet$}%
\newcount\enumDepth     \enumDepth=\z@
\newcount\enumcnt
\def\enumerate{\beginEnv{enumerate}%
   \global\advance\enumDepth by \@ne
   \setenumlead
   \enumcnt=\z@
   \let\itm=\enumerateitem
   \if F\@isVmode\vskip-\parskip\fi
   }
\def\enumerateitem{%
    \par\noindent                 
    \advance\enumcnt by \@ne
    \edef\lab@l{\enumlead \enumcur}%
    \hbox to \z@{\hss \lab@l \enummark
       \hskip .5em\relax}%
    \ignorespaces}%
\def\endenumerate{%
   \global\advance\enumDepth by -\@ne
   \endEnv{enumerate}}%
\def\enumPoints{%
   \def\setenumlead{\ifnum\enumDepth>1
          \edef\enumlead{\enumlead\enumcur.}%
      \else\def\enumlead{}\fi}%
   \def\enumcur{\number\enumcnt}%
   }
\def\enumpoints{\enumPoints}%
\def\enumOutline{%
   \def\setenumlead{\def\enumlead{}}%
   \def\enumcur{\ifcase\enumDepth
     \or\uppercase{\XA\romannumeral\number\enumcnt}%
     \or\LetterN{\the\enumcnt}%
     \or\XA\romannumeral\number\enumcnt
     \or\letterN{\the\enumcnt}%
     \or{\the\enumcnt}%
     \else $\bullet$\space\fi}%
   }
\def\enumoutline{\enumOutline}%
\def\enumNumOutline{%
   \def\setenumlead{\def\enumlead{}}%
   \def\enumcur{\ifcase\enumDepth
      \or{\XA\number\enumcnt}%
      \or\letterN{\the\enumcnt}%
      \or{\XA\romannumeral\number\enumcnt}%
      \else $\bullet$\space\fi}%
   }
\def\enumnumoutline{\enumNumOutline}%
\def\LetterN#1{\count@=#1 \advance\count@ 64 \XA\char\count@}
\def\letterN#1{\count@=#1 \advance\count@ 96 \XA\char\count@}
\def\enummark{.}%
\def\enumlead{}%
\enumpoints
\newbox\@desbox
\newbox\@desline
\newdimen\@glodeswd
\newcount\@deslines
\newif\ifsingleline \singlelinefalse
\def\description#1{\beginEnv{description}%
   \setbox\@desbox=\hbox{#1}%
   \@glodeswd=\wd\@desbox
   \@setenvmargins{\@glodeswd}{0pt}%
   \let\itm=\descriptionitem
   \if F\@isVmode\vskip-\parskip\fi
  }%
\def\descriptionitem#1{%
   \goodbreak\noindent
   \setbox\@desline=\vtop\bgroup
      \hfuzz=100cm\hsize=\@glodeswd
      \rightskip=\z@ \leftskip=\z@
      \raggedright
      \noindent{#1}\par
      \global\@deslines=\prevgraf
      \egroup
   \ifsingleline
     \ifnum\@deslines>1
        \@deslineitm{#1}%
     \else
        \setbox\@desline=\hbox{#1}%
        \ifdim \wd\@desline>\wd\@desbox
            \@deslineitm{#1}%
        \else\@desitm\fi
     \fi
   \else
     \@desitm
   \fi
   \ignorespaces}
\def\@desitm{%
   \noindent
   \hbox to \z@{\hskip-\@glodeswd
     \hbox to \@glodeswd{\vtop to \z@{\box\@desline\vss}%
     \hss}\hss}}%
\def\@deslineitm#1{%
   \hbox{\hskip-\@glodeswd {#1}\hss}%
   \vskip-\parskip\nobreak\noindent
   }
\def\enddescription{\ifhmode\par\fi
   \@setenvmargins{-\wd\@desbox}{0pt}%
   \endEnv{description}}
\def\example{\beginEnv{example}%
   \parskip=\z@ \parindent=\z@
   \baselineskip=\normalbaselineskip
   }%
\def\endexample{\endEnv{example}%
   \noindent}%
\let\blockquote=\example
\let\endblockquote=\endexample
\def\Listing{%
   \beginEnv{Listing}%
   \vskip\EnvDelt@skip
   \baselineskip=\normalbaselineskip
   \parskip=\z@ \parindent=\z@
   \def\\##1{\char92##1}%
   \catcode`\{=\other \catcode`\}=\other
   \catcode`\(=\other \catcode`\)=\other
   \catcode`\"=\other \catcode`\|=\other
   \catcode`\%=\other \catcode`\&=\other        
   \catcode`\-=\other \catcode`\==\other
   \catcode`\$=\other \catcode`\#=\other
   \catcode`\_=\other \catcode`\^=\other
   \catcode`\~=\other
   \obeylines
   \tt\Listingtabs
   \everyListing}%
\def\endListing{\endEnv{Listing}}%
\def\everyListing{\relax}
\def\ListCodeFile#1{%
   \Listing
   \rightskip=\z@ plus 5cm              
   \catcode`\\=\other
   \input #1\relax
   \endListing}
{\catcode`\^^I=\active\catcode`\ =\active
\gdef\Listingtabs{\catcode`\^^I=\active\let^^I\@listingtab
\catcode`\ =\active\let \@listingspace}%
}%
\def\@listingspace{\hskip 0.5em\relax}%
\def\@listingtab{\hskip 4em\relax}%
\def\TeXexample{\beginEnv{TeXexample}%
   \vskip\EnvDelt@skip
   \parskip=\z@ \parindent=\z@
   \baselineskip=\normalbaselineskip
   \def\par{\leavevmode\endgraf}%
   \obeylines
   \catcode`|=\z@
   \ttverbatim
   \@eatpar}%
\def\endTeXexample{%
   \vskip 0pt
   \endgroup
   \endEnv{TeXexample}}%
\def\ttverbatim{\begingroup
   \catcode`\(=\other \catcode`\)=\other
   \catcode`\"=\other \catcode`\[=\other 
   \catcode`\]=\other \catcode`\~=\other
   \let\do=\uncatcode \dospecials 
   \obeyspaces\obeylines
   \def\n{\vskip\baselineskip}%
   \tt}%
\def\uncatcode#1{\catcode`#1=\other}%
{\obeyspaces\gdef {\ }}%
\def\TeXquoteon{\catcode`\|=\active}%
\let\TeXquoteson=\TeXquoteon
\def\TeXquoteoff{\catcode`\|=\other}%
\let\TeXquotesoff=\TeXquoteoff
{\TeXquoteon\obeylines% active "|" calls \ttverbatim
   \gdef|{\ifmmode\vert\else% | is \vert in math mode, but
     \ttverbatim\spaceskip=\ttglue% to use \tt type
     \let^^M=\ \relax% and to ignore ^^M
     \let|=\endgroup\fi}%
}     
\def\ttvert{\hbox{\tt\char`\|}}
\outer\def\begintt{$$\let\par=\endgraf \ttverbatim \parskip=0pt
   \catcode`\|=0 \rightskip=-5pc \ttfinish}
{\catcode`\|=0 |catcode`|\=\other% | is temporary escape character
   |obeylines% end of line is active
   |gdef|ttfinish#1^^M#2\endtt{#1|vbox{#2}|endgroup$$}%
}
\def\beginlines{\par\begingroup\nobreak\medskip\parindent=0pt
   \hrule\kern1pt\nobreak \obeylines \everypar{\strut}}
\def\endlines{\kern1pt\hrule\endgroup\medbreak\noindent}
\def\beginproclaim#1#2#3#4#5{\medbreak\vskip-\parskip
   \global\XA\advance\csname #2\endcsname by \@ne
   \edef\lab@l{\@chaptID\@sectID
      \number\csname #2\endcsname}%
   \tag{#4#5}{\lab@l}%
   \noindent{\bf #1 \lab@l.\space}%
   \begingroup #3}%
\def\endproclaim{%
   \par\endgroup\ifdim\lastskip<\medskipamount
   \removelastskip\penalty55\medskip\fi}%
\newcount\theoremnum           \theoremnum=\z@
\def\theorem#1{\beginproclaim{Theorem}{theoremnum}{\sl}{Thm.}{#1}}
\let\endtheorem=\endproclaim
\def\Theorem#1{Theorem~\use{Thm.#1}}
\newcount\lemmanum             \lemmanum=\z@
\def\lemma#1{\beginproclaim{Lemma}{lemmanum}{\sl}{Lem.}{#1}}
\let\endlemma=\endproclaim
\def\Lemma#1{Lemma~\use{Lem.#1}}
\newcount\corollarynum         \corollarynum=\z@
\def\corollary#1{\beginproclaim{Corollary}{corollarynum}{\sl}{Cor.}{#1}}
\let\endcorollary=\endproclaim
\def\Corollary#1{Corollary~\use{Cor.#1}}
\newcount\definitionnum        \definitionnum=\z@
\def\definition#1{\beginproclaim{Definition}{definitionnum}{\rm}{Def.}{#1}}
\let\enddefinition=\endproclaim
\def\Definition#1{Definition~\use{Def.#1}}
\def\proof{\medbreak\vskip-\parskip\noindent{\it Proof. }}
\def\blackslug{%
   \setbox0\hbox{(}%
   \vrule width.5em height\ht0 depth\dp0}%
\def\QED{\blackslug}%
\def\endproof{\quad\blackslug\par\medskip}

% file: TXSfmts.tex (TeXsis version 2.17)
%%> {Layout macros.}
\catcode`@=11
\def\paper{%
   \auxswitchtrue
   \refswitchtrue
   \texsis
   \def\titlepage{%
      \bgroup
      \let\title=\Title
      \let\endmode=\relax
      \pageno=1}%
   \def\endtitlepage{%
      \endmode
      \goodbreak\bigskip
      \egroup}%
   \autoparens
   \quoteon
   }
\def\Tbf{\fourteenpoint\bf}%
\def\tbf{\twelvepoint\bf}%
\def\preprint{%
   \auxswitchtrue
   \refswitchtrue
   \texsis
   \def\titlepage{%
      \bgroup
      \pageno=1
      \let\title=\Title
      \let\endmode=\relax
      \banner}%
   \def\endtitlepage{%
      \endmode
      \vfil\eject
      \egroup}%
   \autoparens
   \quoteon
   }
\def\Manuscript{%
   \preprint
   \showsectIDfalse
   \showchaptIDfalse
   \def\SectionStyle##1{\uppercase
         \expandafter{\romannumeral ##1}}%
   \RomanTablestrue
   \TablesLast
   \FiguresLast
   \TrueDoubleSpacing
   \def\everyabstract{\TrueDoubleSpacing}
   \def\Tbf{\fourteenpoint\bf\TrueDoubleSpacing}%
   \def\refFormat{\TrueDoubleSpacing}%
   }
\autoload\PhysRevManuscript{PhysRev.txs}%
\def\book{%
   \ContentsSwitchtrue
   \refswitchtrue
   \auxswitchtrue
   \texsis
   \RunningHeadstrue
   \bookpagenumbers
   \def\titlepage{%
      \bgroup
      \pageno=-1
      \let\title=\Title
      \let\endmode=\relax
      \def\FootText{\relax}}%
   \def\endtitlepage{%
      \endmode
      \vfil\eject
      \egroup
      \pageno=1}%
   \def\abstract{%
      \endmode
      \bigskip\bigskip\medskip
      \bgroup\singlespaced
         \let\endmode=\endabstract
         \narrower\narrower
         \everyabstract}%
   \def\endabstract{%
      \medskip\egroup\bigskip}%
   \def\FootText{--\ \tenrm\folio\ --}%
   \def\Tbf{\sixteenpoint\bf}%
   \def\tbf{\fourteenpoint\bf}%
   \twelvepoint
   \doublespaced
   \autoparens
   \quoteon
   }%
\autoload\thesis{thesis.txs}
\autoload\UTthesis{thesis.txs}
\autoload\YaleThesis{thesis.txs}
\def\Letter{%
   \ContentsSwitchfalse
   \refswitchfalse
   \auxswitchfalse
   \texsis
   \singlespaced
   \LetterFormat}%
\def\letter{\Letter}%
\def\Memo{%
   \ContentsSwitchfalse
   \refswitchfalse
   \auxswitchfalse
   \texsis
   \singlespaced
   \MemoFormat}%
\def\memo{\Memo}%
\def\Referee{%
   \ContentsSwitchfalse
   \auxswitchfalse
   \refswitchfalse
   \texsis
   \RefReptFormat}%
\def\referee{\Referee}%
\def\Landscape{%
   \texsis
   \hsize=9in
   \vsize=6.5in
   \voffset=.5in
   \nopagenumbers
   \LandscapeSpecial
}
\def\landscape{\Landscape}%
\def\LandscapeSpecial{\special{papersize=11in,8.5in}}
\def\slides{%
   \quoteon
   \autoparens
   \ATlock
   \pageno=1
   \twentyfourpoint
   \doublespaced
   \raggedright\tolerance=2000
   \hyphenpenalty=500
   \raggedbottom
   \nopagenumbers
   \hoffset=-.25in \hsize=7.0in
   \voffset=-.25in \vsize=9.0in
   \parindent=30pt
   \def\bl{\vskip\normalbaselineskip}%
   \def\np{\vfill\eject}%
   \def\nospace{\nulldelimiterspace=0pt
      \mathsurround=0pt}%
   \def\big##1{{\hbox{$\left##1
      \vbox to2ex{}\right.\nospace$}}}%
   \def\Big##1{{\hbox{$\left##1
      \vbox to2.5ex{}\right.\nospace$}}}%
   \def\bigg##1{{\hbox{$\left##1
       \vbox to3ex{}\right.\nospace$}}}%
   \def\Bigg##1{{\hbox{$\left##1
      \vbox to4ex{}\right.\nospace$}}}%
  }
\autoload\twinout{twin.txs}%
\def\twinprint{%
   \preprint
   \let\t@tl@=\title
   \def\title{\vskip-1.5in\t@tl@}%
   \let\endt@tlep@ge=\endtitlepage
   \def\endtitlepage{\endt@tlep@ge
       \twinformat}%
}
\def\twinformat{%
   \tenpoint\doublespaced
   \def\Tbf{\twelvebf}\def\tbf{\tenbf}%
   \headlineoffset=0pt
   \twinout}%

% file: TXSfigs.tex (TeXsis version 2.17)
%%> {Figures and Tables.}
\catcode`\@=11
\let\NX=\noexpand\let\XA=\expandafter
\offparens
\newcount\tabnum        \tabnum=\z@
\newcount\fignum        \fignum=\z@
\newif\ifRomanTables    \RomanTablesfalse
\newif\ifCaptionList    \CaptionListfalse
\newif\ifFigsLast       \FigsLastfalse
\newif\ifTabsLast       \TabsLastfalse
\def\FiguresLast{\FigsLasttrue}\def\FiguresNow{\FigsLastfalse}
\def\TablesLast{\TabsLasttrue}\def\TablesNow{\TabsLastfalse}
%%tmp \newwrite\fgout         \newwrite\tbout 
%%tmp \newwrite\figlist       \newwrite\tablelist     \newwrite\caplist       
\long\def\figure{\@figure\topinsert}
\long\def\topfigure{\@figure\topinsert}%
\long\def\midfigure{\@figure\midinsert}
\long\def\fullfigure{\@figure\pageinsert}
\long\def\bottomfigure{\@figure\bottominsert}
\long\def\heavyfigure{\@figure\heavyinsert}
\long\def\widefigure{\@figure\widetopinsert}
\long\def\widetopfigure{\@figure\widetopinsert}
\long\def\widefullfigure{\@figure\widepageinsert}
\def\FigureName{Figure}%
\def\TableName{Table}%
\def\@figure#1#2{%
  \vskip 0pt
  \begingroup
    \def\CaptionName{\FigureName}%
    \def\@prefix{Fg.}%
    \let\@count=\fignum
    \let\@FigInsert=#1\relax
    \def\@arg{#2}\ifx\@arg\empty\def\@ID{}%
      \else\LabelParse #2;;\endlist\fi
    \ifFigsLast
      \emsg{\CaptionName\space\@ID. {#2} [storing in \jobname.fg]}%
      \@fgwrite{\@comment> \CaptionName\space\@ID.\space{#2}}%
      \@fgwrite{\string\@FigureItem{\CaptionName}{\@ID}{\NX#1}}%
      \seeCR\let\@next=\@copyfig
    \else
      \emsg{\CaptionName\ \@ID.\ {#2}}%
      \let\endfigure=\@endfigure
      \setbox\@capbox\vbox to 0pt{}%
      \def\@whereCap{N}%
      \let\@next=\@findcap
      \ifx\@FigInsert\midinsert\goodbreak\fi
      \@FigInsert
    \fi \@next}
\def\@endfigure{\relax
   \if B\@whereCap\relax
     \vskip\normalbaselineskip
     \centerline{\box\@capbox}%
   \fi 
   \endinsert \endgroup}%
\def\endfigure{\emsg{> \string\endfigure before \string\figure!}}
\def\figuresize#1{\vglue #1}%
\def\@copyfig#1#2\endfigure{\endgroup
   \ifx#1\par\@fgNXwrite{#2\@endfigure}\else\@fgNXwrite{#1#2\@endfigure}\fi}
\def\@FGinit{\@FileInit\fgout=\jobname.fg[Figures]\gdef\@FGinit{\relax}}
\def\@fgwrite#1{\@FGinit\immediate\write\fgout{#1}}
\long\def\@fgNXwrite#1{\@FGinit\unexpandedwrite\fgout{#1}}
\def\PrintFigures{\ifFigsLast\@PrintFigures\fi}
\def\@PrintFigures{%
   \@fgwrite{\@comment>>> EOF \jobname.fg <<<}%
   \immediate\closeout\fgout
   \begingroup
      \FigsLastfalse
      \vbox to 0pt{\hbox to 0pt{\ \hss}\vss}%
      \offparens
      \catcode`@=11
      \emsg{[Getting figures from file \jobname.fg]}%
      \Input\jobname.fg \relax
   \endgroup}%
\def\@FigureItem#1#2#3{%
   \begingroup
    #3\relax
    \def\@ID{#2}%
    \def\CaptionName{#1}%
    \setbox\@capbox\vbox to 0pt{}\def\@whereCap{N}%
    \@findcap}%
\long\def\table{\@table\topinsert}
\long\def\toptable{\@table\topinsert}%
\long\def\midtable{\@table\midinsert}
\long\def\fulltable{\@table\pageinsert}
\long\def\bottomtable{\@table\bottominsert}
\long\def\heavytable{\@table\heavyinsert}
\long\def\widetable{\@table\widetopinsert}
\long\def\widetoptable{\@table\widetopinsert}
\long\def\widefulltable{\@table\widepageinsert}
\def\@table#1#2{%
  \vskip 0pt
  \begingroup
    \def\CaptionName{\TableName}%
    \def\@prefix{Tb.}%
    \let\@count=\tabnum
    \let\@FigInsert=#1\relax
    \def\@arg{#2}\ifx\@arg\empty\def\@ID{}%
    \else\ifRomanTables
         \global\advance\@count by\@ne
         \edef\@ID{\uppercase\expandafter
            {\romannumeral\the\@count}}%
         \tag{\@prefix#2}{\@ID}%
    \else
        \LabelParse #2;;\endlist\fi
    \fi
    \ifTabsLast
      \emsg{\CaptionName\space\@ID. {#2} [storing in \jobname.tb]}%
      \@tbwrite{\@comment> \CaptionName\space\@ID.\space{#2}}%
      \@tbwrite{\string\@FigureItem{\CaptionName}{\@ID}{\NX#1}}%
      \seeCR\let\@next=\@copytab
    \else
      \emsg{\CaptionName\ \@ID.\ {#2}}%
      \let\endtable=\@endfigure
      \setbox\@capbox\vbox to 0pt{}%
      \def\@whereCap{N}%
      \let\@next=\@findcap
      \ifx\@FigInsert\midinsert\goodbreak\fi
      \@FigInsert
    \fi \@next}
\def\endtable{\emsg{> \string\endtable before \string\table!}}
\def\@copytab#1#2\endtable{\endgroup
    \ifx#1\par\@tbNXwrite{#2\@endfigure}\else\@tbNXwrite{#1#2\@endfigure}\fi}
\def\@TBinit{\@FileInit\tbout=\jobname.tb[Tables]\gdef\@TBinit{\relax}}
\def\@tbwrite#1{\@TBinit\immediate\write\tbout{#1}}
\long\def\@tbNXwrite#1{\@TBinit\unexpandedwrite\tbout{#1}}
\def\PrintTables{\ifTabsLast\@PrintTables\fi}
\def\@PrintTables{%
   \@tbwrite{\@comment>>> EOF \jobname.tb <<<}%
   \immediate\closeout\tbout
   \begingroup
     \TabsLastfalse
     \catcode`@=11
     \offparens
     \emsg{[Getting tables from file \jobname.tb]}%
     \Input\jobname.tb \relax
   \endgroup}%
\newbox\@capbox
\newcount\@caplines
\def\CaptionName{}%
\def\@ID{}%
\def\captionspacing{\normalbaselines}%
\def\@inCaption{F}%
\long\def\caption#1{%
   \def\lab@l{\@ID}%
   \global\setbox\@capbox=\vbox\bgroup
     \def\@inCaption{T}%
     \captionspacing\seeCR
     \dimen@=20\parindent
     \ifdim\colwidth>\dimen@\narrower\narrower\fi
     \noindent{\bf \linkname{\@TagName}{\CaptionName~\@ID}:\space}%
     #1\relax
     \vskip 0pt
     \global\@caplines=\prevgraf
   \egroup
   \ifnum\@caplines=\@ne
     \global\setbox\@capbox=\vbox{\noindent\seeCR
         \hfil{\bf \linkname{\@TagName}{\CaptionName~\@ID}:\space}%
         #1\hfil}\fi
   \if N\@whereCap\def\@whereCap{B}\fi
   \if T\@whereCap
     \centerline{\box\@capbox}%
     \vskip\baselineskip\medskip
   \fi}%
\def\Caption{\begingroup\seeCR\@Caption}%
\long\def\@Caption#1\endCaption{\endgroup
   \ifCaptionList
%      \def\@arg{Figure}\ifx\CaptionName\@arg\let\caplist=\figlist\fi
%      \def\@arg{Table}\ifx\CaptionName\@arg\let\caplist=\tablelist\fi
      \incaplist{#1}\fi 
   \caption{#1}}%
\def\endCaption{\emsg{> \string\endCaption\ called before \string\Caption.}}
\long\def\@findcap#1{%
   \ifx #1\Caption \def\@whereCap{T}\fi
   \ifx #1\caption \def\@whereCap{T}\fi
   #1}%
\def\@whereCap{N}%
\def\ListCaptions{\@ListCaps\caplist=\jobname.cap[List of Captions]
        {\let\FIGLitem=\CAPLitem}}
\def\ListFigureCaptions{%
    \@ListCaps\figlist=\jobname.fgl[List of Figure Captions]
    {\let\FIGLitem=\CAPLitem}}
\def\ListTableCaptions{%
    \@ListCaps\tablelist=\jobname.tbl[List of Table Captions]
    {\let\FIGLitem=\CAPLitem}}
\def\CAPLitem#1#2#3\@endFIGLitem#4{%
   \bigskip
   \begingroup
     \raggedright\tolerance=1700
     \hangindent=1.41\parindent\hangafter=1
     \noindent #1\ #2
     #3 \hskip 0pt plus 10pt
     \vskip 0pt
   \endgroup}%
\def\infiglist{\begingroup\seeCR
     \@infiglist\figlist}
\def\intablelist{\begingroup\seeCR
     \@infiglist\tablelist}
\def\incaplist{\begingroup\seeCR
     \@infiglist\caplist}
\def\FigListWrite#1#2{%
  \ifx#1\figlist\relax   \FigListInit\fi
  \ifx#1\tablelist\relax \TabListInit\fi
  \ifx#1\caplist\relax   \CapListInit\fi
  \edef\@line@{{#2}}%
  \write#1\@line@}%
\def\FigListInit{\@FileInit\figlist=\jobname.fgl[List of Figures]%
        \gdef\FigListInit{\relax}}
\def\TabListInit{\@FileInit\tablelist=\jobname.tbl[List of Tables]%
        \gdef\TabListInit{\relax}}  
\def\CapListInit{\@FileInit\caplist=\jobname.cap[List of Captions]%
        \gdef\CapListInit{\relax}}  
\def\FigListWriteNX#1#2{\writeNX#1{#2}} 
\def\@infiglist#1#2{%
     \FigListWrite#1{\@comment > \CaptionName\space\@ID:}%
     \FigListWrite#1{\string\FIGLitem{\CaptionName} {\@ID.\space}}%
     \@copycap#1#2\endlist
     \FigListWrite#1{{\NX\folio}}%
   \endgroup}%
\def\@copycap#1#2#3\endlist{%
   \ifx#2\space\writeNX#1{#3\@endFIGLitem}%
   \else\writeNX#1{#2#3\@endFIGLitem}\fi}
\def\ListFigures{\@ListCaps\figlist=\jobname.fgl[List of Figures]{}}
\def\ListTables{\@ListCaps\tablelist=\jobname.tbl[List of Tables]{}}
\def\@ListCaps#1=#2[#3]#4{%
   \immediate\closeout#1
   \openin#1=#2 \relax
   \ifeof#1\closein#1
   \else\closein#1\emsg{[Getting #3]}%
     \begingroup
      \showsectIDtrue
      \ATunlock\quoteoff\offparens
      #4
      \input #2 \relax
     \endgroup
   \fi}
\long\def\FIGLitem#1#2#3\@endFIGLitem#4{%
   \medskip
   \begingroup
     \raggedright\tolerance=1700
     \ifx\TOCmargin\undefined\skip0=\parindent
     \else\skip0=\TOCmargin\fi
     \advance\rightskip by \skip0
     \parfillskip=-\skip0
     \hangindent=1.41\parindent\hangafter=1
     \noindent \ifshowsectID #1\ \fi #2
        #3 \hskip 0pt plus 10pt
     \leaddots
     \hbox to 2em{\hss\linkto{page.#4}{#4}}%
     \vskip 0pt
   \endgroup}
\def\Fig#1{\linkto{Fg.#1}{Fig.~\use{Fg.#1}}}    
\def\Figs#1{\linkto{Fg.#1}{Figs.~\use{Fg.#1}}}
\def\Fg#1{\linkto{Fg.#1}{\use{Fg.#1}}}
\def\Tab#1{\linkto{Tb.#1}{Table~\use{Tb.#1}}}
\def\Tbl#1{\linkto{Tb.#1}{Table~\use{Tb.#1}}}
\def\Tb#1{\linkto{Tb.#1}{\use{Tb.#1}}}
\autoload\Tablebody{Tablebod.txs}\autoload\Tablebodyleft{Tablebod.txs}
\autoload\tablebody{Tablebod.txs}
\autoload\epsffile{epsf.tex}    \autoload\epsfbox{epsf.tex}
\autoload\epsfxsize{epsf.tex}   \autoload\epsfysize{epsf.tex}
\autoload\epsfverbosetrue{epsf.tex}\autoload\epsfverbosefalse{epsf.tex}
\obsolete\topFigure\figure \obsolete\midFigure\midfigure
\obsolete\fullFigure\fullfigure \obsolete\TOPFIGURE\figure
\obsolete\MIDFIGURE\midfigure \obsolete\FULLFIGURE\fullfigure
\obsolete\endFigure\endfigure \obsolete\ENDFIGURE\endfigure
\obsolete\topTable\toptable \obsolete\midTable\midtable
\obsolete\fullTable\fulltable \obsolete\TOPTABLE\toptable
\obsolete\MIDTABLE\midtable \obsolete\FULLTABLE\fulltable
\obsolete\endTable\endtable \obsolete\ENDTABLE\endtable
\def\FIG{\@obsolete\FIG\Fig\Fig}%
\def\TBL{\@obsolete\TBL\Tbl\Tbl}%

% file: TXSruled.tex (TeXsis version 2.17)
\catcode`@=11
\catcode`\|=12
\catcode`\&=4
\newcount\ncols         \ncols=\z@
\newcount\nrows         \nrows=\z@
\newcount\curcol        \curcol=\z@
\let\currow=\nrows
\newdimen\thinsize      \thinsize=0.6pt
\newdimen\thicksize     \thicksize=1.5pt
\newdimen\tablewidth    \tablewidth=-\maxdimen
\newdimen\parasize      \parasize=4in
\newif\iftableinfo      \tableinfotrue
\newif\ifcentertables   \centertablestrue
\def\centeredtables{\centertablestrue}%
\def\noncenteredtables{\centertablesfalse}%
\def\nocenteredtables{\centertablesfalse}%
\let\plaincr=\cr
\let\plainspan=\span
\let\plaintab=&
\def\ampersand{\char`\&}%
\let\lparen=(
\let\NX=\noexpand
\def\ruledtable{\relax
    \@BeginRuledTable
    \@RuledTable}%
\def\@BeginRuledTable{%
   \ncols=0\nrows=0
   \begingroup
    \offinterlineskip
    \def~{\phantom{0}}%
    \def\span{\plainspan\omit\relax\colcount\plainspan}%
    \let\cr=\crrule
    \let\CR=\crthick
    \let\nr=\crnorule
    \let\|=\Vb
    \def\hfill{\hskip0pt plus1fill\hbox{}}%
    \ifx\tablestrut\undefined\relax
    \else\let\tstrut=\tablestrut\fi
    \catcode`\|=13 \catcode`\&=13\relax
    \TableActive
    \curcol=1
    \ifdim\tablewidth>-\maxdimen\relax
      \edef\@Halign{\NX\halign to \NX\tablewidth\NX\bgroup\TablePreamble}%
      \tabskip=0pt plus 1fil
    \else
      \edef\@Halign{\NX\halign\NX\bgroup\TablePreamble}%
      \tabskip=0pt
    \fi
    \ifcentertables
       \ifhmode\vskip 0pt\fi
       \line\bgroup\hss
    \else\hbox\bgroup
    \fi}%
\long\def\@RuledTable#1\endruledtable{%
   \vrule width\thicksize
     \vbox{\@Halign
       \thickrule
       #1\killspace
       \tstrut
       \linecount
       \plaincr\thickrule
     \egroup}%
   \vrule width\thicksize
   \ifcentertables\hss\fi\egroup
  \endgroup
  \global\tablewidth=-\maxdimen
  \iftableinfo
      \immediate\write16{[Nrows=\the\nrows, Ncols=\the\ncols]}%
   \fi}%
\def\TablePreamble{%
   \TableItem{####}%
   \plaintab\plaintab
   \TableItem{####}%
   \plaincr}%
\def\@TableItem#1{%
   \hfil\tablespace
   #1\killspace
   \tablespace\hfil
    }%
\def\@tableright#1{%
   \hfil\tablespace\relax
   #1\killspace
   \tablespace\relax}%
\def\@tableleft#1{%
   \tablespace\relax
   #1\killspace
   \tablespace\hfil}%
\let\TableItem=\@TableItem
\def\RightJustifyTables{\let\TableItem=\@tableright}%
\def\LeftJustifyTables{\let\TableItem=\@tableleft}%
\def\NoJustifyTables{\let\TableItem=\@TableItem}%
\def\LooseTables{\let\tablespace=\quad}%
\def\TightTables{\let\tablespace=\space}%
\LooseTables
\def\TrailingSpaces{\let\killspace=\relax}%
\def\NoTrailingSpaces{\let\killspace=\unskip}%
\TrailingSpaces
\def\setRuledStrut{%
   \dimen@=\baselineskip
   \advance\dimen@ by-\normalbaselineskip
   \ifdim\dimen@<.5ex \dimen@=.5ex\fi
   \setbox0=\hbox{\lparen}%
   \dimen1=\dimen@ \advance\dimen1 by \ht0% space above line
   \dimen2=\dimen@ \advance\dimen2 by \dp0% space below line
   \def\tstrut{\vrule height\dimen1 depth\dimen2 width\z@}%
   }%
\def\tstrut{\vrule height 3.1ex depth 1.2ex width 0pt}%
\def\bigitem#1{%
   \dimen@=\baselineskip
   \advance\dimen@ by-\normalbaselineskip
   \ifdim\dimen@<.5ex \dimen@=.5ex\fi
   \setbox0=\hbox{#1}%
   \dimen1=\dimen@ \advance\dimen1 by \ht0
   \dimen2=\dimen@ \advance\dimen2 by \dp0
   \vrule height\dimen1 depth\dimen2 width\z@
   \copy0}%
\def\vctr#1{\hfil\vbox to 0pt{\vss\hbox{#1}\vss}\hfil}%
\def\nextcolumn#1{%
   \plaintab\omit#1\relax\colcount
   \plaintab}%
\def\tab{%
   \nextcolumn{\relax}}%
\let\novb=\tab
\def\vb{%
   \nextcolumn{\vrule width\thinsize}}%
\def\Vb{%
   \nextcolumn{\vrule width\thicksize}}%
\def\dbl{%
   \nextcolumn{\vrule width\thinsize
   \hskip 2\thinsize \vrule width\thinsize}}%
{\catcode`\|=13 \let|0
 \catcode`\&=13 \let&0
 \gdef\TableActive{\let|=\vb \let&=\tab}%
}%
\def\crrule{\killspace
   \tstrut
   \linecount
   \plaincr\tablerule
  }%
\def\crthick{\killspace
   \tstrut
   \linecount
   \plaincr\thickrule
  }%
\def\crnorule{\killspace
   \tstrut
   \linecount
   \plaincr
   }%
\def\crpart{\killspace
   \linecount
   \plaincr}%
\def\tablerule{\noalign{\hrule height\thinsize depth 0pt}}%
\def\thickrule{\noalign{\hrule height\thicksize depth 0pt}}%
\def\cskip{\omit\relax}%
\def\crule{\omit\leaders\hrule height\thinsize depth0pt\hfill}%
\def\Crule{\omit\leaders\hrule height\thicksize depth0pt\hfill}%
\def\linecount{%
   \global\advance\nrows by1
   \ifnum\ncols>0
      \ifnum\curcol=\ncols\relax\else
      \immediate\write16
      {\NX\ruledtable warning: Ncols=\the\curcol\space for Nrow=\the\nrows}%
      \fi\fi
   \global\ncols=\curcol
   \global\curcol=1}%
\def\colcount{\relax
   \global\advance\curcol by 1\relax}%
\long\def\para#1{%
   \vtop{\hsize=\parasize
   \normalbaselines
   \noindent #1\relax
   \vrule width 0pt depth 1.1ex}%
}%
\def\begintable{\relax
    \@BeginRuledTable
    \@begintable}%
\long\def\@begintable#1\endtable{%
   \@RuledTable#1\endruledtable}%

% file: TXSsymb.tex (TeXsis version 2.17)
%%> {Extended math/physics symbols,}
\def\E#1{\hbox{$\times 10^{#1}$}}
\def\square{\hbox{{$\sqcup$}\llap{$\sqcap$}}}%
\def\grad{\nabla}%
\def\del{\partial}%
\def\frac#1#2{{#1\over#2}}
\def\smallfrac#1#2{{\scriptstyle {#1 \over #2}}}
\def\half{\ifinner {\scriptstyle {1 \over 2}}%
          \else {\textstyle {1 \over 2}}\fi}
\def\bra#1{\langle#1\vert}%
\def\ket#1{\vert#1\/\rangle}%
\def\vev#1{\langle{#1}\rangle}%
\def\simge{%
    \mathrel{\rlap{\raise 0.511ex 
        \hbox{$>$}}{\lower 0.511ex \hbox{$\sim$}}}}
\def\simle{%
    \mathrel{\rlap{\raise 0.511ex 
        \hbox{$<$}}{\lower 0.511ex \hbox{$\sim$}}}}
\def\gtsim{\simge}%
\def\ltsim{\simle}%
\def\therefore{%
   \setbox0=\hbox{$.\kern.2em.$}\dimen0=\wd0
   \mathrel{\rlap{\raise.25ex\hbox to\dimen0{\hfil$\cdotp$\hfil}}%
   \copy0}}
\def\|{\ifmmode\Vert\else \char`\|\fi}          
\def\sterling{{\hbox{\it\char'44}}}     
\def\degrees{\hbox{$^\circ$}}%
\def\degree{\degrees}%
\def\real{\mathop{\rm Re}\nolimits}%
\def\imag{\mathop{\rm Im}\nolimits}%
\def\tr{\mathop{\rm tr}\nolimits}%
\def\Tr{\mathop{\rm Tr}\nolimits}%
\def\Det{\mathop{\rm Det}\nolimits}%
\def\mod{\mathop{\rm mod}\nolimits}%
\def\wrt{\mathop{\rm wrt}\nolimits}%
\def\diag{\mathop{\rm diag}\nolimits}%
\def\TeV{{\rm TeV}}%
\def\GeV{{\rm GeV}}%
\def\MeV{{\rm MeV}}%
\def\keV{{\rm keV}}%
\def\eV{{\rm eV}}%
\def\Ry{{\rm Ry}}%
\def\mb{{\rm mb}}%
\def\mub{\hbox{\rm $\mu$b}}%
\def\nb{{\rm nb}}%
\def\pb{{\rm pb}}%
\def\fb{{\rm fb}}%
\def\cmsec{{\rm cm^{-2}s^{-1}}}%
\def\units#1{\hbox{\rm #1}} 
\let\unit=\units
\def\dimensions#1#2{\hbox{$[\hbox{\rm #1}]^{#2}$}}
\def\parenbar#1{{\null\!
   \mathop{\smash#1}\limits
   ^{\hbox{\fiverm(--)}}%
   \!\null}}%
\def\nunubar{\parenbar{\nu}}
\def\ppbar{\parenbar{p}}
\def\buildchar#1#2#3{{\null\!
   \mathop{\vphantom{#1}\smash#1}\limits
   ^{#2}_{#3}%
   \!\null}}%
\def\overcirc#1{\buildchar{#1}{\circ}{}}
\def\sun{{\hbox{$\odot$}}}\def\earth{{\hbox{$\oplus$}}}
\def\slashchar#1{\setbox0=\hbox{$#1$}%
   \dimen0=\wd0
   \setbox1=\hbox{/} \dimen1=\wd1
   \ifdim\dimen0>\dimen1
      \rlap{\hbox to \dimen0{\hfil/\hfil}}%
      #1
   \else
      \rlap{\hbox to \dimen1{\hfil$#1$\hfil}}%
      /
   \fi}%
\def\subrightarrow#1{%
  \setbox0=\hbox{%
    $\displaystyle\mathop{}%
    \limits_{#1}$}%
  \dimen0=\wd0
  \advance \dimen0 by .5em
  \mathrel{%
    \mathop{\hbox to \dimen0{\rightarrowfill}}%
       \limits_{#1}}}%
\newdimen\vbigd@men
\def\vbigl{\mathopen\vbig}
\def\vbigm{\mathrel\vbig}
\def\vbigr{\mathclose\vbig}
\def\vbig#1#2{{\vbigd@men=#2\divide\vbigd@men by 2
   \hbox{$\left#1\vbox to \vbigd@men{}\right.\n@space$}}}
\def\Leftcases#1{\smash{\vbigl\{{#1}}}
\def\Rightcases#1{\smash{\vbigr\}{#1}}}
%% $Revision: 17.4 $  :  $Date: 1999/04/07 01:05:56 $  :  $Author: myers $
%======================================================================*
% First, null definitions for ``missing'' things from other parts of TeXsis
%
% Make double column stuff do nothing nicely:
%
\def\doublecolumns{\relax}\def\enddoublecolumns{\relax}
\def\leftcolrule{\relax}\def\rightcolrule{\relax}
\def\longequation{\relax}\def\endlongequation{\relax}
\def\newcolumn{\relax}
\def\widetopinsert{\topinsert}\def\widepageinsert{\pageinsert}
\def\forceleft{\relax}\def\forceright{\relax}   
\def\SetDoubleColumns#1{%
  \imsg{The double column macros are not a part of mTeXsis.}
  \imsg{If you want to use double column mode, get TXSdcol.tex}
  \imsg{and add \string\input\space TXSdcol.tex to your .tex file.}
}

% ...and other stuff that's not in mTeXsis:

\def\addTOC#1#2#3{\relax}\def\Contents{\relax}  % disable table of contents
\newif\ifContents                               % disable table of contents
\def\ContentsSwitchtrue{\Contentstrue}\def\ContentsSwitchfalse{\Contentsfalse}

\def\obsolete#1#2{\let#1=#2\relax #2}           % \obsolete is silent here

\let\Input=\input                               % TeXsis synonym
\newdimen\colwidth      \colwidth=\hsize        % default column width
\def\ORGANIZATION{}%                            % default is empty

% You shouldn't use style files that might not be around when using
% mTeXsis.  But we will try to be friendly about it...

\newhelp\@utohelp{%
loadstyle: The definition of the macro named above is actually contained^^J%
in a style file, and so it cannot be used with mTeXsis.  If you really^^J%
need to load the definition from that file, you should do so explicitly^^J%
at the begining of your manuscript file, with %
    '\string\input\space stylefilename.txs'^^J}

\Ignore
\def\loadstyle#1#2{% disable \loadstyle in mTeXsis
   \newlinechar=10                              % ^^J is line break
   \errhelp=\@utohelp                           % longer help message
   \emsg{> Whoops! Trying to load \string#1\space from style file #2.}%
   \errmessage{You cannot use macro definitions from style files in mTeXsis}}
\endIgnore

% -- These make mtexsis more friendly than TeX or TeXsis when printing
%    e-prints, but you won't find your errors as easily.

\hbadness=10000         % don't complain about overfull hboxes!
\overfullrule=0pt       % and don't mark them in the output!
\vbadness=10000         % and don't complain about vboxes either.

% -- Run time setup (in place of \everyjob):

\ATunlock
\SetDate                                % sets \adate and \edate
\ReadAUX                                % reads .aux file, if it exists
\def\fmtname{TeXsis}\def\fmtversion{2.17}%
\def\revdate{1 January 1998}%
\def\imsg#1{\emsg{\@comment #1}}%
\imsg{=========================================================== \@comment}
\imsg{This is mTeXsis, the core macros from TeXsis.}
\imsg{You can get the complete TeXsis package (and avoid this annoying}
\imsg{advertisement) from ftp://lifshitz.ph.utexas.edu/texsis, }
\imsg{or from a CTAN server near you (in macros/texsis).}
\imsg{See the README and INSTALL files there for more information.}
\imsg{============================================================ \@comment}
\emsg{m\fmtname\space version \fmtversion\space (\revdate)  loaded.}%
\ATlock                                 % lock internal @ macros
\texsis                                 % assumed, since they asked for it
%%>>> EOF mtexsis.tex <<<
%%%%%%%%%%%%%%%%%%%%%%%%%%%%%%%%%%%%%%%%%%%%%%%%%%%%%%%%%%%%%%
%% begin bookmacros.tex
\quoteoff
%redefinegreek.tex
%Redefine upper case greek letters so that particle names are slanted
%Make new definitions to correspond to old defintions
\global\mathchardef\DELTA="7001
\global\mathchardef\LAMBDA="7003
\global\mathchardef\XI="7004
\global\mathchardef\SIGMA="7006
\global\mathchardef\UPSILON="7007
\global\mathchardef\OMEGA="700A
\global\mathchardef\Delta="7101
\global\mathchardef\Lambda="7103
\global\mathchardef\Xi="7104
\global\mathchardef\Sigma="7106
\global\mathchardef\Upsilon="7107
\global\mathchardef\Omega="710A
%%%%%%%%%%%%%%%%%%%%%%%%%%%%%%%%%%%%%%%%%%%%%%%%%%%%%%%%%%%%%%%%%
%%%%%%%%%%%%%%%%%%%%%%%%%%%%%%%%%%%
%\input [bettya.lbl100.sports.macros]mssymb

%		*****	  MSSYMB.TeX	*****		       4 Nov 85
%
%	This file contains the definitions for the symbols in the two
%	"extra symbols" fonts created at the American Math. Society.

\catcode`\@=11

\font\tenmsa=msam10
\font\sevenmsa=msam7
\font\fivemsa=msam5
\font\tenmsb=msbm10
\font\sevenmsb=msbm7
\font\fivemsb=msbm5
\newfam\msafam
\newfam\msbfam
\textfont\msafam=\tenmsa  \scriptfont\msafam=\sevenmsa
  \scriptscriptfont\msafam=\fivemsa
\textfont\msbfam=\tenmsb  \scriptfont\msbfam=\sevenmsb
  \scriptscriptfont\msbfam=\fivemsb

\def\hexnumber@#1{\ifnum#1<10 \number#1\else
 \ifnum#1=10 A\else\ifnum#1=11 B\else\ifnum#1=12 C\else
 \ifnum#1=13 D\else\ifnum#1=14 E\else\ifnum#1=15 F\fi\fi\fi\fi\fi\fi\fi}

\def\msa@{\hexnumber@\msafam}
\def\msb@{\hexnumber@\msbfam}
\global\mathchardef\boxdot="2\msa@00
\global\mathchardef\boxplus="2\msa@01
\global\mathchardef\boxtimes="2\msa@02
\global\mathchardef\square="0\msa@03
\global\mathchardef\blacksquare="0\msa@04
\global\mathchardef\centerdot="2\msa@05
\global\mathchardef\lozenge="0\msa@06
\global\mathchardef\blacklozenge="0\msa@07
\global\mathchardef\circlearrowright="3\msa@08
\global\mathchardef\circlearrowleft="3\msa@09
\global\mathchardef\rightleftharpoons="3\msa@0A
\global\mathchardef\leftrightharpoons="3\msa@0B
\global\mathchardef\boxminus="2\msa@0C
\global\mathchardef\Vdash="3\msa@0D
\global\mathchardef\Vvdash="3\msa@0E
\global\mathchardef\vDash="3\msa@0F
\global\mathchardef\twoheadrightarrow="3\msa@10
\global\mathchardef\twoheadleftarrow="3\msa@11
\global\mathchardef\leftleftarrows="3\msa@12
\global\mathchardef\rightrightarrows="3\msa@13
\global\mathchardef\upuparrows="3\msa@14
\global\mathchardef\downdownarrows="3\msa@15
\global\mathchardef\upharpoonright="3\msa@16
\let\restriction=\upharpoonright
\global\mathchardef\downharpoonright="3\msa@17
\global\mathchardef\upharpoonleft="3\msa@18
\global\mathchardef\downharpoonleft="3\msa@19
\global\mathchardef\rightarrowtail="3\msa@1A
\global\mathchardef\leftarrowtail="3\msa@1B
\global\mathchardef\leftrightarrows="3\msa@1C
\global\mathchardef\rightleftarrows="3\msa@1D
\global\mathchardef\Lsh="3\msa@1E
\global\mathchardef\Rsh="3\msa@1F
\global\mathchardef\rightsquigarrow="3\msa@20
\global\mathchardef\leftrightsquigarrow="3\msa@21
\global\mathchardef\looparrowleft="3\msa@22
\global\mathchardef\looparrowright="3\msa@23
\global\mathchardef\circeq="3\msa@24
\global\mathchardef\succsim="3\msa@25
\global\mathchardef\gtrsim="3\msa@26
\global\mathchardef\gtrapprox="3\msa@27
\global\mathchardef\multimap="3\msa@28
\global\mathchardef\therefore="3\msa@29
\global\mathchardef\because="3\msa@2A
\global\mathchardef\doteqdot="3\msa@2B
\let\Doteq=\doteqdot
\global\mathchardef\triangleq="3\msa@2C
\global\mathchardef\precsim="3\msa@2D
\global\mathchardef\lesssim="3\msa@2E
\global\mathchardef\lessapprox="3\msa@2F
\global\mathchardef\eqslantless="3\msa@30
\global\mathchardef\eqslantgtr="3\msa@31
\global\mathchardef\curlyeqprec="3\msa@32
\global\mathchardef\curlyeqsucc="3\msa@33
\global\mathchardef\preccurlyeq="3\msa@34
\global\mathchardef\leqq="3\msa@35
\global\mathchardef\leqslant="3\msa@36
\global\mathchardef\lessgtr="3\msa@37
\global\mathchardef\backprime="0\msa@38
\global\mathchardef\risingdotseq="3\msa@3A
\global\mathchardef\fallingdotseq="3\msa@3B
\global\mathchardef\succcurlyeq="3\msa@3C
\global\mathchardef\geqq="3\msa@3D
\global\mathchardef\geqslant="3\msa@3E
\global\mathchardef\gtrless="3\msa@3F
\global\mathchardef\sqsubset="3\msa@40
\global\mathchardef\sqsupset="3\msa@41
%\global\mathchardef ="3\msa@42          (\triangleright)
%\global\mathchardef ="3\msa@43          (\triangleleft)
\global\mathchardef\trianglerighteq="3\msa@44
\global\mathchardef\trianglelefteq="3\msa@45
\global\mathchardef\bigstar="0\msa@46
\global\mathchardef\between="3\msa@47
\global\mathchardef\blacktriangledown="0\msa@48
\global\mathchardef\blacktriangleright="3\msa@49
\global\mathchardef\blacktriangleleft="3\msa@4A
%\global\mathchardef ="3\msa@4D          (\triangle)
\global\mathchardef\blacktriangle="0\msa@4E
\global\mathchardef\triangledown="0\msa@4F
\global\mathchardef\eqcirc="3\msa@50
\global\mathchardef\lesseqgtr="3\msa@51
\global\mathchardef\gtreqless="3\msa@52
\global\mathchardef\lesseqqgtr="3\msa@53
\global\mathchardef\gtreqqless="3\msa@54
\global\mathchardef\Rrightarrow="3\msa@56
\global\mathchardef\Lleftarrow="3\msa@57
\global\mathchardef\veebar="2\msa@59
\global\mathchardef\barwedge="2\msa@5A
\global\mathchardef\doublebarwedge="2\msa@5B
\global\mathchardef\angle="0\msa@5C
\global\mathchardef\measuredangle="0\msa@5D
\global\mathchardef\sphericalangle="0\msa@5E
\global\mathchardef\varpropto="3\msa@5F
\global\mathchardef\smallsmile="3\msa@60
\global\mathchardef\smallfrown="3\msa@61
\global\mathchardef\Subset="3\msa@62
\global\mathchardef\Supset="3\msa@63
\global\mathchardef\Cup="2\msa@64
\let\doublecup=\Cup
\global\mathchardef\Cap="2\msa@65
\let\doublecap=\Cap
\global\mathchardef\curlywedge="2\msa@66
\global\mathchardef\curlyvee="2\msa@67
\global\mathchardef\leftthreetimes="2\msa@68
\global\mathchardef\rightthreetimes="2\msa@69
\global\mathchardef\subseteqq="3\msa@6A
\global\mathchardef\supseteqq="3\msa@6B
\global\mathchardef\bumpeq="3\msa@6C
\global\mathchardef\Bumpeq="3\msa@6D
\global\mathchardef\lll="3\msa@6E
\let\llless=\lll
\global\mathchardef\ggg="3\msa@6F
\let\gggtr=\ggg
\global\mathchardef\circledS="0\msa@73
\global\mathchardef\pitchfork="3\msa@74
\global\mathchardef\dotplus="2\msa@75
\global\mathchardef\backsim="3\msa@76
\global\mathchardef\backsimeq="3\msa@77
\global\mathchardef\complement="0\msa@7B
\global\mathchardef\intercal="2\msa@7C
\global\mathchardef\circledcirc="2\msa@7D
\global\mathchardef\circledast="2\msa@7E
\global\mathchardef\circleddash="2\msa@7F
\def\ulcorner{\delimiter"4\msa@70\msa@70 }
\def\urcorner{\delimiter"5\msa@71\msa@71 }
\def\llcorner{\delimiter"4\msa@78\msa@78 }
\def\lrcorner{\delimiter"5\msa@79\msa@79 }
\def\yen{\mathhexbox\msa@55 }
\def\checkmark{\mathhexbox\msa@58 }
\def\circledR{\mathhexbox\msa@72 }
\def\maltese{\mathhexbox\msa@7A }
\global\mathchardef\lvertneqq="3\msb@00
\global\mathchardef\gvertneqq="3\msb@01
\global\mathchardef\nleq="3\msb@02
\global\mathchardef\ngeq="3\msb@03
\global\mathchardef\nless="3\msb@04
\global\mathchardef\ngtr="3\msb@05
\global\mathchardef\nprec="3\msb@06
\global\mathchardef\nsucc="3\msb@07
\global\mathchardef\lneqq="3\msb@08
\global\mathchardef\gneqq="3\msb@09
\global\mathchardef\nleqslant="3\msb@0A
\global\mathchardef\ngeqslant="3\msb@0B
\global\mathchardef\lneq="3\msb@0C
\global\mathchardef\gneq="3\msb@0D
\global\mathchardef\npreceq="3\msb@0E
\global\mathchardef\nsucceq="3\msb@0F
\global\mathchardef\precnsim="3\msb@10
\global\mathchardef\succnsim="3\msb@11
\global\mathchardef\lnsim="3\msb@12
\global\mathchardef\gnsim="3\msb@13
\global\mathchardef\nleqq="3\msb@14
\global\mathchardef\ngeqq="3\msb@15
\global\mathchardef\precneqq="3\msb@16
\global\mathchardef\succneqq="3\msb@17
\global\mathchardef\precnapprox="3\msb@18
\global\mathchardef\succnapprox="3\msb@19
\global\mathchardef\lnapprox="3\msb@1A
\global\mathchardef\gnapprox="3\msb@1B
\global\mathchardef\nsim="3\msb@1C
\global\mathchardef\napprox="3\msb@1D
%\global\mathchardef ="3\msb@20          (\subsetneq)
%\global\mathchardef ="3\msb@21          (\supsetneq)
\global\mathchardef\nsubseteqq="3\msb@22
\global\mathchardef\nsupseteqq="3\msb@23
\global\mathchardef\subsetneqq="3\msb@24
\global\mathchardef\supsetneqq="3\msb@25
%\global\mathchardef ="3\msb@26          (\subsetneqq)
%\global\mathchardef ="3\msb@27          (\supsetneqq)
\global\mathchardef\subsetneq="3\msb@28
\global\mathchardef\supsetneq="3\msb@29
\global\mathchardef\nsubseteq="3\msb@2A
\global\mathchardef\nsupseteq="3\msb@2B
\global\mathchardef\nparallel="3\msb@2C
\global\mathchardef\nmid="3\msb@2D
\global\mathchardef\nshortmid="3\msb@2E
\global\mathchardef\nshortparallel="3\msb@2F
\global\mathchardef\nvdash="3\msb@30
\global\mathchardef\nVdash="3\msb@31
\global\mathchardef\nvDash="3\msb@32
\global\mathchardef\nVDash="3\msb@33
\global\mathchardef\ntrianglerighteq="3\msb@34
\global\mathchardef\ntrianglelefteq="3\msb@35
\global\mathchardef\ntriangleleft="3\msb@36
\global\mathchardef\ntriangleright="3\msb@37
\global\mathchardef\nleftarrow="3\msb@38
\global\mathchardef\nrightarrow="3\msb@39
\global\mathchardef\nLeftarrow="3\msb@3A
\global\mathchardef\nRightarrow="3\msb@3B
\global\mathchardef\nLeftrightarrow="3\msb@3C
\global\mathchardef\nleftrightarrow="3\msb@3D
\global\mathchardef\divideontimes="2\msb@3E
\global\mathchardef\varnothing="0\msb@3F
\global\mathchardef\nexists="0\msb@40
\global\mathchardef\mho="0\msb@66
\global\mathchardef\thorn="0\msb@67
\global\mathchardef\beth="0\msb@69
\global\mathchardef\gimel="0\msb@6A
\global\mathchardef\daleth="0\msb@6B
\global\mathchardef\lessdot="3\msb@6C
\global\mathchardef\gtrdot="3\msb@6D
\global\mathchardef\ltimes="2\msb@6E
\global\mathchardef\rtimes="2\msb@6F
\global\mathchardef\shortmid="3\msb@70
\global\mathchardef\shortparallel="3\msb@71
\global\mathchardef\smallsetminus="2\msb@72
\global\mathchardef\thicksim="3\msb@73
\global\mathchardef\thickapprox="3\msb@74
\global\mathchardef\approxeq="3\msb@75
\global\mathchardef\succapprox="3\msb@76
\global\mathchardef\precapprox="3\msb@77
\global\mathchardef\curvearrowleft="3\msb@78
\global\mathchardef\curvearrowright="3\msb@79
\global\mathchardef\digamma="0\msb@7A
\global\mathchardef\varkappa="0\msb@7B
\global\mathchardef\hslash="0\msb@7D
\global\mathchardef\hbar="0\msb@7E
\global\mathchardef\backepsilon="3\msb@7F
% Use the next 4 lines with AMS-TeX:
%\def\Bbb{\relaxnext@\ifmmode\let\next\Bbb@\else
% \def\next{\Err@{Use \string\Bbb\space only in math mode}}\fi\next}
%\def\Bbb@#1{{\Bbb@@{#1}}}
%\def\Bbb@@#1{\noaccents@\fam\msbfam#1}
% Use the next 4 lines if NOT using AMS-TeX:
\def\Bbb{\ifmmode\let\next\Bbb@\else
 \def\next{\errmessage{Use \string\Bbb\space only in math mode}}\fi\next}
\def\Bbb@#1{{\Bbb@@{#1}}}
\def\Bbb@@#1{\fam\msbfam#1}
%%%%%%%%%%%%%%%%%%%%%%%%%%%%%%%%%%%%%%%%%%%%%%%%%%%%%%%%%
%%%%%%%%%%%%%%%%%%%%%%%%%%%%%%%%%%%%%%%%%%%%%%%%%%%%%%%%%

\catcode`\@=12
%         rpp_txspatch.tex
%rpp_txspatch.tex
%%%%%%%%%%%%%%%%%%%%%%%%%%%%%%%%%%%%%%%%%%%%%%%%%%%%%%%%%
%These macros are designed to be used both with the Sports Section
%and with the minireviews.
%They contain macros that redefine displayed and aligned equations
%allowing for non-centered equations.
%New macros are defined in the spirit of \Ref and \Eq to allow
%for the labeling and referencing of Chapters, Sections, Figure ranges, etc.
%Several macros are defined that increase counts and label Figures, Tables,
%Equations, Sections, Chapters, etc. that are bypassed by the Databook.
%11pt, 10pt, 12pt are redefined to take advantage of the `correct' cmr
%fonts available at LBL.
%New enumerated lists are defined in the TEXSIS style using, for example
%lower case alpha.
%Minor changes to \etal, \eg, \cf, \etc, \today 
%%%%%%%%%%%%%%%%%%%%%%%%%%%%%%%%%%%%%%%%%%%%%%%%%%%%%%%%%
%BEGINNING OF RPPPATCH MACROS:
%
\ATunlock       % make @ a letter for "hidden" macro names
%
%%%%%%%%%%%%%%%%%%%%%%%%%%%%%%%%%%%%%%%%%%%%%%%%%%%%%%%%%
\def\refFormat{\catcode`\^^M=10}
\def\Refs#1#2{Refs.~\use{Ref.#1},\use{Ref.#2}}
\def\Refsand#1#2{Refs.~\use{Ref.#1} and~\use{Ref.#2}}
\def\Refsrange#1#2{Refs.~\use{Ref.#1}--\use{Ref.#2}}
\superrefsfalse % gives references in square brackets
%%%%%%%%%%%%%%%%%%%%%%%%%%%%%%%%%%%%%%%%%%%%%%%%%%%%%%%%%
\def\Chapter#1{Chapter~\use{Chap.#1}}
\def\Section#1{Section~\use{Sec.#1}}
\def\Chap#1{Chap.~\use{Chap.#1}}
\def\Sec#1{Sec.~\use{Sec.#1}}
\def\Table#1{Table~\use{Tb.#1}}
\def\Equation#1{Equation~\use{Eq.#1}}
\def\Figure#1{Figure~\use{Fg.#1}}
\def\Figsrange#1#2{Figs.~\use{Fg.#1}--\use{Fg.#2}}
\def\Reference#1{Reference~\use{Ref.#1}}
%%%%%%%%%%%%%%%%%%%%%%%%%%%%%%%%%%%%%%%%%%%%%%%%%%%%%%%%%
\def\Eqsrange#1#2{Eqs.~(\use{Eq.#1})--(\use{Eq.#2})}
%%%%%%%%%%%%%%%%%%%%%%%%%%%%%%%%%%%%%%%%%%%%%%%%%%%%%%%%%
%
%
%%%%%%%%%%%%%%%%%%%%%%%%%%%%%%%%%%%%%%%%%%%%%%%%%%%%%%%%%
\def\eqReset{\global\eqnum=0}
%
%%%%%%%%%%%%%%%%%%%%%%%%%%%%%%%%%%%%%%%%%%%%%%%%%%%%%
%FOR DATABOOK TO BUMP UP NUMBERS THAT ARE GONE AROUND
\def\bumpupequationnumber{\global\advance\eqnum by 1\relax}
\def\bumpupchapternumber{\global\advance\chapternum by 1\relax}
\def\bumpupsectionnumber{\global\advance\sectionnum by 1\relax}
\def\bumpupsectionnumber{\global\Resetsection}
\def\bumpupsubsectionnumber{\global\advance\subsectionnum by 1\relax}
\def\bumpupfigurenumber{\global\advance\fignum by 1\relax}
\def\bumpuptablenumber{\global\advance\tabnum by 1\relax}
\def\bumpdownequationnumber{\global\advance\eqnum by -1\relax}
\def\bumpdownchapternumber{\global\advance\chapternum by -1\relax}
\def\bumpdownsectionnumber{\global\advance\sectionnum by -1\relax}
\def\bumpdownsubsectionnumber{\global\advance\subsectionnum by -1\relax}
\def\bumpdownfigurenumber{\global\advance\fignum by -1\relax}
\def\bumpdowntablenumber{\global\advance\tabnum by -1\relax}
\def\labelfigure#1{\tag{Fg.#1}{\the\chapternum.\the\fignum}}
\def\labeltable#1{\tag{Tb.#1}{\the\chapternum.\the\tabnum}}
\def\labelequation#1{\tag{Eq.#1}{\the\chapternum.\the\eqnum}}
\def\labelchapter#1{\label{Chap.#1}}
\def\labelsection#1{\tag{Sec.#1}{\the\chapternum.\the\sectionnum}}
\def\labelsubsection#1{\tag{Sec.#1}{\the\chapternum.\the\sectionnum.%
\the\subsectionnum}}
\def\labelsubsubsection#1{\tag{Sec.#1}{\the\chapternum.\the\sectionnum.%
\the\subsectionnum.\the\subsubsectionnum}}
%%%%%%%%%%%%%%%%%%%%%%%%%%%%%%%%%%%%%%%%%%%%%%%%%%%%%
%%%%%%%%%%%%%%%%%%%%%%%%%%%%%%%%%%%%%%%%%%%%%%%%%%%%%%%%%%%%%%%%%
%
%
%%%%%%%%%%%%%%%%%%%%%%%%%%%%%%%%%%%%%%%%%%%%%%%%%%%%%%%%%
%And just do the same to \EQNalign and \EQNdoublealign
%----------------------------------------------------------------------*
%  \RPPdisplaylines is like \displaylines, as described in 
%  The TeXbook, except that you can also use \EQN for equation numbers.
%  And they can be easily left-adjusted using \lefteqnside and \righteqnside
%  syntax: "\EQNdisplaylines{... <equation> \EQN <label> \cr ...ditto...\cr}"
%%%%%%%%%%%%%%%%%%%%%%%%%%%%%%%%%%%%%%%%%%%%%%%
%THESE DEFINITIONS ARE TO BE USED TO GIVE MORE SPACE in RPPdisplaylines,
%RPPalign, and  RPPdoublealign
\def\crsmallskip{\cr\noalign{\smallskip}}
\def\crmedskip{\cr\noalign{\medskip}}
\def\crbigskip{\cr\noalign{\bigskip}}
\def\crskip{\cr\noalign{\smallskip}}
%
%------
\newskip\lefteqnside
\newskip\righteqnside
\newdimen\lefteqnsidedimen \lefteqnsidedimen=22pt % FOR SPORTS SECTION
\lefteqnside =0pt\relax\righteqnside=0pt plus 1fil  % would get you left flush
\lefteqnside=0pt plus 1fil\relax\righteqnside =0pt % would get you right flush
\lefteqnside =0pt plus 1fil % equivalent to EQNdisplaylines
\righteqnside=0pt plus 1fil % equivalent to EQNdisplaylines
\lefteqnsidedimen=30pt % FOR MINIREVIEW etc.
\lefteqnside =30pt % RPP default
\righteqnside=0pt plus 1fil  %  RPP default
%------
\def\RPPdisplaylines#1{%                % make \EQN re-\def local
      \@EQNcr                             % change \def of \EQN to get label
    \openup 2\jot
    \displ@y                            % reduce interline spacing (from Plain)
   \halign{\hbox to \displaywidth{$\relax\hskip\lefteqnside{\displaystyle##}%
               \hskip\righteqnside$}%
   &\llap{$\relax\@@EQN{##}$}\crcr      % then eqn number template
    #1\crcr}%                           % now apply template to argument
    \@EQNuncr                          % put \EQN back to normal
    }

%  \RPPalign is just like \eqalign except that you can use \EQN to get
%  equation numbers.  Just put ...\EQN <label> \cr on a line you want to get
%  a label.
%  Allows RPP left-adjusted equations

\long\def\RPPalign#1{%  replacement for \eqalign using \EQN
  \@EQNcr                               % re-def \EQN to see label before \cr
    \openup 2\jot
   \displ@y                              % reduce interline spacing
%     \tabskip=\centering                 % leftskip for LHS
     \tabskip=\lefteqnside                 % leftskip for LHS
   \halign to\displaywidth{%             % alignment
   \hfil$\relax\displaystyle{##}$%       % template for LHS
     \tabskip=0pt                        % no skip between LHS and RHS
   &$\leavevmode\relax\displaystyle{{}##}$\hfil     % template for RHS
%      \tabskip=\centering               % rightskip for RHS
     \tabskip=\righteqnside                 % leftskip for LHS
  &\llap{$\relax\@@EQN{##}$}%           % template for eqn number, parses label
     \tabskip=0pt\crcr                   % no skip at end of eqn number
    #1\crcr}%                            % now apply template to argument
   }


                                        
%  \RPPdoublealign makes a double equation alignment, i.e. three
%  columns. \EQN works as usual.
%  Allows RPP left-adjusted equations

\def\RPPdoublealign#1{%                 % make re-def of \EQN local
   \@EQNcr                              % re-def \EQN to see label before \cr
    \openup 2\jot
   \displ@y                             % reduce interline spacing
%   \tabskip=\centering                  % leftskip for LHS
     \tabskip=\lefteqnside                 % leftskip for LHS
   \halign to\displaywidth{%            %
      \hfil$\relax\displaystyle{##}$%   % template for LHS
      \tabskip=0pt                      % no skip between LHS and middle
   &$\relax\displaystyle{{}##}$\hfil%   % template for middle
      \tabskip=0pt                      % no skip between middle and RHS
   &$\relax\displaystyle{{}##}$\hfil%   % template for RHS
%      \tabskip=\centering               % rightskip for RHS
     \tabskip=\righteqnside                 % leftskip for LHS
   &\llap{$\relax\@@EQN{##}$}%          % eqn number template parses label
      \tabskip=0pt\crcr                 % no skip at end of eqn number
   #1\crcr}%                            % now apply template to argument
   \@EQNuncr                          % put \EQN back to normal
   }%

%----------------------------------------------------------------------*


%----------------------------------------------------------------------*
\def\today{\ifcase\month\or%gives you todays date
  January\or February\or March\or April\or May\or June\or
  July\or August\or September\or October\or November\or December\fi
  \space\number\day, \number\year}
\def\fildec#1{\ifnum#1<10 0\fi\the#1}
\newcount\hour \newcount\minute
\def\TimeOfDay%
{% day clock by A. Ogawa
   \hour\time\divide\hour by 60
   \minute-\hour\multiply\minute by 60 \advance\minute\time
   \fildec\hour:\fildec\minute
}
\let\thetime=\TimeOfDay
%----------------------------------------------------------------------*
\ATlock         % make @ not a letter (hides internal macros).
\def\eg{\hbox{\it e.g.}}     \def\cf{\hbox{\it cf.}}
\def\etc{\hbox{\it etc.}}     \def\ie{\hbox{\it i.e.}}
\def\vs{{\it vs.}}
%----------------------------------------------------------------------*
%----------------------------------------------------------------------*
%FONT REDEFINTIONS
%
% USE CORRECT font for elevenex
%
%%++ 11 pt fonts:
\def\elevenfonts{%
   \global\font\elevenrm=cmr10 scaled \magstephalf
   \global\font\eleveni=cmmi10 scaled \magstephalf
   \global\font\elevensy=cmsy10 scaled \magstephalf
   \global\font\elevenex=cmex10 scaled \magstephalf
   \global\font\elevenbf=cmbx10 scaled \magstephalf
   \global\font\elevensl=cmsl10 scaled \magstephalf
   \global\font\eleventt=cmtt10 scaled \magstephalf
   \global\font\elevenit=cmti10 scaled \magstephalf
   \global\font\elevenss=cmss10 scaled \magstephalf
   \global\font\elevenbxti=cmbxti10 scaled \magstephalf
   \skewchar\eleveni='177%%
   \skewchar\elevensy='60%%
   \hyphenchar\eleventt=-1%%
   \moreelevenfonts                            % any other custom fonts
   \gdef\elevenfonts{\relax}}%

\def\moreelevenfonts{\relax}                    % User customization hook

%%++ 12 pt fonts:
%
% USE 12 point fonts where available
%
\def\twelvefonts{%  initialize 12pt fonts
   \global\font\twelverm=cmr12 
   \global\font\twelvei=cmmi10 scaled \magstep1%%
   \global\font\twelvesy=cmsy10 scaled \magstep1%%
   \global\font\twelveex=cmex10 scaled \magstep1%%
   \global\font\twelvebf=cmbx12
   \global\font\twelvesl=cmsl12
   \global\font\twelvett=cmtt12
   \global\font\twelveit=cmti12
   \global\font\twelvess=cmss12
   \skewchar\twelvei='177%%
   \skewchar\twelvesy='60%%
   \hyphenchar\twelvett=-1%%
   \moretwelvefonts                             % any other custom fonts
   \gdef\twelvefonts{\relax}}

\def\moretwelvefonts{\relax}

%========================================================================
% FONT SIZES
%       Macros to change font sizes. Each of these load fonts the first
% time it is called.
% 
% REDEFINE 10 point so that subscripts are size 8 and subsub are size 7
%
\newskip\strutskip
\def\strut{\vrule height 0.8\strutskip depth 0.3\strutskip width 0pt}

%       Switch to 10 point type
\message{10pt,}
\def\tenpoint{%                         % 10pt already loaded by default
   \def\rm{\fam0\tenrm}%
   \textfont0=\tenrm\scriptfont0=\eightrm\scriptscriptfont0=\sevenrm
   \textfont1=\teni\scriptfont1=\eighti\scriptscriptfont1=\seveni
   \textfont2=\tensy\scriptfont2=\eightsy\scriptscriptfont2=\sevensy
% 
%  USE eightex and sevenex
% 
   \textfont3=\tenex\scriptfont3=\eightex\scriptscriptfont3=\sevenex
   \textfont4=\tenit\scriptfont4=\eightit\scriptscriptfont4=\sevenit
   \textfont\itfam=\tenit\def\it{\fam\itfam\tenit}%
   \textfont\slfam=\tensl\def\sl{\fam\slfam\tensl}%
   \textfont\ttfam=\tentt\def\tt{\fam\ttfam\tentt}%
   \textfont\bffam=\tenbf
   \scriptfont\bffam=\eightbf
   \scriptscriptfont\bffam=\sevenbf\def\bf{\fam\bffam\tenbf}%
% 
% REDEFINE 10mib point so that subscripts are size 8 and subsub are size 7
% Also use eightmib and sevenmib for mib fonts
% 
   \def\mib{%
      \tenmibfonts
      \textfont0=\tenbf\scriptfont0=\eightbf
      \scriptscriptfont0=\sevenbf
      \textfont1=\tenmib\scriptfont1=\eighti
      \scriptscriptfont1=\seveni
      \textfont2=\tenbsy\scriptfont2=\eightsy
      \scriptscriptfont2=\sevensy}%
   \def\scr{\scrfonts
      \global\textfont\scrfam=\tenscr\fam\scrfam\tenscr}%
   \tt\ttglue=.5emplus.25emminus.15em
   \normalbaselineskip=12pt
   \setbox\strutbox=\hbox{\vrule height 8.5pt depth 3.5pt width 0pt}%
   \normalbaselines\rm\singlespaced
% Extra fonts -- bfit and itbf are bold faced italic; emphfont can be changed
   \let\emphfont=\it
   \let\bfit=\tenbxti
   \let\itbf=\tenbxti
% Define a super boldface with italic bold, math bold
   \let\boldface=\boldtenpoint
% Define the strut
\def\setstrut{\strutskip = \baselineskip}\setstrut%
\def\strut{\vrule height 0.7\strutskip depth 0.3\strutskip width 0pt}%
      }%

%       Switch to 11 point type
% 
% REDEFINE 11 point so that subscripts are size 8 and subsub are size 7
% 
\message{11pt,}
\def\elevenpoint{\elevenfonts           % load 11pt fonts if needed
   \def\rm{\fam0\elevenrm}%
   \textfont0=\elevenrm\scriptfont0=\eightrm\scriptscriptfont0=\sevenrm
   \textfont1=\eleveni\scriptfont1=\eighti\scriptscriptfont1=\seveni
   \textfont2=\elevensy\scriptfont2=\eightsy\scriptscriptfont2=\sevensy
   \textfont3=\elevenex\scriptfont3=\elevenex\scriptscriptfont3=\elevenex
   \textfont\itfam=\elevenit\def\it{\fam\itfam\elevenit}%
   \textfont\slfam=\elevensl\def\sl{\fam\slfam\elevensl}%
   \textfont\ttfam=\eleventt\def\tt{\fam\ttfam\eleventt}%
   \textfont\bffam=\elevenbf
   \scriptfont\bffam=\eightbf
   \scriptscriptfont\bffam=\sevenbf\def\bf{\fam\bffam\elevenbf}%
   \def\mib{%
      \elevenmibfonts
      \textfont0=\elevenbf\scriptfont0=\eightbf
      \scriptscriptfont0=\sevenbf
      \textfont1=\elevenmib\scriptfont1=\eightmib
      \scriptscriptfont1=\sevenmib
      \textfont2=\elevenbsy\scriptfont2=\eightsy
      \scriptscriptfont2=\sevensy}%
   \def\scr{\scrfonts
      \global\textfont\scrfam=\elevenscr\fam\scrfam\elevenscr}%
   \tt\ttglue=.5emplus.25emminus.15em
   \normalbaselineskip=13pt
   \setbox\strutbox=\hbox{\vrule height 9pt depth 4pt width 0pt}%
% Extra fonts -- bfit and itbf are bold faced italic; emphfont can be changed
   \let\emphfont=\it
   \let\bfit=\elevenbxti
   \let\itbf=\elevenbxti
% Define the strut
\def\setstrut{\strutskip = \baselineskip}\setstrut%
\def\strut{\vrule height 0.7\strutskip depth 0.3\strutskip width 0pt}%
% Define a super boldface with italic bold, math bold
   \let\boldface=\boldelevenpoint
   \normalbaselines\rm\singlespaced}%

%       Switch to 12 point type
\message{12pt,}
\def\twelvepoint{\twelvefonts\ninefonts % load 12pt and 9pt fonts if needed
   \def\rm{\fam0\twelverm}%
   \textfont0=\twelverm\scriptfont0=\ninerm\scriptscriptfont0=\sevenrm
   \textfont1=\twelvei\scriptfont1=\ninei\scriptscriptfont1=\seveni
   \textfont2=\twelvesy\scriptfont2=\ninesy\scriptscriptfont2=\sevensy
   \textfont3=\twelveex\scriptfont3=\twelveex\scriptscriptfont3=\twelveex
   \textfont\itfam=\twelveit\def\it{\fam\itfam\twelveit}%
   \textfont\slfam=\twelvesl\def\sl{\fam\slfam\twelvesl}%
   \textfont\ttfam=\twelvett\def\tt{\fam\ttfam\twelvett}%
   \textfont\bffam=\twelvebf
   \scriptfont\bffam=\ninebf
   \scriptscriptfont\bffam=\sevenbf\def\bf{\fam\bffam\twelvebf}%
   \def\mib{%
      \twelvemibfonts\tenmibfonts
      \textfont0=\twelvebf\scriptfont0=\ninebf
      \scriptscriptfont0=\sevenbf
% USE mib fonts for sub and subsub
      \textfont1=\twelvemib\scriptfont1=\ninemib
      \scriptscriptfont1=\sevenmib
      \textfont2=\twelvebsy\scriptfont2=\ninesy
      \scriptscriptfont2=\sevensy}%
   \def\scr{\scrfonts
      \global\textfont\scrfam=\twelvescr\fam\scrfam\twelvescr}%
   \tt\ttglue=.5emplus.25emminus.15em
   \normalbaselineskip=14pt
   \setbox\strutbox=\hbox{\vrule height 10pt depth 4pt width 0pt}%
   \let\emphfont=\it
% Extra fonts -- bfit and itbf are bold faced italic; emphfont can be changed
   \let\bfit=\twelvebxti
   \let\itbf=\twelvebxti
% Define the strut
\def\setstrut{\strutskip = \baselineskip}\setstrut%
\def\strut{\vrule height 0.7\strutskip depth 0.3\strutskip width 0pt}%
% Define a super boldface with italic bold, math bold
   \let\boldface=\boldtwelvepoint
   \normalbaselines\rm\singlespaced}%

%------------------------------------------------------------------------
%
%LOADING FONTS NOT LOADED IN plain.tex or in TEXSIS
%USED in 10pt and 12pt for RPP
	\font\sevenex=cmex10 scaled 667
	\font\eightex=cmex10 scaled 800
	\font\eightss cmss8
	\font\sevenss cmss10 scaled 666
	\font\eightbf cmbx8
	\font\eighti cmmi8
	\font\eightit cmti8
	\font\sevenit cmti7
	\font\eightrm cmr8
	\font\eightsy cmsy8
   \skewchar\eightsy='60%%
	\font\eightmib cmmib8
   \skewchar\eightmib='177%%
	\font\sevenmib cmmib7
   \skewchar\sevenmib='177%%
	\font\ninemib cmmib9
   \skewchar\ninemib='177%%
	\font\tenbxti cmbxti10
	\font\twelvebxti cmbxti10 scaled 1200
\font\Bigrm cmr17 scaled \magstep1
%
\def\extrasportsfonts{%
	\font\eightbsy cmbsy10 scaled 800
	\font\eightssbf cmssbx10 scaled 800
	\font\eightssi cmssi8
	\font\niness cmss9
	\font\msxmten=msam10 %  obscure mathematical symbols
	\font\tenssbf cmssbx10
	\font\elevenssbf cmssbx10 scaled 1095
	\font\twelvessbf cmssbx10 scaled 1200
\font\Bigbf cmbx12 scaled 1440
\font\Bigti cmti12 scaled \magstep2
\let\boldhelv \tenssbf
\let\helv \tenss
\def\interheadbf{\tenbf\bf\mib}
\let\hf\boldhead
\let\headfont\boldhead
\gdef\extrasportsfonts{\relax}}%
%
\def\extraminifonts{%
   \font\msxmtwelve=msam10  scaled 1200 %  obscure mathematical symbols
\gdef\extraminifonts{\relax}}%
%
%------------------------------------------------------------------------
%------------------------------------------------------------------------
% Starting defaults:

\gdef\Tbf{\twelvepoint\bf\mib}
\gdef\tbf{\tenpoint\bf\mib}
%
%DEFINING BOLDTENPOINT 
%
\def\boldtenpoint{\tenpoint\bf\mib%
   \textfont\itfam=\tenbxti\def\it{\fam\itfam\tenbxti}%
\relax}
\def\boldelevenpoint{\elevenpoint\bf\mib%
   \textfont\itfam=\elevenbxti\def\it{\fam\itfam\elevenbxti}%
\relax}
\def\boldtwelvepoint{\twelvepoint\bf\mib%
   \textfont\itfam=\twelvebxti\def\it{\fam\itfam\twelvebxti}%
\relax}
%
%
\def\etal{\hbox{\it et~al.}}
\tenpoint\rm

%       \enumalpha sets up an outline starting from lowercase alpha
% numbers numbers rather than Roman caps.
     
\def\enumalpha{%  enumerate a list in outline form, numbers first
   \def\setenumlead{\def\enumlead{}}%           % no leading part of a label.
   \def\enumcur{\ifcase\enumDepth               % for a given level choose...
      \or\letterN{\the\enumcnt}%                % 1) LC letter
      \or{\XA\romannumeral\number\enumcnt}%     % 2) LC roman numeral
      \or{\XA\number\enumcnt}%                  % 3) arabic number
      \else $\bullet$\space\fi}%                % or a bullet
   }
\def\enumnumoutline{\enumalpha}                 % backward compatability (2.13)

%       \enumroman sets up an outline starting from lowercase roman
% numbers numbers rather than Roman caps.
     
\def\enumroman{%  enumerate a list in outline form, numbers first
   \def\setenumlead{\def\enumlead{}}%           % no leading part of a label.
   \def\enumcur{\ifcase\enumDepth               % for a given level choose...
      \or{\XA\romannumeral\number\enumcnt}%     % 1) LC roman numeral
      \or\letterN{\the\enumcnt}%                % 2) LC letter
      \or{\XA\number\enumcnt}%                  % 3) arabic number
      \else $\bullet$\space\fi}%                % or a bullet
   }
\def\enumnumoutline{\enumroman}                 % backward compatability (2.13)
%%%%%%%%%%%%%%%%%%%%%%%%%%%%%%%%%%%%%%%%%%%%%%%%%%%%%%%%%%%%%%%
%%%%%%%%%%%%%%%%%%%%%%%%%%%%%%%%%%%%%%%%%%%%%%%%%%%%%%%%%%%%%%%
%%%%%%%%%%%%%%%%%%%%%%%%%%%%%%%%%%%%%%%%%%%%%%%%%%%%%%%%%%%%%%%
%%%%SAVE AND FIX%This patch allows breaking of references to balance columns
%%%%SAVE AND FIX%USAGE:  \EjectHeretrue before the `\reference' of the reference
%%%%SAVE AND FIX%USAGE that will appear at the end of the first column
%%%%SAVE AND FIX%USAGE:  \EjectHerefalse after the `\endreference' of the same reference
%%%%SAVE AND FIX%
%%%%SAVE AND FIX%
%%%%SAVE AND FIX\ATunlock       % make @ a letter for "hidden" macro names
%%%%SAVE AND FIX\newif\ifEjectHere \EjectHerefalse
%%%%SAVE AND FIX\ifEjectHere
%%%%SAVE AND FIX\message{HERE I AM}
%%%%SAVE AND FIX\fi
%%%%SAVE AND FIX%  \@GetRefText gets the reference text as everything up to the next
%%%%SAVE AND FIX%  \endreference and writes it to the .ref file.  #1 is the label.
%%%%SAVE AND FIX
%%%%SAVE AND FIX\def\@GetRefText#1{%                    % get and write reference text
%%%%SAVE AND FIX  \ifnullname                           % is name null (set by \@tagref)
%%%%SAVE AND FIX    \p@nctwrite;\relax                  % y: write ";" and line break
%%%%SAVE AND FIX    \begingroup\quoteoff                % turn off special "
%%%%SAVE AND FIX      \@refwrite{\@comment Reference text for %%
%%%%SAVE AND FIX      "#1" defined on page \number\pageno.}% 
%%%%SAVE AND FIX    \endgroup                           % let " be whatever it was
%%%%SAVE AND FIX  \else                                 % n: write "." and number
%%%%SAVE AND FIX    \ifnum\refnum>1                     % first reference?
%%%%SAVE AND FIX      \p@nctwrite.\fi                   % no: write trailing .
%%%%SAVE AND FIX\ifEjectHere
%%%%SAVE AND FIX\@refwrite{\string\vfill\string\eject}
%%%%SAVE AND FIX\fi
%%%%SAVE AND FIX    \begingroup\quoteoff%\@parensoff     % turn off special ",),(
%%%%SAVE AND FIX      \@refwrite{\@comment }%           %
%%%%SAVE AND FIX      \@refwrite{\@comment (\the\refnum)}% write comment to .ref file
%%%%SAVE AND FIX      \@refwrite{\@comment Reference text for %%
%%%%SAVE AND FIX      "#1" defined on page \number\pageno.}%  
%%%%SAVE AND FIX    \endgroup                           % let " be whatever it was
%%%%SAVE AND FIX    \@refwrite{\NX\@refitem{\the\refnum}}%  write ref number to .ref
%%%%SAVE AND FIX  \fi                                   %
%%%%SAVE AND FIX%                                       % Write reference to .ref file
%%%%SAVE AND FIX  \begingroup                           % start group to write text
%%%%SAVE AND FIX   \def\endreference{\NX\endreference}% % def these to nothing
%%%%SAVE AND FIX   \def\reference{\NX\reference}%       %   so we can play with them
%%%%SAVE AND FIX   \def\ref{\NX\ref}%                   %   without using them
%%%%SAVE AND FIX   \obeylines                           % so ^M seen by \@copyref
%%%%SAVE AND FIX   \@copyref                            % copy text line to .ref file
%%%%SAVE AND FIX}%                                
%%%%SAVE AND FIX
%%%%SAVE AND FIX\ATlock         % make @ not a letter (hides internal macros).
%%%%%%%%%%%%%%%%%%%%%%%%%%%%%%%%%%%%%%%%%%%%%%%%%%%%%%%%%%%%%%
%%%%%%%%%%%%%%%%%%%%%%%%%%%%%%%%%%%%%%%%%%%%%%%%%%%%%%%%%%%%%%
%       sportspatch
%
%%%%%%%%%%%%%%%%%%%%%%%%%%%%%%%%%%%%%%%%%%%%%%%%%%%%%%%%%%%%%%%%%%%
%sportsfont.tex
%%%%%%%%%%%%%%%%%%%%%%%%%%%%%%%%%%%%%%%%%%%%%%%%%%%%%%%%%%%%%%%%%%%
%This set of macros are the major revisions to the TEXSIS package
%Which are applicable only to the Sports Section.
%They include modifications to \chapter, \section, \caption, \figure,
%\table and equations which effect the numbering (no section numbers).
%Macros are also included which change the name of the aux files
%dependent on whether RPP or the Databook is being produced.
%Some new fonts are included, and a font definition for bold headers.
%
%%%%%%%%%%%%%%%%%%%%%%%%%%%%%%%%%%%%%%%%%%%%%%%%%%%%%%%%%%%%%%%%%%%
%
\font\cmsq=cmssq8 scaled 1200
\extrasportsfonts
%
%
{\twelvepoint\mib\relax}
{\tenpoint\mib\relax}
\rm
%defining bold fonts for major headings -- bold greek etc.  (BOLD 12)
\def\boldhead{%
  \fourteenpoint%
   \bf\mib
   }
\def\boldheaddb{%
  \twelvepoint%
%  \elevenpoint%
   \bf\mib
   }
%
%
\rm
%%%%%%%%%%%%%%%%%%%%%%%%%%%%%%%%%%%%%%%%%%%%%%%%%%%%%%%%%%%%%%%%%%%
\newbox\colonbox
\setbox\colonbox=\hbox{:}
%sportspatch.tex
%% TXSsects.tex                                 TeXsis version 2.15
\catcode`@=11
%-------------------------------------*
% CHAPTER LEVEL: \chapter{<title>} causes a page break, prints a
% chapter title, and resets important counters.  The macro \label may be
% used to assign the chapter number to a name if it occurs in the title.

\def\chapter#1{%        begin a new chapter in a document
  \global\advance\chapternum by \@ne            % increment chapt. #
  \global\sectionnum=\z@                        % reset section # to zero
  \global\def\@sectID{}%                        % section ID is null
  \global\def\S@sectID{}%                       % Ssection ID is null
%
%  Chapter ID:
%
  \edef\lab@l{\ChapterStyle{\the\chapternum}}%  % chapter ID for \label
  \ifshowchaptID                                % show chapter id?
    \global\edef\@chaptID{\lab@l.}%             % yes: define chapter ID
    \r@set                                      %    and reset most counters
  \else\edef\@chaptID{}\fi                      % else chapt ID is null
  \everychapter                                 % user can customize here
%
%  Print Title:
%
%
%  Running Headline and Table of Contents:
%
  \begingroup                                   % add to Table of Contents
    \def\label##1{}%                            % disable \label
    \xdef\ChapterTitle{#1}%                     % save title for user
    \def\n{}\def\nl{}\def\mib{}%                % now turn off \n, etc
    \setHeadline{#1}%                           % make title the running head
    \emsg{Chapter \@chaptID\space #1}%          % announce in log file
    \def\@quote{\string\@quote\relax}%          % in case of \quoteon
    \addTOC{0}{\NX\TOCcID{\lab@l.}#1}{\folio}%  % add to Table of Contents
  \endgroup                                     %
%
  \@Mark{#1}%                                   % page mark chapter heading
  \s@ction                                      % end of section processing
  \afterchapter}                                % more user customization


\def\everychapter{\relax}%    user ``hook'' performed before every \chapter
\def\afterchapter{\relax}%    user ``hook'' performed after every \chapter


%       \ChapterStyle lets you change the style in which the chapter number
% is displayed (i.e. \romannumerals, \Romannumerals, or \arabic

\def\ChapterStyle#1{#1}                         % default is nothing

\def\setChapterID#1{\edef\@chaptID{#1.}}        % so you can change \@chaptID


% \r@set restest all the counters for a new chapter or section

\def\r@set{%                    % resets counters for \theorem, \EQN, etc...
  \global\subsectionnum=\z@                     % subsection number = 0
  \global\subsubsectionnum=\z@                  % subsubsection number = 0
  \ifx\eqnum\undefined\relax                    % is TXSeqns.tex loaded?
    \else\global\eqnum=\z@\fi                   % yes: reset equation number
  \ifx\theoremnum\undefined\relax               % is TXSfigs.tex loaded?
  \else                                         % no: ignore its counters
    \global\theoremnum=\z@                      % yes: reset counters
    \global\lemmanum=\z@                        %
    \global\corollarynum=\z@                    %
    \global\definitionnum=\z@                   %
    \global\fignum=\z@                          %
    \ifRomanTables\relax                        %
    \else\global\tabnum=\z@\fi                  % do not reset if Roman
  \fi}

\long\def\s@ction{%  generic end processing common to chapters and sections
  \checkquote                                   % check for balanced quotes
  \checkenv                                     % check for balanced envmnt
  \nobreak\smallbreak                           % skip down a bit
  \vskip 0pt}                                   % force vertical mode

% TeXsperts note: the \@Mark \@noMark construction above is required because
% we are using a \mark which contains an \else, and the marking operation
% appears in conditional text (the \ifnum\chapternum ... \fi). -EAM

\def\@Mark#1{%  put a \mark with conditional text on the page
   \begingroup
     \def\label##1{}%                           % disable \label
     \def\goodbreak{}%                          % disable \goodbreak
     \def\mib{}\def\n{}%                        % disable \mib and \n
     \mark{#1\NX\else\lab@l}%                     % put mark on page
   \endgroup}%

\def\@noMark#1{\relax}%                         % do nothing

%-------------------------*
% \setHeadline{text} sets the running headline to the text, but
% if the text is so long that the user breaks it with \n then
% only the first part up to the \n is used, followed by ...
% (Move this to TXShead.tex at some point...)

\def\setHeadline#1{\@setHeadline#1\n\endlist}   % set the \HeadText

\def\@setHeadline#1\n#2\endlist{% #1 is everything up to first \n
   \def\@arg{#2}\ifx\@arg\empty                 % Is there \n in text?
      \global\edef\HeadText{#1}%                % No: just use #1
   \else                                        % there was, so...
      \global\edef\HeadText{#1\dots}%           %   use first part up to \n
   \fi
}




%--------------------------------------------------*
% SECTION LEVELS: \section{<title>} makes a section level break in
% text.  It does *not* skip to a new page, but it will reset everything
% and display the title in the \tbf typestyle (which defaults to \bf if
% undefined).  
\def\SectionFont{\twelvepoint\boldface}

%HERE
\def\section#1{%        create a new section of a document
   \vskip\sectionskip                           % make some space
   \goodbreak\pagecheck\sectionminspace         % new page if needed
   \global\advance\sectionnum by \@ne           % increment section counter
%
%  Section ID:
%
   \edef\lab@l{\@chaptID\SectionStyle{\the\sectionnum}}% For \label
   \ifshowsectID                                % show section number?
     \global\edef\@sectID{\SectionStyle{\the\sectionnum}.}% save for later
     \global\edef\@fullID{\lab@l.\space\space}% % what we will use here
  \global\subsectionnum=\z@                     % subsection number = 0
  \global\subsubsectionnum=\z@                  % subsubsection number = 0
\else\gdef\@fullID{}\fi                         % otherwise  section ID is empty
   \everysection                                % user customization here
%
%  Print the Section title:
%
   \ifx\tbf\undefined\def\tbf{\bf}\fi           % default \tbf is \bf
   \vbox{%                                      % keep heading in \vbox
         {\raggedright\SectionFont     % Ragged \bf title
     \setbox0=\hbox{\noindent\SectionFont\@fullID}%% find width of ID 
     \hangindent=\wd0 \hangafter=1              % ... for hanging indent 
 \noindent\@fullID                              %  which has section ID
     {#1}}}\relax                               % Print title ragged in \tbf
%zzz   \nobreak\smallbreak                         % skip down some below title
%
%  Table of Contents and Running Headlines:
%
   \begingroup                                  % group for \contents, etc.
     \def\label##1{}%                           % disable \label
     \global\edef\SectionTitle{#1}%             % or nothing
     \def\n{}\def\nl{}\def\mib{}%               % turn off \n, etc
     \ifnum\chapternum=0\setHeadline{#1}\fi     % no chapt. no. -> set headine
     \emsg{Section \@fullID #1}%                % announce in log file
     \def\@quote{\string\@quote\relax}%         % in case of \quoteon
     \addTOC{1}{\NX\TOCsID{\lab@l.}#1}{\folio}% % Table of Contents entry
   \endgroup                                    % end group
   \s@ction                                     % checkenv, etc..
   \aftersection}                               % user can customize


\def\everysection{\relax}%  user ``hook'' performed before every \section
\def\aftersection{\relax}%  user ``hook'' performed after every \section


\def\setSectionID#1{\edef\@sectID{#1.}}         % so you can set it


% \SectionStyle is style to display section number

\def\SectionStyle#1{#1}                         % default is nothing


%  \pagecheck<dimen> will skip to a new page if there is less than 
% the <dimen> of space on the remainder of the page.  For example,
% \pagecheck{0.333333\vsize} skips to a new page if there is less than
% 1/3 of the page left  

\def\pagecheck#1{% skip to new page if less than #1 left on this one
   \dimen@=\pagegoal                            % get total page size
   \advance\dimen@ by -\pagetotal               % get page space remaining
   \ifdim\dimen@>0pt                            % is there some left, but
   \ifdim\dimen@< #1\relax                      % not enough to look good?
      \vfil\break \fi\fi}                       % then skip to new page

%HERE
%HERE
\def\Resetsection{%        create a new section of a document
   \global\advance\sectionnum by \@ne           % increment section counter
%
%  Section ID:
%
   \global\edef\lab@l{\@chaptID\SectionStyle{\the\sectionnum}}% For \label
   \ifshowsectID                                % show section number?
     \global\edef\@sectID{\SectionStyle{\the\sectionnum}.}% save for later
     \global\edef\@fullID{\lab@l.\space\space}% % what we will use here
  \global\subsectionnum=\z@                     % subsection number = 0
  \global\subsubsectionnum=\z@                  % subsubsection number = 0
\else\gdef\@fullID{}\fi                         % otherwise  section ID is empty
   \everysection                                % user customization here
%
   \aftersection}                               % user can customize





%HERE

%--------------------------------------------------*
% \subsection makes a subsection

\def\printsubsectionstyle{\tenpoint\boldface}
\def\subsection#1{%                     % create a subsection of a document
 \ifnum\subsectionnum=0
  \par
   \else
   \vskip\subsectionskip                        % make some space
   \fi
   \goodbreak\pagecheck\sectionminspace         % new page if needed
   \global\advance\subsectionnum by \@ne        % increment counter
   \subsubsectionnum=\z@                        % reset subsubsection
%
%  Subsection ID:
%
   \edef\lab@l{\@chaptID\@sectID\SubsectionStyle{\the\subsectionnum}}%
   \ifshowsectID                                % show section number?
     \global\edef\@fullID{\lab@l.\space\space}% % yes: define it
   \else\gdef\@fullID{}\fi                      % otherwise it's empty
   \everysubsection                             % user can customize
%
   \begingroup                                  % group for \contents, etc.
     \def\label##1{}%                           % disable \label
     \global\edef\SubsectionTitle{#1}%          % or nothing
     \def\n{}\def\nl{}\def\mib{}%               % disable \n, etc
     \emsg{\@fullID #1}%                   % announce in log file
     \def\@quote{\string\@quote\relax}%         % in case of \quoteon
     \addTOC{2}{\NX\TOCsID{\lab@l.}#1}{\folio}% % Table of Contents entry
   \endgroup                                    % end \contents group
   \s@ction                                     % end of section 
%  Print the subsection title
%
%zzz   \vbox{%                                      % heading in \vbox
     {\raggedright\tbf                          % Ragged \tbf title
     \setbox0=\hbox{\noindent\printsubsectionstyle\@fullID}% % find width...
     \hangindent=\wd0 \hangafter=1              % ... for hanging indent
     \noindent\printsubsectionstyle\@fullID                          % show subsection number
	\printsubsectionstyle\bfit                 % tenpoint bold italic
     {#1}\hbox{\copy\colonbox}\relax}%zzz}%                  % print the title
\nobreak
%
   \aftersubsection\nobreak}                            % room to customize


\def\everysubsection{\relax}%  user ``hook'' performed before every \subsection
\def\aftersubsection{\relax}%  user ``hook'' performed after every \subsection
\def\SubsectionStyle#1{#1}                      % how to display number


%--------------------------------------------------*
% \subsubsection makes a subsubsection, mainly for use if there
% are no chapters in the document.
\subsectionskip=\smallskipamount
\def\printsubsubsectionstyle{\tenpoint\it}

\def\subsubsection#1{%                  % create a subsubsection of a document
  \ifnum\subsectionnum=0   
  \par
   \else
   \vskip\subsectionskip                        % make some space
   \fi
   \goodbreak\pagecheck\sectionminspace         % new page if needed
   \global\advance\subsubsectionnum by \@ne     % increment counter
%
%  Sub-subsection ID:
%
   \edef\lab@l{\@chaptID\@sectID\SectionStyle{\the\subsectionnum}.%%
           \SectionStyle{\the\subsubsectionnum}}% for \label
   \ifshowsectID                                % show section number?
     \global\edef\@fullID{\lab@l.\space\space}% % yes: define it
   \else\gdef\@fullID{}\fi                      % else it's empty
   \everysubsubsection                          % user can customize here
%
%  Print the sub-subsection title
%
   \begingroup                                  % group for \contents, etc.
     \def\label##1{}%                           % disable \label
     \global\edef\SubsectionTitle{#1}%          % or nothing
     \def\n{}\def\nl{}\def\mib{}%               % turn off \n, etc
     \emsg{\@fullID #1}%                        % announce in log file
     \def\@quote{\string\@quote\relax}%         % in case of \quoteon
     \addTOC{3}{\NX\TOCsID{\lab@l.}#1}{\folio}% % Table of Contents entry
   \endgroup                                    % end group
   \s@ction                                     % end of section
%zzzz\vbox{%                                    % heading in \vbox
     {\raggedright\tbf                          % Ragged \tbf title
     \setbox0=\hbox{\noindent\printsubsectionstyle\@fullID}%  % find width
     \hangindent=\wd0 \hangafter=1              % hanging indent after 1st line
     \printsubsectionstyle\noindent\@fullID     % show subsection number
    \printsubsubsectionstyle
     #1\hbox{:}\relax}%zzz}%                    % print the title
%       \nobreak\smallbreak                     % skip down some below title
%
   \aftersubsection}                            % room to customize


\def\everysubsubsection{\relax}%      % user hook  before every \subsubsection
\def\aftersubsubsection{\relax}%      % user hook  after every \subsubsection
\def\SubsubsectionStyle#1{#1}   % how to display subsubsection number



%======================================================================*
% CAPTIONS.
%       A caption is specified by \Caption ...\endCaption, or by
% \caption{<text>}.  If \caption or \Caption is the first token after 
% \figure or \table then the caption appears at the top; otherwise it 
% appears at the bottom. Additional spacing can be added with any legal 
% \vskip.
     
\newbox\@capbox                                 % box to hold caption text
\newcount\@caplines                             % number of lines in caption
\def\CaptionName{}                              % default Caption Name
\def\@ID{}                                      % default \@ID is null
     
\def\caption#1{% create a caption for a figure or table
   \def\lab@l{\@ID}%                              % save \@ID for \label
   \global\setbox\@capbox=\vbox\bgroup          % start box for caption
    \def\@inCaption{T}%                         % flag we are in caption
    \normalbaselines                            % singlespacing
    \dimen@=20\parindent                        % if column is wide then
    \ifdim\colwidth>\dimen@\narrower\fi%   use \narrower caption
    \noindent{\bf \CaptionName~\@ID:\space}%    % label ``Figure~nnn''
    #1\relax                                    % caption text
    \vskip0pt                                   % end paragraph
    \global\@caplines=\prevgraf                 % get number of lines
   \egroup                                      % now caption is in \vbox
   \ifnum\@ne=\@caplines                        % if only one line then
    \global\setbox\@capbox=\vbox\bgroup         % reset caption box
             {\bf \CaptionName~\@ID:\space}%    % label ``Figure~nnn''
       #1\hfil\egroup                           % with LEFT fill glue
   \fi                                          %
   \def\@inCaption{F}%                          % no longer in caption
   \if N\@whereCap\def\@whereCap{B}\fi          % not top, so bottom
   \if T\@whereCap                              % if top caption then
     \centerline{\box\@capbox}%                 %   print it
     \vglue 3pt                                 %   with space below
   \fi                                          %
   }

\def\@inCaption{F}%     default
     
\long\def\Caption#1\endCaption{\caption{#1}}

\def\endCaption{\emsg{> \NX\endCaption called before \NX\Caption.}}
\def\endcaption{\emsg{> try using \NX\caption{ text... }}}

%======================================================================*
%======================================================================*
%======================================================================*
%
%
%  \EQNOparse <label>;;\endlist  parses <text> as an equation number
%  label. It separates out letters, if present. (eg. label;a for equation a,
%  but ";" does not appear in output.)  It tags the label and puts the equation
%  number in the display by calling \@EQNOdisplay.

\def\EQNOparse#1;#2;#3\endlist{% parse equation label, look for ``;''
  \if ?#3?\relax                                % if #3 is null, no ";" present
    \global\advance\eqnum by\@ne                % so advance count 
    \edef\tnum{\@chaptID\the\eqnum}%            % construct eqn number
    \Eqtag{#1}{\tnum}%                          % tag the eqn number
    \@EQNOdisplay{#1}%                           % display equation number
  \else\stripblanks #2\endlist                  % remove any blanks
    \edef\p@rt{\tok}%                           % and save in \p@rt
    \if a\p@rt\relax                            % if ";" present is it ";a"?
      \global\advance\eqnum by\@ne\fi           % yes: advance counter anyway
    \edef\tnum{\@chaptID\the\eqnum}%            % construct eqn number
    \Eqtag{#1}{\tnum}%                          % and tag equation #nn
    \edef\tnum{\@chaptID\the\eqnum\p@rt}        % also make eqn # w/ letter
    \Eqtag{#1;\p@rt}{\tnum}%                    %   and tag #nn;x
    \@EQNOdisplay{#1;#2}%                        % display equation number
  \fi                                           %
  \global\let\?=\tnum                           % \? is last eqn # (ala PHYZZX)
  \relax}%                                      % end of definition

%==================================================*
% PARSE LABELS WITHOUT SECTID
%
%  \LabelParsewo <text>;;\endlist parses <text> as an equation,
%  figure or table label.  It separates out letters, if present (eg. 
%  label;a for Figure 2a, but the ";" does not appear in output.)  
%  It increments the counter \@count, if appropriate, and builds an ID in
%  \@ID.  It also \tag's the label with the value of the ID,
%  preceeding the label with the prefix \@prefix to distinguish its type.

\def\LabelParsewo#1;#2;#3\endlist{% 
  \if ?#3?\relax                                % if #3 is null, no ";" present
    \global\advance\@count by\@ne               % so advance count 
    \xdef\@ID{\@chaptID\the\@count}%            % construct value for label
    \tag{\@prefix#1}{\@ID}%                     % tag the fig number
  \else                                         % if ; , look for letter
    \stripblanks #2\endlist                     % remove any blanks from letter
    \edef\p@rt{\tok}%                           % and save in \p@rt
    \if a\p@rt\relax                            % if ";" present is it ";a"?
      \global\advance\@count by\@ne\fi          % yes: advance counter anyway
    \xdef\@ID{\@chaptID\the\@count}%            % construct figure ID
    \tag{\@prefix#1}{\@ID}%                     % and tag figure #nn
    \xdef\@ID{\@chaptID\the\@count\p@rt}%       % also fig # w/ letter
    \tag{\@prefix#1;\p@rt}{\@ID}%               %   and tag #nn;x
  \fi                                           % end \if ?#3?
}                                               % end of \LabelParsewo

\def\@ID{}                                      % default \@ID is null
     

%%%%%%%%%%%%%%%%%%%%%%%%%%%%%%%%%%%%%%%%%%%%%
%
%REDEFINE FIGURE SO THAT SECTID is GONE     
% \@figure is the generic figure making macro. If \FigsLast is
% set it writes the figure material out to a file to be read
% in later.  Otherwise it puts the figure material into an \insert
% given by #1.  #2 is the label which follows \figure
     
\def\@figure#1#2{%      generic routine to make and label a figure
  \vskip 0pt                                    % force vertial mode and anchor
  \begingroup                                   % hide our calculations
   \let\@count=\fignum                          % counter is \fignum
   \def\@prefix{Fg.}%                           % prefix for tagging labels
   \if ?#2?\relax \def\@ID{}%                   % if no label \@ID is null
   \else\LabelParsewo #2;;\endlist\fi           % else parse label into \@ID
   \def\CaptionName{Figure}%                    % type of insert is ``Figure''
   \ifFigsLast                                  % Save Figures til end?
    \emsg{\CaptionName\space\@ID. {#2} [storing in \jobname.fg]}% .log it
    \@fgwrite{\@comment> \CaptionName\space\@ID.\space{#2}}% write figure number
    \@fgwrite{\NX\@FigureItem{\CaptionName}{\@ID}{\NX#1}}%       % to the figure file
    \newlinechar=`\^^M                          % <CR> breaks lines in file 
    \obeylines                                  % to see ^^M in text
    \let\@next=\@copyfig                        % go copy figure text to file
   \else                                        % save figure in .fg file
    #1\relax                                    % start figure insert
    \setbox\@capbox\vbox to 0pt{}%              % clears caption in \@capbox
    \def\@whereCap{N}%                          % no caption (yet)
    \emsg{\CaptionName\ \@ID.\ {#2}}%           % announce in .log
    \let\endfigure=\@endfigure                  % how to quit
    \let\endFigure=\@endfigure                  % how to quit
    \let\ENDFIGURE=\@endfigure                  % how to quit
    \let\@next=\@findcap                        % look for \caption or \Caption
   \fi
   \@next}

     

%REDEFINE TABLE SO THAT SECTID is gone     
%   \@table is the generic table making macro. It also calls \@findcap,
% which looks to see if a \Caption comes first and acts accordingly.
     
\def\@table#1#2{% generic processing of table
  \vskip 0pt                                    % force vertial mode and anchor
  \begingroup                                   %
   \def\CaptionName{Table}%                     % type of insert is ``Table''
   \def\@prefix{Tb.}%                           % prefix for tagging labels
   \let\@count=\tabnum                          % counter is \fignum
   \if ?#2?\relax \def\@ID{}%                   % if no label \@ID is null
   \else                                        %
     \ifRomanTables                             % if Roman number tables
      \global\advance\@count by\@ne             %   advance count
      \edef\@ID{\uppercase\expandafter          %   define \@ID as uppercase
         {\romannumeral\the\@count}}%           %   Roman numerals
      \tag{\@prefix#2}{\@ID}%                   %   and tag it
     \else                                      %
       \LabelParsewo #2;;\endlist\fi            % else parse label --> \@ID
   \fi                                          %
   \ifTabsLast                                  % Save Tables til end?
    \emsg{\CaptionName\space\@ID. {#2} [storing in \jobname.tb]}% .log it
    \@tbwrite{\@comment> \CaptionName\space\@ID.\space{#2}}% write table number
    \@tbwrite{\NX\@FigureItem{\CaptionName}{\@ID}{\NX#1}}% % to the table file
    \newlinechar=`\^^M                      % <CR> breaks lines in file output
    \obeylines                                  % to see ^^M
    \let\@next=\@copytab                        % go copy figure text to file
   \else                                        % save figure in .fg file
    #1\relax                                    % do the insert
    \setbox\@capbox\vbox to 0pt{}%              % clears caption in \@capbox
    \def\@whereCap{N}%                          % no caption (yet)
    \emsg{\CaptionName\ \@ID.\ {#2}}%           % announce in .log
    \let\endtable=\@endfigure                   % how to quit
    \let\endTable=\@endfigure                   % how to quit
    \let\ENDTABLE=\@endfigure                   % how to quit
    \let\@next=\@findcap                        % look for \caption or \Caption
   \fi                                          %
   \@next}                                      

%==================================================*
%==================================================*
% INITIALIZATION. \AUXinit is used to open jobname.aux for output to
% save these definitions for the next run.  It opens the file and then
% disables itself so it doesn't try to open the file again.
%%%%%%%%%%%%%%%%%%%%%%%%%%%%%%%%%%%%%%%%%%%%%%%%%%%%%%%%%%%%%%%%%
%%%%%%%%%%%%%%%%%%%%%%%%%%%%%%%%%%%%%%%%%%%%%%%%%%%%%%%%%%%%%%%%%
%%%%%%%%%%%%%%%%%%%%%%%%%%%%%%%%%%%%%%%%%%%%%%%%%%%%%%%%%%%%%%%%%
%%%%%%%%%%%%%%%%%%%%%%%%%%%%%%%%%%%%%%%%%%%%%%%%%%%%%%%%%%%%%%
%%%%%%%%%%%%%%%%%%%%%%%%%%%%%%%%%%%%%%%%%%%%%%%%%%%%%%%%%%%%%%%%%%%%
%%%%%%%%%%%%%%%%%%%%%%%%%%%%%%%%%%%
\def\beginRPPonly{\ifnum\BigBookOrDataBooklet=1 \relax} % FOR RPP SPORTS
\def\beginDBonly{\ifnum\BigBookOrDataBooklet=2 \relax} % FOR DATABOOK
\let\endDBonly\fi
\let\endRPPonly\fi
\def\rppordb{\ifnum\BigBookOrDataBooklet=1 rpp\else db\fi}
\ifnum\BigBookOrDataBooklet=1
\def\AUXinit{%  once only initialization of auxilliary file
  \ifauxswitch                                  % if auxswitch true:
    \immediate\openout\auxfileout=\jobname\rppordb.aux  % open new .aux file 4 output
  \else                                         % if auxswitch false: 
    \gdef\auxout##1##2{}%               % turn off \tag 's writing to .aux file
  \fi
  \gdef\AUXinit{\relax}}                        % disable \AUXinit
\else
\def\AUXinit{%  once only initialization of auxilliary file
  \ifauxswitch                                  % if auxswitch true:
    \immediate\openout\auxfileout=\jobname\rppordb.aux  % open new .aux file 4 output
  \else                                         % if auxswitch false: 
    \gdef\auxout##1##2{}%               % turn off \tag 's writing to .aux file
  \fi
  \gdef\AUXinit{\relax}}                        % disable \AUXinit
\fi

% -- \auxout{csname}{value} writes the definition to the \jobname.aux file

\def\auxout#1#2{\AUXinit                        % initialize .aux file 
   \immediate\write\auxfileout{%                % write to file
   \NX\expandafter\NX\gdef                      % \expandafter\gdef...
   \NX\csname #1\NX\endcsname{#2}}%             %    \csname{data}
   }

%  \ReadAUX looks for a file called jobname.aux  and if it exists reads
%  it in.  This file should have tag and label definitions from a
%  previous run.   

\ifnum\BigBookOrDataBooklet=1
\def\ReadAUX{%  reads in the auxilliary file from a previous run
   \openin\auxfilein=\jobname\rppordb.aux               % open old .aux file 4 input
   \ifeof\auxfilein\closein\auxfilein           % if EOF it's empty.
   \else\closein\auxfilein                      % else close it and... 
     \begingroup                                % fix up special characters...
      \unSpecial                                %
      \input \jobname\rppordb.aux \relax                 % ...and read in the file
     \endgroup                                  % back to special characters
   \fi}                                         % else ignore it
\else
\def\ReadAUX{%  reads in the auxilliary file from a previous run
   \openin\auxfilein=\jobname\rppordb.aux               % open old .aux file 4 input
   \ifeof\auxfilein\closein\auxfilein           % if EOF it's empty.
   \else\closein\auxfilein                      % else close it and... 
     \begingroup                                % fix up special characters...
      \unSpecial                                %
      \input \jobname\rppordb.aux \relax                 % ...and read in the file
     \endgroup                                  % back to special characters
   \fi}                                         % else ignore it
\fi
\ReadAUX

%==================================================*
%==================================================*
\catcode`@=12
%==================================================*
%
%----------------------------------------------------------------------*
%%%%%%%%%%%%%%%%%%%%%%%%%%%%%%%%%%%%%%%%%%%%%%%%%%%%%%%%%%%%%%
%       head.tex
 \global\font\elevenbf=cmbx10 scaled \magstephalf
%%%%%%%%%%%%%%%%%%%%%%%%%%%%%%%%%%%%%%%%%%%%%%%%%%%%%%%%%%%%%%%%%
%HEADFIRST is the FIRST HALF OF THE TITLE
%HEADSECOND is the SECOND HALF OF THE TITLE (for multititles)
%HEADhbox temporary box used in headlines, final box
%HEADvbox temporary box used in headlines, input to HEADhbox
%RUNHEADhbox is the running head title, contains Chapter number, title
%onemorechapter is the number of the next chapter (for 2 chapter running heads)
%titlelinewidth is the width of the title line
%%%%%%%%%%%%%%%%%%%%%%%%%%%%%%%%%%%%%%%%%%%%%%%%%%%%%%%%%%%%%%%%%
\newbox\HEADFIRST%                         create new box
\newbox\HEADSECOND%                        create new box
\newbox\HEADhbox%                          create new box
\newbox\HEADvbox%                          create new box
\newbox\RUNHEADhbox%                          create new box
\newtoks\RUNHEADtok%                          create new box
\newcount\onemorechapter
\newdimen\titlelinewidth
\newdimen\movehead
\movehead=0pt
\titlelinewidth=.5pt
%%%%%%%%%%%%%%%%%%%%%%%%%%%%%%%%%%%%%%%%%%%%%%%%%%%%%%%%%%%%%%%%%
%For the running head
%The running head font is bold italic twelvepoint
	\def\runningheadfont{\twelvepoint\boldface\bfit}
%We can eliminate the running head (Default)
	\def\norunninghead{\setbox\RUNHEADhbox\hbox{\hss}}
	\norunninghead
%We can eliminate the chapter number in the  running head
	\def\nochapternumberrunninghead#1%
	{\setbox\RUNHEADhbox\hbox{\runningheadfont %
	#1}
        \WWWhead{\string\wwwtitle{#1}}%
}
%Normal sports section running head
	\def\runninghead#1{\setbox\RUNHEADhbox\hbox{\runningheadfont%
	\the\chapternum.~#1}%
        \WWWhead{\string\wwwtitle{#1}}%
         \RUNHEADtok={#1}}
%
%Sometimes we want a running head to cover 2 chapters
	\def\doublerunninghead#1#2{%
	\onemorechapter=\chapternum\relax
	\advance\onemorechapter by 1\relax
	\setbox\RUNHEADhbox\hbox{\runningheadfont%
	\the\chapternum.~#1, \the\onemorechapter.~#2}}
%%%%%%%%%%%%%%%%%%%%%%%%%%%%%
%Default section number for the sports section is III
%%%%%%%%%%%%%%%%%%%%%%%%%%%%%%%%%%%%%%%%%%%%%%%%%
%We define what we want as the title -- normal is chapter number, title
%We also label the chapter as the job name, chapter number is increased by 1
\def\heading#1{\chapter{#1}\label{Chap.\jobname}%
\setbox\HEADFIRST=\hbox{\boldhead\the\chapternum.~#1}
\printtheheading}
\def\smallerheading#1{\chapter{#1}\label{Chap.\jobname}%
\setbox\HEADFIRST=\hbox{\elevenbf\the\chapternum.~#1}
\printtheheading}
\def\printtheheading{\relax}
%printtheheading is only used for the databooklet
\def\notitleheading#1{%
   	   \chapter{#1}\label{Chap.\jobname}%
	   \setbox\HEADFIRST=\hbox{\boldhead\the\chapternum.~#1}}
\def\doubleheading#1#2{\chapter{#1}\label{Chap.\jobname}%
	   \centerline{\boldhead\hfill\the\chapternum.~#1\hfill}\vskip .1in%
	   \centerline{\boldhead\hfill #2\hfill}\vskip .2in}
\def\nochapterdoubleheading#1#2{%
	   \centerline{\boldhead\hfill #1\hfill}\vskip .1in%
	   \centerline{\boldhead\hfill #2\hfill}\vskip .2in}
\def\nochapterdbheading#1{%
	   \centerline{\boldhead\hfill #1\hfill}\vskip .1in}%
%Sometimes we want to have just the chapter number and no title
%We do not bump up the chapter number in this case!
\def\nochapterheading#1{%
    \label{Chap.\jobname}%
     \setbox\HEADFIRST=\hbox{\boldhead\the\chapternum.~#1}
            }
%Sometimes we want to have just the title and no chapter number
\def\nochapternumberheading#1{%
    \label{Chap.\jobname}%
    \setbox\HEADFIRST=\hbox{\boldhead~#1}
            }
%and Sometimes we Just want to label the chapter number
\def\nochapterheadingnochapternumber{%
    \label{Chap.\jobname}%
    \setbox\HEADFIRST=\hbox{\hss}
            }
%Sometimes we have to split the title over two lines
\def\multiheading#1#2{%
    \chapter{#1}\label{Chap.\jobname}%
    \setbox\HEADFIRST=\hbox{\boldhead\the\chapternum.~#1}
            \setbox\HEADSECOND=\hbox{\boldhead #2}}
%%%%%%%%%%%%%%%%%%%%%%%%%%%%%%%%%%%%%%%%%%%%%%%%%
%And then we can define the headline
%We have 4 kinds of headlines depending on even/odd and whether
%or not it is the first page of the section
%NOTE THAT SOMETIMES WE DO NOT WANT TO USE THE CONSTRUCTION
%FOR FIRST PAGE HEADLINES -- e.g, in the TEXT
%IN THAT CASE, take away the \Firstpage=\pageno in xxx.tex
%or in xxx.body
%THEN ADVANCE vsize by .3in for ALL pages of xxx
\headline={\ifnum\pageno=\Firstpage\firstoneq\else\restofthem\fi}
\def\firstoneq{\ifodd\pageno\firstheadodd\else\firstheadeven\fi}
\def\restofthem{\ifodd\pageno\contheadodd\else\contheadeven\fi}
%%%%%%%%%%%%%%%%%%%%%%%%%%%%%%%%%%%%%%%%%%%%%%%%%
\def\firstheadeven{%
  %\message{THIS IS FIRSTPAGE}%
\setbox\HEADvbox=\vtop to 1.15in{%
   \vglue .2in%
   \hbox to \fullhsize{%
    \boldhead  {\elevenssbf\Folio}\quad\copy\RUNHEADhbox\hss}%
   \vskip .1in%
   \hrule depth 0pt height \titlelinewidth
   \vskip .25in%
   \hbox to \fullhsize{\boldhead\hss\copy\HEADFIRST\hss}%
   \hbox to \fullhsize{\vrule height 18pt width 0pt%
           \boldhead\hss\copy\HEADSECOND\hss}%
   \vss%
             }%
    \setbox\HEADhbox=\hbox{\raise.85in\copy\HEADvbox}%
    \dp\HEADhbox=0pt\ht\HEADhbox=0pt\copy\HEADhbox%
     }
\def\firstheadodd{%
  \message{THIS IS FIRSTPAGE}%
  \setbox\HEADvbox=\vtop to 1.15in{%
   \vglue .2in%
   \hbox to \fullhsize{%
    \hss\copy\RUNHEADhbox\boldhead\quad{\elevenssbf\Folio}}%
   \vskip .1in%
   \hrule depth 0pt height \titlelinewidth
   \vskip .25in%
   \hbox to \fullhsize{\boldhead\hss\copy\HEADFIRST\hss}%
   \hbox to \fullhsize{\vrule height 18pt width 0pt%
           \boldhead\hss\copy\HEADSECOND\hss}%
   \vss%
             }%
    \setbox\HEADhbox=\hbox{\raise.85in\copy\HEADvbox}%
    \dp\HEADhbox=0pt\ht\HEADhbox=0pt\copy\HEADhbox%
     }
\def\contheadeven{%
  \setbox\HEADvbox=\vtop to .85in{%
   \vglue .2in%
   \hbox to \fullhsize{%
    \boldhead  {\elevenssbf\Folio}\quad\copy\RUNHEADhbox\hss}%
   \vskip .1in%
   \hrule depth 0pt height \titlelinewidth
   \vss%
             }%
    \setbox\HEADhbox=\hbox{\raise.55in\copy\HEADvbox}%
    \dp\HEADhbox=0pt\ht\HEADhbox=0pt\copy\HEADhbox%
     }
\def\contheadodd{%
  \setbox\HEADvbox=\vtop to .85in{%
   \vglue .2in%
   \hbox to \fullhsize{%
    \hss\copy\RUNHEADhbox\boldhead\quad{\elevenssbf\Folio}}%
   \vskip .1in%
   \hrule depth 0pt height \titlelinewidth
   \vss%
             }%
    \setbox\HEADhbox=\hbox{\raise.55in\copy\HEADvbox}%
    \dp\HEADhbox=0pt\ht\HEADhbox=0pt\copy\HEADhbox%
     }
%SOMETIMES WE JUST WANT TO PRINT THE PAGE NUMBER
\def\pagenumberonly{\setbox\RUNHEADhbox\hbox{\hss}%
	\setbox\HEADFIRST\hbox{\hss}%
	\titlelinewidth=0pt}
%HERE
%%%%%%%%%%%%%%%%%%%%%%%%%%%%%%%%%%%%%%%%%%%%%%%%%%%%%%%%%%%%%%
%%%%%%%%%%%%%%%%%%%%%%%%%%%%%%%%%%%%%%%%%%%%%%%%%%%%%%%%%%%%%%
\input rotate
\let\RMPpageno=\pageno
\def\scaleit#1#2{\rotdimen=\ht#1\advance\rotdimen by \dp#1%
    \hbox to \rotdimen{\hskip\ht#1\vbox to \wd#1{\rotstart{#2 #2 scale}%
    \box#1\vss}\hss}\rotfinish}
%rppmac.tex
%%%%%%%%%%%%%%%%%%%%%%%%%%%%%%%%%%%%%%%%%%%%%%%%%%%%%%%%%%%%%%%%%
%These are macros that are used in the Sports Section and
%not in the minireviews.  They are more general macros
%and are not usually dependent on TEXSIS.
%A new definition of item is presented.  \item is defined herein so
%so that items are indented by \itemindent rather than \parindent.
%A definition for \poormanbold is given.
%Table macros such as \centertab, \dashalign, \tablerule are given.
%Macros for adding figures in postscript are given.
%Macros for use in equations, new symbols are given.
%Macros for making boxes (\boxit) and figures (\figbox) and inserting
%same (\figinsert), (\insertpsfigure) are given.
%Macros for \refline, \refstar, etc. are given
%Macros for making indices are given.
%Macros for redefining Greek Characters are given.
%Macros for Bleeder Tabs and Crop Marks are given.
%Macros for two-column format, including headers, are given.
%Miscelaneous definits such as \Folio, \blackbox, \fn, \nocropmarks,
%\RPPonly, \DBonly are given.
%%%%%%%%%%%%%%%%%%%%%%%%%%%%%%%%%%%%%%%%%%%%%%%%%%%%%%%%%%%%%%%%%
%%%%%%%%%%%%%%%%%%%%%%%%%%%%%%%%%%%%%%%%%%%%%%%%%%%%%%%%%%%%%%
%%%%%%%%%%%%%%%%%%%%%%%%%%%%%%%%%%%%%%%%%%%%%%%%%%%%%%%%%%%%%%
%%%%%%%%%%%%%%%%%%%%%%%%%%%%%%%%%%%%%%%%%%%%%%%%%%%%%%%%%%%%%%
\def\WhoDidIt#1{%
	\noindent #1\smallskip}       
%%%%%%%%%%%%%%%%%%%%%%%%%%%%%%%%%%%%%%%%%%%%%%%%%%%%%%%%%%%%%%
\hyphenation{%
    brems-strah-lung
    Dan-ko-wych
    Fuku-gi-ta
    Gav-il-let
    Gla-show
    mono-pole
    mono-poles
    Sad-ler
}
% NOTE the following will redefine
% \hang which is used in the definition of \item
% so that \item is indented by itemindent rather than parindent
\let\HANG=\hang
\let\Item=\item
\let\Itemitem=\itemitem
\newdimen\itemindent
\itemindent=20pt
\def\hang{\hangindent\parindent}
\def\hang{\hangindent\itemindent}
\def\item{\par\hang\textindent}
\def\textindent#1{\indent\llap{#1\enspace}\ignorespaces}
\def\textindent#1{\bgroup\parindent=\itemindent\indent%
	\llap{#1\enspace}\egroup\ignorespaces}
\def\itemitem{\par\bgroup\parindent=\itemindent\indent\egroup
      \hangindent2\itemindent \textindent}
\EnvLeftskip=\itemindent
\EnvRightskip=0pt
%%%%%%%%%%%%%%%%%%%%%%%%%%%%%%%%%%%%%%%%%%%%%%%%%%%%%%%%%%%%%%
%%%%%%%%%%%%%%%%%%%%%%%%%%%%%%%%%%%%%%%%%%%%%%%%%%%%%%%%%%%%%%%%%%%%
\long\def\poormanbold#1%
% The purpose of this macro is to typeset text in poor man's bold.
% #1  The text to be bolded.
{%
    \leavevmode\hbox%
    {%
        \hbox to  0pt{#1\hss}\raise.3pt%
        \hbox to .3pt{#1\hss}%
        \hbox to  0pt{#1\hss}\raise.3pt%
        \hbox        {#1\hss}%
        \hss%
    }%
}
%
%
%
\def\columnbreaknopar{{\parfillskip=0pt\par}\vfill\eject\noindent\ignorespaces}
\def\columnbreakpar{\vfill\eject\ignorespaces}
%
%
%%%%%%%%%%%%%%%%%%%%%%%%%%%%%%%%%%%%%%%%%%%%%%%%%%%%%%%%%%%%%%%%%%%%
%
\long\def\XsecFigures#1#2#3#4#5#6#7%
{%
    \ifnum\IncludeXsecFigures = 0 %
        \vfill%
        \Page#1%
        \centerline{\figbox{\twelvepoint\bf #2 FIGURE}{7.75in}{4.8in}}%
        \vfill%
        \Page#4%
        \centerline{\figbox{\twelvepoint\bf #5 FIGURE}{7.75in}{4.8in}}%
        \vfill%
    \else%
        \vbox%
        {%
            \Page#1%
            \hbox to \hsize%
            {%
                \vtop to 4in%
                {%
                    \hsize = 0in%
                    \special%
                    {%
                        insert rpp$figures:#3.ps,%
                        top=13.4in,left=0.0in,%
                        magnification=1300,%
                        string="/translate{pop pop}def"%
                    }%
                    \vss%
                }%
                \hss%
            }%
            \vglue.5in%
            \Page#4%
            \hbox to \hsize%
            {%
                \vtop to 6in%
                {%
                    \hsize = 0in%
                    \special%
                    {%
                        insert rpp$figures:#6.ps,%
                        top=13.4in,left=0.0in,%
                        magnification=1300,%
                        string="/translate{pop pop}def"%
                    }%
                    \vss%
                }%
                \hss%
            }%
            \vss%
        }%
        \fi%
    \vfill%
    #7%
    \lastpagenumber%
}
%%%%%%%%%%%%%%%%%%%%%%%%%%%%%%%%%%%%%%%%%%%%%%%%%%%%%%%%%%%%%%%%%
%refmac.tex
%
%REFINEMENTS TO REFERENCES
%
\gdef\refline{\parskip=2pt\medskip\hrule width 2in\vskip 3pt%
	\tolerance=10000\pretolerance=10000}
\gdef\refstar{\parindent=20pt\vskip2pt\item{\hss$^\ast}}
\gdef\databookrefstar{{^\ast}}
\gdef\refdstar{\parindent=20pt\vskip2pt\item{\hss$^{\ast\ast}$}}
\gdef\refdagger{\parindent=20pt\vskip2pt\item{\hss$^\dagger$}}
\gdef\refstar{\parindent=20pt\vskip2pt\item{\hss$^\ast$}}
\gdef\refddagger{\parindent=20pt\vskip2pt\item{\hss$^\ddagger$}}
\gdef\refdstar{\parindent=20pt\vskip2pt\item{\hss$^{\ast\ast}$}}
\gdef\refdagger{\parindent=20pt\vskip2pt\item{\hss$^\dagger$}}
\gdef\refddagger{\parindent=20pt\vskip2pt\item{\hss$^\ddagger$}}
%%%%%%%%%%%%%%%%%%%%%%%%%%%%%%%%%%%%%%%%%%%%%%%%%%%%%%%%%%%%%%%%%
\def\refFormat{%redefine refskip format for the list of references
\catcode`\^^M=10           
\def\refskip{\vskip0pt plus 2pt}% % no space but some stretch
           \refindent=20pt% Reset refindent from 2em
           \leftskip=20pt% Reset leftskip from 2em
            }
\def\Refskip{\vskip0pt plus 2pt}% % no space but some stretch
%%%%%%%%%%%%%%%%%%%%%%%%%%%%%%%%%%%%%%%%%%%%%%%%%%%%%%%%%%%%%%%%%
%tabmac.tex
%TABLE MACROS -- FOR BETTY
%
\def\tableheaddoublerule{\noalign{\vglue2pt\hrule\vskip3pt\hrule\smallskip}}
\def\tableheadsinglerule{\noalign{\medskip\hrule\smallskip}}
\def\tablefootdoublerule{\noalign{\vskip 3pt\hrule\vskip3pt\hrule\smallskip}}
\def\tablerule{\noalign{\hrule}}
\def\thicktablerule{\noalign{\hrule height .95pt}}
\def\Tablerule{\noalign{\vskip 2pt}\noalign{\hrule}\noalign{\vskip 2pt}}
\def\crule{\cr\tablerule}
%\def\BEGINtable{\par\vskip 3pt\bgroup}
%\def\ENDtable{\egroup\par\vskip 3pt}
%\def\begintable{\par\vskip 3pt\bgroup}
%\def\endtable{\egroup\par\vskip 3pt}
\def\pmalign#1#2{$\llap{$#1$}\pm\rlap{$#2$}$}
\def\dashalign#1#2{\llap{#1}\hbox{--}\rlap{#2}}
\def\decalign#1#2{\llap{#1}.\rlap{#2}}
\let\da=\decalign
\def\ph{\phantom{1}\relax}
\def\phh{\phantom{11}\relax}
\def\centertab#1{\hfil#1\hfil}
\def\righttab#1{\hfil#1}
\def\lefttab#1{#1\hfil}
\let\ct=\centertab
\let\rt=\righttab
\let\lt=\lefttab
\def\TSTRUT{\vrule height 12pt depth 5pt width 0pt}
\def\shortstrut{\vrule height 10pt depth 4pt width 0pt}
\def\tstrut{\vrule height 10pt depth 4pt width 0pt}
\def\Tstrut{\vrule height 11pt depth 4pt width 0pt}
\def\TStrut{\vrule height 13pt depth 5pt width 0pt}
\def\sstrut{\vrule height 11.25pt depth 2.5pt width 0pt}
\def\nostrut{\vrule height 0pt depth 0pt width 0pt}
\def\omitstrut{\omit\vrule height 0pt depth 4pt width 0pt}
\def\afterlboxstrut{%
   \noalign{\vskip-4pt}\omit\vrule height 0pt depth 4pt width 0pt}
\def\highstrut{\vrule height 13pt depth 4pt width 0pt}
\def\deepstrut{\vrule height 10pt depth 6pt width 0pt}
\def\endstrut{\vrule height 0pt depth 6pt width 0pt}
\def\topstrut{\vrule height 4pt depth 6pt width 0pt}
%%%%%%%%%%%%%%%%%%%%%%%%%%%%%%%%%%%%%%%%%%%%%%%%%%%%%%%%%%%%%%
%psmac.tex
%%%%%%psmacros -- to add figures in postscript to dvi files
%
%% Conversion from small-points to points.
\def\sptopt{65536}
%% Lengths for offsets and scalings
\newdimen\PSOutputWidth
\newdimen\PSInputWidth
\newdimen\PSOffsetX
\newdimen\PSOffsetY
%% Draws a fiducial at (0,0)
\def\PSOrigin{%
    0 0 moveto 10 0 lineto stroke 0 0 moveto 0 10 lineto stroke}
%% Scales PostScript output when used in PS file postheader
\def\PSScale{%
    \number\PSOutputWidth \space \number\PSInputWidth \space div \space 
    dup scale}% implements PostScript scaling
%% Offsets PostScript output when used in PS file postheader
\def\PSOffset{%
    \number\PSOffsetX \space \sptopt \space div %
    \number\PSOffsetY \space \sptopt \space div \space translate}
%% Apply both PostScript transformations
\def\PSTransform{%
    \PSScale \space \PSOffset}
%% General transform to undo UGS-PostScript transformations
\def\UGSTransform{%
    \PSScale \space \PSOffset \space 27 600 translate -90 rotate}
%%
%% Specific output formatting
%%
%% Position UGS-PostScript landscape output with full 11" reduced to
%% column of width \hsize.
%%
\def\UGSLandInput#1{%
    \PSOffsetX=0in                 % no offset
    \PSOffsetY=0in                 % no offset
    \PSOutputWidth=\hsize      % fit in text column
    \PSInputWidth=11in             % width of a full landscape page
    \special{#1 \PSOrigin \space \UGSTransform}}
%%
%\def\UGSmod#1{%
%    \PSOffsetX=-0.52in       % Actual offset is this divided by 0.77
%    \PSOffsetY=-1.22in
%    \PSOutputWidth=\hsize      % fit in text column
%    \PSInputWidth=7.78in           % width of a full landscape page
%    \special{#1 \space \UGSTransform}}
%%
%% Position UGS-PostScript landscape output with HPLOT 8 1/8" box reduced to
%% column of width \hsize.
%%
\def\UGSInput#1{\PSOutputWidth=\hsize%
    \PSOffsetX=-125pt              % X offset to HPLOT box
    \PSOffsetY=0in                 % no offset
    \PSInputWidth=8.125in          % width of HPLOT box
    \special{#1 \PSOrigin \space \UGSTransform}}
%%
%% Position Adobe Illustrator output (portrait mode) with 8 1/2" page reduced
%% to column of width \hsize.
%%
\def\AIInput#1{\PSOutputWidth=\hsize%
    \PSOffsetX= 0in                % 0in = no offset 
    \PSOffsetY= 0in                % 0in = no offset
    \PSInputWidth=8.5in            % width of page
    \special{#1 \PSOrigin \space \PSTransform}}
%END of psmacros
%%%%%%%%%%%%%%%%%%%%%%%%%%%%%%%%%%%%%%%%%%%%%%%%%%%%%%%%%%%%%%%%%
%eqmac.tex
%
%
\def\mbox#1{{\ifmmode#1\else$#1$\fi}}
%
\def\widebar{\overline}
\def\widevec{\overrightarrow}
%
% The following definitions are for creating bigger math accents
% (vector, bar, tilde, hat) AND putting them in a vbox of 0 height
% So you don't get extra vertical space in the text
% Example \Widehat x
%
\def\Widevec#1{\vbox to 0pt{\vss\hbox{$\overrightarrow #1$}}}
\def\Widebar#1{\vbox to 0pt{\vss\hbox{$\overline #1$}}}
\def\Widetilde#1{\vbox to 0pt{\vss\hbox{$\widetilde #1$}}}
\def\Widehat#1{\vbox to 0pt{\vss\hbox{$\widehat #1$}}}
%
\def\ttbar{\mbox{t\Widebar t}}
\def\uubar{\mbox{u\Widebar u}}
\def\ccbar{\mbox{c\Widebar c}}
\def\qqbar{\mbox{q\Widebar q}}
\def\ggbar{\mbox{g\Widebar g}}
\def\ssbar{\mbox{s\Widebar s}}
\def\ppbar{\mbox{p\Widebar p}}
\def\ddbar{\mbox{d\Widebar d}}
\def\bbbar{\mbox{b\Widebar b}}
\def\Kbar{\mbox{\Widebar K}}
\def\Dbar{\mbox{\Widebar D}}
\def\Bbar{\mbox{\Widebar B}}
\def\Abar{\mbox{\Widebar A}}
\def\barA{\mbox{\Widebar A}}
\def\xbar{\mbox{\Widebar x}}
\def\zbar{\mbox{\Widebar z}}
\def\fbar{\mbox{\Widebar f}}
\def\ebar{\mbox{\Widebar e}}
\def\cbar{\mbox{\Widebar c}}
\def\tbar{\mbox{\Widebar t}}
\def\sbar{\mbox{\Widebar s}}
\def\ubar{\mbox{\Widebar u}}
\def\rbar{\mbox{\Widebar r}}
\def\dbar{\mbox{\Widebar d}}
\def\bbar{\mbox{\Widebar b}}
\def\qbar{\mbox{\Widebar q}}
\def\gbar{\mbox{\Widebar g}}
\def\barg{\mbox{\Widebar g}}
\def\abar{\mbox{\Widebar a}}
\def\pvec{\mbox{\Widevec p}}
\def\rvec{\mbox{\Widevec r}}
\def\Jvec{\mbox{\Widevec J}}
\def\pbar{\mbox{\Widebar p}}
\def\nbar{\mbox{\Widebar n}}
\def\alphahat{\widehat\alpha}
\def\Lambdabar{\mbox{\Widebar \Lambda}}
\def\Omegabar{\mbox{\Widebar \Omega}}
\def\mubar{\mbox{\Widebar \mu}}
\def\betavec{\mbox{{\Widevec \beta}}}
\def\Xibar{\mbox{\Widebar \Xi}}
\def\xibar{\mbox{\Widebar \xi}}
\def\Gammabar{\mbox{\Widebar \Gamma}}
\def\thetabar{\mbox{\Widebar \theta}}
\def\nubar{\mbox{\Widebar \nu}}
\def\ellbar{\mbox{\Widebar \ell}}
\def\alphabar{\mbox{\Widebar \alpha}}
\def\ellbar{\mbox{\Widebar \ell}}
\def\psibar{\mbox{\Widebar \psi}}
%
%%%%%%%%%%%%%%%%%%%%%%%%%%%%%%%%%%%%%%%%%%%%%%%%%%%%%%%%%%%%%%
%
%FOR FRACTIONS:
%
\def\frac#1#2{{\displaystyle{#1 \over #2}}}% USEAGE \frac{A}{B}
\def\textfrac#1#2{{\textstyle{#1 \over #2^{\ftstrut}}}}% USEAGE \frac{A}{B}
\def\ftstrut{\vrule height 5pt depth 0pt width 0pt}
%
%
%%%%%%%%%%%%%%%%%%%%%%%%%%%%%%%%%%%%%%%%%%%%%%%%%%%%%%%%%%%%%%
%UNITS
%
\def\GeV{\ifmmode{\hbox{ GeV }}\else{GeV}\fi}
\def\MeV{\ifmmode{\hbox{ MeV }}\else{MeV}\fi}
\def\keV{\ifmmode{\hbox{ keV }}\else{keV}\fi}
\def\eV{\ifmmode{\hbox{ eV }}\else{eV}\fi}
\def\GV{\ifmmode{{\rm GeV}/c}\else{GeV/$c$}\fi}
\def\invTV{\ifmmode{({\rm TeV}/c)^{-1}}\else{(TeV/$c)^{-1}$}\fi}
\def\TV{\ifmmode{{\rm TeV}/c}\else{TeV/$c$}\fi}
%
\def\cm{{\rm cm}}         %math mode centimeters.
\def\mum{\ifmmode{\mu{\rm m}}\else{$\mu$m}\fi}
\def\mus{\ifmmode{\mu{\rm s}}\else{$\mu$s}\fi}
%
\def\lum{\ifmmode{{\rm cm}^{-2}{\rm s}^{-1}}%
   \else{cm$^{-2}$s$^{-1}$}\fi}%
\def\lstd{\ifmmode{10^{33}\,{\rm cm}^{-2}{\rm s}^{-1}}%
   \else{$10^{33}\,$cm$^{-2}$s$^{-1}$}\fi}%
\def\hilstd{\ifmmode{10^{34}\,{\rm cm}^{-2}{\rm s}^{-1}}%
   \else{$10^{34}\,$cm$^{-2}$s$^{-1}$}\fi}%
%
%
%%%%%%%%%%%%%%%%%%%%%%%%%%%%%%%%%%%%%%%%%%%%%%%%%%%%%%%%%%%%%%
% MATHEMATICAL SYMBOLS
%
% The following indespensible macro (Vacuum Expectation Value) is from phyzzx.
\def\VEV#1{\left\langle #1\right\rangle}
%
\def\trademark{\,\hbox{\msxmten\char'162}\,}
\def\bigcircle{\,\hbox{\tensy\char'015}\,}
\def\gsim{\,\hbox{\msxmten\char'046}\,}
\def\lsim{\,\hbox{\msxmten\char'056}\,}
\def\simg{\,\hbox{\msxmten\char'046}\,}
\def\siml{\,\hbox{\msxmten\char'056}\,}
%
\def\ttt#1{\times 10^{#1}}
%
\def\mone{$^{-1}$}
\def\mtwo{$^{-2}$}
\def\mthree{$^{-3}$}
%
\def\abseta{\ifmmode{|\eta|}\else{$|\eta|$}\fi}
\def\abs#1{\left| #1\right|}
%
\def\pperp{\ifmmode{p_\perp}\else{$p_\perp$}\fi}
%
\def\deg{\ifmmode{^\circ}\else{$^\circ$}\fi}%
%
\def\missEt{\ifmmode{/\mkern-11mu E_t}\else{${/\mkern-11mu E_t}$}\fi}
\def\missEt{\ifmmode{\hbox{missing-}E_t}\else{$\hbox{missing-}E_t$}\fi}
\let\etmiss\missEt
%
\def\elln{ \ln\,}
%
\def\to{\rightarrow}%  Restores \to to proper definition (phyzzx renamed it)
%
\def\star{\tenrm *}
\def\headstar{\Twelvebf *}
%
\def\comma{\ ,\ }% USED in offset equations
\def\semi{\ ;\ }% USED in offset equations
\def\period{\ .\ }% USED in offset equations
%
%%%%%%%%%%%%%%%%%%%%%%%%%%%%%%%%%%%%%%%%%%%%%%%%%%%%%%%%%%%%%%
%
%%%%%%%%%%%%%%%%%%%%%%%%%%%%%%%%%%%%%%%%%%%%%%%%%%%%%%%%%%%%%%
%boxfigmac.tex
%
\newdimen\Linewidth                \Linewidth=0.001in
\newdimen\boxsideindent                \boxsideindent=0.5in
\newdimen\halfboxsideindent                \halfboxsideindent=0.25in
\newdimen\boxheightindent                \boxheightindent=0.05in
\newdimen\figboxwidth                \figboxwidth=4.25in
\newdimen\figboxheight                \figboxheight=4.25in
%
%
\long\def\boxit#1#2#3%
% The purpose of this macro is to put a box around lines of text.
% #1  The number of the box (or a box register) containing the lines
%     of text around which a solid box is to be drawn.
% #2  The amount of white space, that should surround the text.
% #3  The thickness, in inches, of the rules used to draw the box.
% For example, \boxit{\ParName}{0.03in}{0.05in}
{%
    \vbox%
    {%
        \hrule height #3%
        \hbox%
        {%
            \vrule width #3%
            \vbox%
            {%
                \kern #2%
                \hbox%
                {%
                    \kern #2%
                    \vbox{\hsize=\wd#1\noindent\copy#1}%
                    \kern #2%
                }%
                \kern #2%
            }%
            \vrule width #3%
        }%
        \hrule height #3%
    }%
}
\def\boxA{%
\setbox0=\hbox{A}\boxit{0}{1pt}{.5pt}%
}
\def\boxB{%
\setbox0=\hbox{B}\boxit{0}{1pt}{.5pt}%
}
\def\boxplain{%
\setbox0=\hbox{\phantom{\vrule height .5em width .5em}}\boxit{0}{1pt}{.5pt}%
}
%
\def\squareA{\leavevmode\lower 2pt\hbox{\boxA}}
\def\squareB{\leavevmode\lower 2pt\hbox{\boxB}}
\def\plainsquare{\leavevmode\lower 2pt\hbox{\boxplain}}
%
\def\Boxit#1#2#3%
%Boxit is more generalized version of Gary's boxit in some respects, and
%a less general in others.  For example it expects just one line of text,
%not a whole paragraph.  But, it allows you to set the vertical and horizontal
%space that surround the text differently.
% The purpose of this macro is to put a box around 1 line of text.
% #1  The number of the box (or a box register) containing the lines
%     of text around which a solid box is to be drawn.
% #2  The amount of horizontal white space, that should surround the text.
% #3  The amount of vertical white space, that should surround the text.
%     The thickness, of the rules used to draw the box is set with \Linewidth.
% For example, \Boxit{\ParName}{0.03in}{0.05in}
{%
%    \vbox% HERE IS TRY
    \vtop% HERE IS TRY
    {%
        \hrule height \Linewidth%
        \hbox%
        {%
            \vrule width \Linewidth%
            \vbox%
            {%
                \kern #3% vertical space
                \hbox%
                {%
                    \kern #2% horizontal space
                    \vbox{\hbox to 0in{\hss\copy#1\hss}}%
                    \kern #2% horizontal space
                }%
                \kern #3%vertical space
            }%
            \vrule width \Linewidth%
        }%
        \hrule height \Linewidth%
    }%
}
%
%
\def\figbox#1#2#3%
{\setbox0=\hbox{#1}\dp0=0pt\ht0=0pt\figboxwidth=#2\relax\figboxheight=#3\relax%
\divide\figboxwidth by 2\relax%
\divide\figboxheight by 2\relax%
\Boxit{0}{\figboxwidth}{\figboxheight}
}%
%
\def\Figbox#1#2#3%
{%
\halfboxsideindent=\boxsideindent\divide\halfboxsideindent by 2\relax%
\hglue\halfboxsideindent%
\setbox0=\hbox{#1}\figboxwidth=#2\relax\figboxheight=#3\relax%
\divide\figboxwidth by 2\relax%
\divide\figboxheight by 2\relax%
\advance\figboxwidth by -\boxsideindent\relax%
\advance\figboxheight by -\boxheightindent\relax%
\Boxit{0}{\figboxwidth}{\figboxheight}}%
%
%figbox#1#2#3 is used to leave space for figures in the text and should
%be used between midinsert and endinsert
%What you get is a vbox with a label in the middle and a line drawn around
%it --   it uses Boxit
%#1 is the text that goes in the box
%#2 is horizontal dimension
%#3 is vertical dimension
%Example  \figbox{FIGURE J}{3.4in}{.2in}
%Example of side by side:
%   \hbox{\figbox{FIGURE H}{1.7in}{.2in}\figbox{FIGURE I}{1.7in}{.7in}}
%Example of beginning a paragraph
%    \leavevmode\hbox{\figbox{FIGURE J}{3.4in}{.2in}} Now is the time
%Example of putting a figbox inside a paragraph
%  Now is the first time \hbox{\figbox{FIGURE J}{3.4in}{.2in}} Now is the time
%The difference between Figbox and figbox is that Figbox indents the
%box .5in from the left-hand margin and right-handmargin -- This .5in can
%be changed by resetting \boxsideindent
%If \boxsideindent = 0pt and \boxheightindent=0pt then \figbox==\Figbox
%
%%%%%%%%%%%%%%%%%%%%%%%%%%%%%%%%%%%%%%%%%%%%%%%%%%%%%%%%%%%%%%%%%%%%%%%%%%%%%%%%%
%
%LOI FIGURES
%
%
% LABELED FIGURE CAPTION -- DON'S Style
\def\figcaption#1#2{%
\bgroup\Tenpoint\par\noindent\narrower FIG.~#1. #2 \smallskip\egroup}
%
%
%USAGE:\figcaption#1#2{\whatever}{This is the caption.}
%
\def\figinsert#1#2{%
\ifdraft{\vrule height #1 depth 0pt width 0.5pt}%
\vbox to 40pt{\hbox to 0pt{\qquad\qquad#2 \hss}\vss}%
\vbox to -40pt{\hbox to 0pt{\qquad\qquad\hss}\vss}%
\else{\vrule height #1 depth 0pt width 0pt}%
\noindent\AIInput{disk$physics00:[deg.loi.physfigs]#2.ps}\fi}
%
\def\figsize#1#2{%
\ifdraft{\vrule height #1 depth 0pt width 0.5pt}%
\vbox to 40pt{\hbox to 0pt{\qquad#2 \hss}\vss}%
\vbox to -40pt{\hbox to 0pt{\qquad\hss}\vss}%
\else{\vrule height #1 depth 0pt width 0pt}\fi}
%
%
%%%%%%%%%%%%%%%%%%%%%%%%%%%%%%%%%%%%%%%%%%%%%%%%%%
%% \input psfig
% Psfig/TeX 
\def\PsfigVersion{1.9}
% dvips version
%
% All psfig/tex software, documentation, and related files
% in this distribution of psfig/tex are 
% Copyright 1987, 1988, 1991 Trevor J. Darrell
%
% Permission is granted for use and non-profit distribution of psfig/tex 
% providing that this notice is clearly maintained. The right to
% distribute any portion of psfig/tex for profit or as part of any commercial
% product is specifically reserved for the author(s) of that portion.
%
% *** Feel free to make local modifications of psfig as you wish,
% *** but DO NOT post any changed or modified versions of ``psfig''
% *** directly to the net. Send them to me and I'll try to incorporate
% *** them into future versions. If you want to take the psfig code 
% *** and make a new program (subject to the copyright above), distribute it, 
% *** (and maintain it) that's fine, just don't call it psfig.
%
% Bugs and improvements to trevor@media.mit.edu.
%
% Thanks to Greg Hager (GDH) and Ned Batchelder for their contributions
% to the original version of this project.
%
% Modified by J. Daniel Smith on 9 October 1990 to accept the
% %%BoundingBox: comment with or without a space after the colon.  Stole
% file reading code from Tom Rokicki's EPSF.TEX file (see below).
%
% More modifications by J. Daniel Smith on 29 March 1991 to allow the
% the included PostScript figure to be rotated.  The amount of
% rotation is specified by the "angle=" parameter of the \psfig command.
%
% Modified by Robert Russell on June 25, 1991 to allow users to specify
% .ps filenames which don't yet exist, provided they explicitly provide
% boundingbox information via the \psfig command. Note: This will only work
% if the "file=" parameter follows all four "bb???=" parameters in the
% command. This is due to the order in which psfig interprets these params.
%
%  3 Jul 1991	JDS	check if file already read in once
%  4 Sep 1991	JDS	fixed incorrect computation of rotated
%			bounding box
% 25 Sep 1991	GVR	expanded synopsis of \psfig
% 14 Oct 1991	JDS	\fbox code from LaTeX so \psdraft works with TeX
%			changed \typeout to \ps@typeout
% 17 Oct 1991	JDS	added \psscalefirst and \psrotatefirst
%

% From: gvr@cs.brown.edu (George V. Reilly)
%
% \psdraft	draws an outline box, but doesn't include the figure
%		in the DVI file.  Useful for previewing.
%
% \psfull	includes the figure in the DVI file (default).
%
% \psscalefirst width= or height= specifies the size of the figure
% 		before rotation.
% \psrotatefirst (default) width= or height= specifies the size of the
% 		 figure after rotation.  Asymetric figures will
% 		 appear to shrink.
%
% \psfigurepath#1	sets the path to search for the figure
%
% \psfig
% usage: \psfig{file=, figure=, height=, width=,
%			bbllx=, bblly=, bburx=, bbury=,
%			rheight=, rwidth=, clip=, angle=, silent=}
%
%	"file" is the filename.  If no path name is specified and the
%		file is not found in the current directory,
%		it will be looked for in directory \psfigurepath.
%	"figure" is a synonym for "file".
%	By default, the width and height of the figure are taken from
%		the BoundingBox of the figure.
%	If "width" is specified, the figure is scaled so that it has
%		the specified width.  Its height changes proportionately.
%	If "height" is specified, the figure is scaled so that it has
%		the specified height.  Its width changes proportionately.
%	If both "width" and "height" are specified, the figure is scaled
%		anamorphically.
%	"bbllx", "bblly", "bburx", and "bbury" control the PostScript
%		BoundingBox.  If these four values are specified
%               *before* the "file" option, the PSFIG will not try to
%               open the PostScript file.
%	"rheight" and "rwidth" are the reserved height and width
%		of the figure, i.e., how big TeX actually thinks
%		the figure is.  They default to "width" and "height".
%	The "clip" option ensures that no portion of the figure will
%		appear outside its BoundingBox.  "clip=" is a switch and
%		takes no value, but the `=' must be present.
%	The "angle" option specifies the angle of rotation (degrees, ccw).
%	The "silent" option makes \psfig work silently.
%

% check to see if macros already loaded in (maybe some other file says
% "\input psfig") ...
\ifx\undefined\psfig\else\endinput\fi

%
% from a suggestion by eijkhout@csrd.uiuc.edu to allow
% loading as a style file. Changed to avoid problems
% with amstex per suggestion by jbence@math.ucla.edu

\let\LaTeXAtSign=\@
\let\@=\relax
\edef\psfigRestoreAt{\catcode`\@=\number\catcode`@\relax}
%\edef\psfigRestoreAt{\catcode`@=\number\catcode`@\relax}
\catcode`\@=11\relax
\newwrite\@unused
\def\ps@typeout#1{{\let\protect\string\immediate\write\@unused{#1}}}
\ps@typeout{psfig/tex \PsfigVersion}

%% Here's how you define your figure path.  Should be set up with null
%% default and a user useable definition.

\def\figurepath{./}
\def\psfigurepath#1{\edef\figurepath{#1}}

%
% @psdo control structure -- similar to Latex @for.
% I redefined these with different names so that psfig can
% be used with TeX as well as LaTeX, and so that it will not 
% be vunerable to future changes in LaTeX's internal
% control structure,
%
\def\@nnil{\@nil}
\def\@empty{}
\def\@psdonoop#1\@@#2#3{}
\def\@psdo#1:=#2\do#3{\edef\@psdotmp{#2}\ifx\@psdotmp\@empty \else
    \expandafter\@psdoloop#2,\@nil,\@nil\@@#1{#3}\fi}
\def\@psdoloop#1,#2,#3\@@#4#5{\def#4{#1}\ifx #4\@nnil \else
       #5\def#4{#2}\ifx #4\@nnil \else#5\@ipsdoloop #3\@@#4{#5}\fi\fi}
\def\@ipsdoloop#1,#2\@@#3#4{\def#3{#1}\ifx #3\@nnil 
       \let\@nextwhile=\@psdonoop \else
      #4\relax\let\@nextwhile=\@ipsdoloop\fi\@nextwhile#2\@@#3{#4}}
\def\@tpsdo#1:=#2\do#3{\xdef\@psdotmp{#2}\ifx\@psdotmp\@empty \else
    \@tpsdoloop#2\@nil\@nil\@@#1{#3}\fi}
\def\@tpsdoloop#1#2\@@#3#4{\def#3{#1}\ifx #3\@nnil 
       \let\@nextwhile=\@psdonoop \else
      #4\relax\let\@nextwhile=\@tpsdoloop\fi\@nextwhile#2\@@#3{#4}}
% 
% \fbox is defined in latex.tex; so if \fbox is undefined, assume that
% we are not in LaTeX.
% Perhaps this could be done better???
\ifx\undefined\fbox
% \fbox code from modified slightly from LaTeX
\newdimen\fboxrule
\newdimen\fboxsep
\newdimen\ps@tempdima
\newbox\ps@tempboxa
\fboxsep = 3pt
\fboxrule = .4pt
\long\def\fbox#1{\leavevmode\setbox\ps@tempboxa\hbox{#1}\ps@tempdima\fboxrule
    \advance\ps@tempdima \fboxsep \advance\ps@tempdima \dp\ps@tempboxa
   \hbox{\lower \ps@tempdima\hbox
  {\vbox{\hrule height \fboxrule
          \hbox{\vrule width \fboxrule \hskip\fboxsep
          \vbox{\vskip\fboxsep \box\ps@tempboxa\vskip\fboxsep}\hskip 
                 \fboxsep\vrule width \fboxrule}
                 \hrule height \fboxrule}}}}
\fi
%
%%%%%%%%%%%%%%%%%%%%%%%%%%%%%%%%%%%%%%%%%%%%%%%%%%%%%%%%%%%%%%%%%%%
% file reading stuff from epsf.tex
%   EPSF.TEX macro file:
%   Written by Tomas Rokicki of Radical Eye Software, 29 Mar 1989.
%   Revised by Don Knuth, 3 Jan 1990.
%   Revised by Tomas Rokicki to accept bounding boxes with no
%      space after the colon, 18 Jul 1990.
%   Portions modified/removed for use in PSFIG package by
%      J. Daniel Smith, 9 October 1990.
%
\newread\ps@stream
\newif\ifnot@eof       % continue looking for the bounding box?
\newif\if@noisy        % report what you're making?
\newif\if@atend        % %%BoundingBox: has (at end) specification
\newif\if@psfile       % does this look like a PostScript file?
%
% PostScript files should start with `%!'
%
{\catcode`\%=12\global\gdef\epsf@start{%!}}
\def\epsf@PS{PS}
%
\def\epsf@getbb#1{%
%
%   The first thing we need to do is to open the
%   PostScript file, if possible.
%
\openin\ps@stream=#1
\ifeof\ps@stream\ps@typeout{Error, File #1 not found}\else
%
%   Okay, we got it. Now we'll scan lines until we find one that doesn't
%   start with %. We're looking for the bounding box comment.
%
   {\not@eoftrue \chardef\other=12
    \def\do##1{\catcode`##1=\other}\dospecials \catcode`\ =10
    \loop
       \if@psfile
	  \read\ps@stream to \epsf@fileline
       \else{
	  \obeyspaces
          \read\ps@stream to \epsf@tmp\global\let\epsf@fileline\epsf@tmp}
       \fi
       \ifeof\ps@stream\not@eoffalse\else
%
%   Check the first line for `%!'.  Issue a warning message if its not
%   there, since the file might not be a PostScript file.
%
       \if@psfile\else
       \expandafter\epsf@test\epsf@fileline:. \\%
       \fi
%
%   We check to see if the first character is a % sign;
%   if so, we look further and stop only if the line begins with
%   `%%BoundingBox:' and the `(atend)' specification was not found.
%   That is, the only way to stop is when the end of file is reached,
%   or a `%%BoundingBox: llx lly urx ury' line is found.
%
          \expandafter\epsf@aux\epsf@fileline:. \\%
       \fi
   \ifnot@eof\repeat
   }\closein\ps@stream\fi}%
%
% This tests if the file we are reading looks like a PostScript file.
%
\long\def\epsf@test#1#2#3:#4\\{\def\epsf@testit{#1#2}
			\ifx\epsf@testit\epsf@start\else
\ps@typeout{Warning! File does not start with `\epsf@start'.  It may not be a PostScript file.}
			\fi
			\@psfiletrue} % don't test after 1st line
%
%   We still need to define the tricky \epsf@aux macro. This requires
%   a couple of magic constants for comparison purposes.
%
{\catcode`\%=12\global\let\epsf@percent=%\global\def\epsf@bblit{%BoundingBox}}
%
%
%   So we're ready to check for `%BoundingBox:' and to grab the
%   values if they are found.  We continue searching if `(at end)'
%   was found after the `%BoundingBox:'.
%
\long\def\epsf@aux#1#2:#3\\{\ifx#1\epsf@percent
   \def\epsf@testit{#2}\ifx\epsf@testit\epsf@bblit
	\@atendfalse
        \epsf@atend #3 . \\%
	\if@atend	
	   \if@verbose{
		\ps@typeout{psfig: found `(atend)'; continuing search}
	   }\fi
        \else
        \epsf@grab #3 . . . \\%
        \not@eoffalse
        \global\no@bbfalse
        \fi
   \fi\fi}%
%
%   Here we grab the values and stuff them in the appropriate definitions.
%
\def\epsf@grab #1 #2 #3 #4 #5\\{%
   \global\def\epsf@llx{#1}\ifx\epsf@llx\empty
      \epsf@grab #2 #3 #4 #5 .\\\else
   \global\def\epsf@lly{#2}%
   \global\def\epsf@urx{#3}\global\def\epsf@ury{#4}\fi}%
%
% Determine if the stuff following the %%BoundingBox is `(atend)'
% J. Daniel Smith.  Copied from \epsf@grab above.
%
\def\epsf@atendlit{(atend)} 
\def\epsf@atend #1 #2 #3\\{%
   \def\epsf@tmp{#1}\ifx\epsf@tmp\empty
      \epsf@atend #2 #3 .\\\else
   \ifx\epsf@tmp\epsf@atendlit\@atendtrue\fi\fi}


% End of file reading stuff from epsf.tex
%%%%%%%%%%%%%%%%%%%%%%%%%%%%%%%%%%%%%%%%%%%%%%%%%%%%%%%%%%%%%%%%%%%

%%%%%%%%%%%%%%%%%%%%%%%%%%%%%%%%%%%%%%%%%%%%%%%%%%%%%%%%%%%%%%%%%%%
% trigonometry stuff from "trig.tex"
\chardef\psletter = 11 % won't conflict with \begin{letter} now...
\chardef\other = 12

\newif \ifdebug %%% turn me on to see TeX hard at work ...
\newif\ifc@mpute %%% don't need to compute some values
\c@mputetrue % but assume that we do

\let\then = \relax
\def\r@dian{pt }
\let\r@dians = \r@dian
\let\dimensionless@nit = \r@dian
\let\dimensionless@nits = \dimensionless@nit
\def\internal@nit{sp }
\let\internal@nits = \internal@nit
\newif\ifstillc@nverging
\def \Mess@ge #1{\ifdebug \then \message {#1} \fi}

{ %%% Things that need abnormal catcodes %%%
	\catcode `\@ = \psletter
	\gdef \nodimen {\expandafter \n@dimen \the \dimen}
	\gdef \term #1 #2 #3%
	       {\edef \t@ {\the #1}%%% freeze parameter 1 (count, by value)
		\edef \t@@ {\expandafter \n@dimen \the #2\r@dian}%
				   %%% freeze parameter 2 (dimen, by value)
		\t@rm {\t@} {\t@@} {#3}%
	       }
	\gdef \t@rm #1 #2 #3%
	       {{%
		\count 0 = 0
		\dimen 0 = 1 \dimensionless@nit
		\dimen 2 = #2\relax
		\Mess@ge {Calculating term #1 of \nodimen 2}%
		\loop
		\ifnum	\count 0 < #1
		\then	\advance \count 0 by 1
			\Mess@ge {Iteration \the \count 0 \space}%
			\Multiply \dimen 0 by {\dimen 2}%
			\Mess@ge {After multiplication, term = \nodimen 0}%
			\Divide \dimen 0 by {\count 0}%
			\Mess@ge {After division, term = \nodimen 0}%
		\repeat
		\Mess@ge {Final value for term #1 of 
				\nodimen 2 \space is \nodimen 0}%
		\xdef \Term {#3 = \nodimen 0 \r@dians}%
		\aftergroup \Term
	       }}
	\catcode `\p = \other
	\catcode `\t = \other
	\gdef \n@dimen #1pt{#1} %%% throw away the ``pt''
}

\def \Divide #1by #2{\divide #1 by #2} %%% just a synonym

\def \Multiply #1by #2%%% allows division of a dimen by a dimen
       {{%%% should really freeze parameter 2 (dimen, passed by value)
	\count 0 = #1\relax
	\count 2 = #2\relax
	\count 4 = 65536
	\Mess@ge {Before scaling, count 0 = \the \count 0 \space and
			count 2 = \the \count 2}%
	\ifnum	\count 0 > 32767 %%% do our best to avoid overflow
	\then	\divide \count 0 by 4
		\divide \count 4 by 4
	\else	\ifnum	\count 0 < -32767
		\then	\divide \count 0 by 4
			\divide \count 4 by 4
		\else
		\fi
	\fi
	\ifnum	\count 2 > 32767 %%% while retaining reasonable accuracy
	\then	\divide \count 2 by 4
		\divide \count 4 by 4
	\else	\ifnum	\count 2 < -32767
		\then	\divide \count 2 by 4
			\divide \count 4 by 4
		\else
		\fi
	\fi
	\multiply \count 0 by \count 2
	\divide \count 0 by \count 4
	\xdef \product {#1 = \the \count 0 \internal@nits}%
	\aftergroup \product
       }}

\def\r@duce{\ifdim\dimen0 > 90\r@dian \then   % sin(x+90) = sin(180-x)
		\multiply\dimen0 by -1
		\advance\dimen0 by 180\r@dian
		\r@duce
	    \else \ifdim\dimen0 < -90\r@dian \then  % sin(-x) = sin(360+x)
		\advance\dimen0 by 360\r@dian
		\r@duce
		\fi
	    \fi}

\def\Sine#1%
       {{%
	\dimen 0 = #1 \r@dian
	\r@duce
	\ifdim\dimen0 = -90\r@dian \then
	   \dimen4 = -1\r@dian
	   \c@mputefalse
	\fi
	\ifdim\dimen0 = 90\r@dian \then
	   \dimen4 = 1\r@dian
	   \c@mputefalse
	\fi
	\ifdim\dimen0 = 0\r@dian \then
	   \dimen4 = 0\r@dian
	   \c@mputefalse
	\fi
%
	\ifc@mpute \then
        	% convert degrees to radians
		\divide\dimen0 by 180
		\dimen0=3.141592654\dimen0
%
		\dimen 2 = 3.1415926535897963\r@dian %%% a well-known constant
		\divide\dimen 2 by 2 %%% we only deal with -pi/2 : pi/2
		\Mess@ge {Sin: calculating Sin of \nodimen 0}%
		\count 0 = 1 %%% see power-series expansion for sine
		\dimen 2 = 1 \r@dian %%% ditto
		\dimen 4 = 0 \r@dian %%% ditto
		\loop
			\ifnum	\dimen 2 = 0 %%% then we've done
			\then	\stillc@nvergingfalse 
			\else	\stillc@nvergingtrue
			\fi
			\ifstillc@nverging %%% then calculate next term
			\then	\term {\count 0} {\dimen 0} {\dimen 2}%
				\advance \count 0 by 2
				\count 2 = \count 0
				\divide \count 2 by 2
				\ifodd	\count 2 %%% signs alternate
				\then	\advance \dimen 4 by \dimen 2
				\else	\advance \dimen 4 by -\dimen 2
				\fi
		\repeat
	\fi		
			\xdef \sine {\nodimen 4}%
       }}

% Now the Cosine can be calculated easily by calling \Sine
\def\Cosine#1{\ifx\sine\UnDefined\edef\Savesine{\relax}\else
		             \edef\Savesine{\sine}\fi
	{\dimen0=#1\r@dian\advance\dimen0 by 90\r@dian
	 \Sine{\nodimen 0}
	 \xdef\cosine{\sine}
	 \xdef\sine{\Savesine}}}	      
% end of trig stuff
%%%%%%%%%%%%%%%%%%%%%%%%%%%%%%%%%%%%%%%%%%%%%%%%%%%%%%%%%%%%%%%%%%%%

\def\psdraft{
	\def\@psdraft{0}
	%\ps@typeout{draft level now is \@psdraft \space . }
}
\def\psfull{
	\def\@psdraft{100}
	%\ps@typeout{draft level now is \@psdraft \space . }
}

\psfull

\newif\if@scalefirst
\def\psscalefirst{\@scalefirsttrue}
\def\psrotatefirst{\@scalefirstfalse}
\psrotatefirst

\newif\if@draftbox
\def\psnodraftbox{
	\@draftboxfalse
}
\def\psdraftbox{
	\@draftboxtrue
}
\@draftboxtrue

\newif\if@prologfile
\newif\if@postlogfile
\def\pssilent{
	\@noisyfalse
}
\def\psnoisy{
	\@noisytrue
}
\psnoisy
%%% These are for the option list.
%%% A specification of the form a = b maps to calling \@p@@sa{b}
\newif\if@bbllx
\newif\if@bblly
\newif\if@bburx
\newif\if@bbury
\newif\if@height
\newif\if@width
\newif\if@rheight
\newif\if@rwidth
\newif\if@angle
\newif\if@clip
\newif\if@verbose
\def\@p@@sclip#1{\@cliptrue}


%\newif\if@decmpr
% Causes confusing error messages if the needed file is present
% but NOT in .Z format.  (LBL, M. Gelbaum, 5-18-1992)

%%% GDH 7/26/87 -- changed so that it first looks in the local directory,
%%% then in a specified global directory for the ps file.
%%% RPR 6/25/91 -- changed so that it defaults to user-supplied name if
%%% boundingbox info is specified, assuming graphic will be created by
%%% print time.
%%% TJD 10/19/91 -- added bbfile vs. file distinction, and @decmpr flag
%%% MG 5/18/92 -- Removed @decmpr flag (see comment above).

\def\@p@@sfigure#1{\def\@p@sfile{null}\def\@p@sbbfile{null}
	        \openin1=#1.bb
		\ifeof1\closein1
	        	\openin1=\figurepath#1.bb
			\ifeof1\closein1
			        \openin1=#1
				\ifeof1\closein1%
				       \openin1=\figurepath#1
					\ifeof1
					   \ps@typeout{Error, File #1 not found}
						\if@bbllx\if@bblly
				   		\if@bburx\if@bbury
			      				\def\@p@sfile{#1}%
			      				\def\@p@sbbfile{#1}%
							%\@decmprfalse
				  	   	\fi\fi\fi\fi
					\else\closein1
				    		\def\@p@sfile{\figurepath#1}%
				    		\def\@p@sbbfile{\figurepath#1}%
						%\@decmprfalse
	                       		\fi%
			 	\else\closein1%
					\def\@p@sfile{#1}
					\def\@p@sbbfile{#1}
					%\@decmprfalse
			 	\fi
			\else
				\def\@p@sfile{\figurepath#1}
				\def\@p@sbbfile{\figurepath#1.bb}
				%\@decmprtrue 
			\fi
		\else
			\def\@p@sfile{#1}
			\def\@p@sbbfile{#1.bb}
			%\@decmprtrue
		\fi}

\def\@p@@sfile#1{\@p@@sfigure{#1}}

\def\@p@@sbbllx#1{
		%\ps@typeout{bbllx is #1}
		\@bbllxtrue
		\dimen100=#1
		\edef\@p@sbbllx{\number\dimen100}
}
\def\@p@@sbblly#1{
		%\ps@typeout{bblly is #1}
		\@bbllytrue
		\dimen100=#1
		\edef\@p@sbblly{\number\dimen100}
}
\def\@p@@sbburx#1{
		%\ps@typeout{bburx is #1}
		\@bburxtrue
		\dimen100=#1
		\edef\@p@sbburx{\number\dimen100}
}
\def\@p@@sbbury#1{
		%\ps@typeout{bbury is #1}
		\@bburytrue
		\dimen100=#1
		\edef\@p@sbbury{\number\dimen100}
}
\def\@p@@sheight#1{
		\@heighttrue
		\dimen100=#1
   		\edef\@p@sheight{\number\dimen100}
		%\ps@typeout{Height is \@p@sheight}
}
\def\@p@@swidth#1{
		%\ps@typeout{Width is #1}
		\@widthtrue
		\dimen100=#1
		\edef\@p@swidth{\number\dimen100}
}
\def\@p@@srheight#1{
		%\ps@typeout{Reserved height is #1}
		\@rheighttrue
		\dimen100=#1
		\edef\@p@srheight{\number\dimen100}
}
\def\@p@@srwidth#1{
		%\ps@typeout{Reserved width is #1}
		\@rwidthtrue
		\dimen100=#1
		\edef\@p@srwidth{\number\dimen100}
}
\def\@p@@sangle#1{
		%\ps@typeout{Rotation is #1}
		\@angletrue
%		\dimen100=#1
		\edef\@p@sangle{#1} %\number\dimen100}
}
\def\@p@@ssilent#1{ 
		\@verbosefalse
}
\def\@p@@sprolog#1{\@prologfiletrue\def\@prologfileval{#1}}
\def\@p@@spostlog#1{\@postlogfiletrue\def\@postlogfileval{#1}}
\def\@cs@name#1{\csname #1\endcsname}
\def\@setparms#1=#2,{\@cs@name{@p@@s#1}{#2}}
%
% initialize the defaults (size the size of the figure)
%
\def\ps@init@parms{
		\@bbllxfalse \@bbllyfalse
		\@bburxfalse \@bburyfalse
		\@heightfalse \@widthfalse
		\@rheightfalse \@rwidthfalse
		\def\@p@sbbllx{}\def\@p@sbblly{}
		\def\@p@sbburx{}\def\@p@sbbury{}
		\def\@p@sheight{}\def\@p@swidth{}
		\def\@p@srheight{}\def\@p@srwidth{}
		\def\@p@sangle{0}
		\def\@p@sfile{} \def\@p@sbbfile{}
		\def\@p@scost{10}
		\def\@sc{}
		\@prologfilefalse
		\@postlogfilefalse
		\@clipfalse
		\if@noisy
			\@verbosetrue
		\else
			\@verbosefalse
		\fi
}
%
% Go through the options setting things up.
%
\def\parse@ps@parms#1{
	 	\@psdo\@psfiga:=#1\do
		   {\expandafter\@setparms\@psfiga,}}
%
% Compute bb height and width
%
\newif\ifno@bb
\def\bb@missing{
	\if@verbose{
		\ps@typeout{psfig: searching \@p@sbbfile \space  for bounding box}
	}\fi
	\no@bbtrue
	\epsf@getbb{\@p@sbbfile}
        \ifno@bb \else \bb@cull\epsf@llx\epsf@lly\epsf@urx\epsf@ury\fi
}	
\def\bb@cull#1#2#3#4{
	\dimen100=#1 bp\edef\@p@sbbllx{\number\dimen100}
	\dimen100=#2 bp\edef\@p@sbblly{\number\dimen100}
	\dimen100=#3 bp\edef\@p@sbburx{\number\dimen100}
	\dimen100=#4 bp\edef\@p@sbbury{\number\dimen100}
	\no@bbfalse
}
% rotate point (#1,#2) about (0,0).
% The sine and cosine of the angle are already stored in \sine and
% \cosine.  The result is placed in (\p@intvaluex, \p@intvaluey).
\newdimen\p@intvaluex
\newdimen\p@intvaluey
\def\rotate@#1#2{{\dimen0=#1 sp\dimen1=#2 sp
%            	calculate x' = x \cos\theta - y \sin\theta
		  \global\p@intvaluex=\cosine\dimen0
		  \dimen3=\sine\dimen1
		  \global\advance\p@intvaluex by -\dimen3
% 		calculate y' = x \sin\theta + y \cos\theta
		  \global\p@intvaluey=\sine\dimen0
		  \dimen3=\cosine\dimen1
		  \global\advance\p@intvaluey by \dimen3
		  }}
\def\compute@bb{
		\no@bbfalse
		\if@bbllx \else \no@bbtrue \fi
		\if@bblly \else \no@bbtrue \fi
		\if@bburx \else \no@bbtrue \fi
		\if@bbury \else \no@bbtrue \fi
		\ifno@bb \bb@missing \fi
		\ifno@bb \ps@typeout{FATAL ERROR: no bb supplied or found}
			\no-bb-error
		\fi
		%
%\ps@typeout{BB: \@p@sbbllx, \@p@sbblly, \@p@sbburx, \@p@sbbury} 
%
% store height/width of original (unrotated) bounding box
		\count203=\@p@sbburx
		\count204=\@p@sbbury
		\advance\count203 by -\@p@sbbllx
		\advance\count204 by -\@p@sbblly
		\edef\ps@bbw{\number\count203}
		\edef\ps@bbh{\number\count204}
		%\ps@typeout{ psbbh = \ps@bbh, psbbw = \ps@bbw }
		\if@angle 
			\Sine{\@p@sangle}\Cosine{\@p@sangle}
	        	{\dimen100=\maxdimen\xdef\r@p@sbbllx{\number\dimen100}
					    \xdef\r@p@sbblly{\number\dimen100}
			                    \xdef\r@p@sbburx{-\number\dimen100}
					    \xdef\r@p@sbbury{-\number\dimen100}}
%
% Need to rotate all four points and take the X-Y extremes of the new
% points as the new bounding box.
                        \def\minmaxtest{
			   \ifnum\number\p@intvaluex<\r@p@sbbllx
			      \xdef\r@p@sbbllx{\number\p@intvaluex}\fi
			   \ifnum\number\p@intvaluex>\r@p@sbburx
			      \xdef\r@p@sbburx{\number\p@intvaluex}\fi
			   \ifnum\number\p@intvaluey<\r@p@sbblly
			      \xdef\r@p@sbblly{\number\p@intvaluey}\fi
			   \ifnum\number\p@intvaluey>\r@p@sbbury
			      \xdef\r@p@sbbury{\number\p@intvaluey}\fi
			   }
%			lower left
			\rotate@{\@p@sbbllx}{\@p@sbblly}
			\minmaxtest
%			upper left
			\rotate@{\@p@sbbllx}{\@p@sbbury}
			\minmaxtest
%			lower right
			\rotate@{\@p@sbburx}{\@p@sbblly}
			\minmaxtest
%			upper right
			\rotate@{\@p@sbburx}{\@p@sbbury}
			\minmaxtest
			\edef\@p@sbbllx{\r@p@sbbllx}\edef\@p@sbblly{\r@p@sbblly}
			\edef\@p@sbburx{\r@p@sbburx}\edef\@p@sbbury{\r@p@sbbury}
%\ps@typeout{rotated BB: \r@p@sbbllx, \r@p@sbblly, \r@p@sbburx, \r@p@sbbury}
		\fi
		\count203=\@p@sbburx
		\count204=\@p@sbbury
		\advance\count203 by -\@p@sbbllx
		\advance\count204 by -\@p@sbblly
		\edef\@bbw{\number\count203}
		\edef\@bbh{\number\count204}
		%\ps@typeout{ bbh = \@bbh, bbw = \@bbw }
}
%
% \in@hundreds performs #1 * (#2 / #3) correct to the hundreds,
%	then leaves the result in @result
%
\def\in@hundreds#1#2#3{\count240=#2 \count241=#3
		     \count100=\count240	% 100 is first digit #2/#3
		     \divide\count100 by \count241
		     \count101=\count100
		     \multiply\count101 by \count241
		     \advance\count240 by -\count101
		     \multiply\count240 by 10
		     \count101=\count240	%101 is second digit of #2/#3
		     \divide\count101 by \count241
		     \count102=\count101
		     \multiply\count102 by \count241
		     \advance\count240 by -\count102
		     \multiply\count240 by 10
		     \count102=\count240	% 102 is the third digit
		     \divide\count102 by \count241
		     \count200=#1\count205=0
		     \count201=\count200
			\multiply\count201 by \count100
		 	\advance\count205 by \count201
		     \count201=\count200
			\divide\count201 by 10
			\multiply\count201 by \count101
			\advance\count205 by \count201
			%
		     \count201=\count200
			\divide\count201 by 100
			\multiply\count201 by \count102
			\advance\count205 by \count201
			%
		     \edef\@result{\number\count205}
}
\def\compute@wfromh{
		% computing : width = height * (bbw / bbh)
		\in@hundreds{\@p@sheight}{\@bbw}{\@bbh}
		%\ps@typeout{ \@p@sheight * \@bbw / \@bbh, = \@result }
		\edef\@p@swidth{\@result}
		%\ps@typeout{w from h: width is \@p@swidth}
}
\def\compute@hfromw{
		% computing : height = width * (bbh / bbw)
	        \in@hundreds{\@p@swidth}{\@bbh}{\@bbw}
		%\ps@typeout{ \@p@swidth * \@bbh / \@bbw = \@result }
		\edef\@p@sheight{\@result}
		%\ps@typeout{h from w : height is \@p@sheight}
}
\def\compute@handw{
		\if@height 
			\if@width
			\else
				\compute@wfromh
			\fi
		\else 
			\if@width
				\compute@hfromw
			\else
				\edef\@p@sheight{\@bbh}
				\edef\@p@swidth{\@bbw}
			\fi
		\fi
}
\def\compute@resv{
		\if@rheight \else \edef\@p@srheight{\@p@sheight} \fi
		\if@rwidth \else \edef\@p@srwidth{\@p@swidth} \fi
		%\ps@typeout{rheight = \@p@srheight, rwidth = \@p@srwidth}
}
%		
% Compute any missing values
\def\compute@sizes{
	\compute@bb
	\if@scalefirst\if@angle
% at this point the bounding box has been adjsuted correctly for
% rotation.  PSFIG does all of its scaling using \@bbh and \@bbw.  If
% a width= or height= was specified along with \psscalefirst, then the
% width=/height= value needs to be adjusted to match the new (rotated)
% bounding box size (specifed in \@bbw and \@bbh).
%    \ps@bbw       width=
%    -------  =  ---------- 
%    \@bbw       new width=
% so `new width=' = (width= * \@bbw) / \ps@bbw; where \ps@bbw is the
% width of the original (unrotated) bounding box.
	\if@width
	   \in@hundreds{\@p@swidth}{\@bbw}{\ps@bbw}
	   \edef\@p@swidth{\@result}
	\fi
	\if@height
	   \in@hundreds{\@p@sheight}{\@bbh}{\ps@bbh}
	   \edef\@p@sheight{\@result}
	\fi
	\fi\fi
	\compute@handw
	\compute@resv}

%
% \psfig
% usage : \psfig{file=, height=, width=, bbllx=, bblly=, bburx=, bbury=,
%			rheight=, rwidth=, clip=}
%
% "clip=" is a switch and takes no value, but the `=' must be present.
\def\psfig#1{\vbox {
	% do a zero width hard space so that a single
	% \psfig in a centering enviornment will behave nicely
	%{\setbox0=\hbox{\ }\ \hskip-\wd0}
	%
	\ps@init@parms
	\parse@ps@parms{#1}
	\compute@sizes
	%
	\ifnum\@p@scost<\@psdraft{
		%
		\special{ps::[begin] 	\@p@swidth \space \@p@sheight \space
				\@p@sbbllx \space \@p@sbblly \space
				\@p@sbburx \space \@p@sbbury \space
				startTexFig \space }
		\if@angle
			\special {ps:: \@p@sangle \space rotate \space} 
		\fi
		\if@clip{
			\if@verbose{
				\ps@typeout{(clip)}
			}\fi
			\special{ps:: doclip \space }
		}\fi
		\if@prologfile
		    \special{ps: plotfile \@prologfileval \space } \fi
%		\if@decmpr{
%			\if@verbose{
%			\ps@typeout{psfig: including \@p@sfile.Z \space }
%			}\fi
%			\special{ps: plotfile "`zcat \@p@sfile.Z" \space }
%		}\else{
			\if@verbose{
				\ps@typeout{psfig: including \@p@sfile \space }
			}\fi
			\special{ps: plotfile \@p@sfile \space }
%		}\fi
		\if@postlogfile
		    \special{ps: plotfile \@postlogfileval \space } \fi
		\special{ps::[end] endTexFig \space }
		% Create the vbox to reserve the space for the figure.
		\vbox to \@p@srheight sp{
		% 1/92 TJD Changed from "true sp" to "sp" for magnification.
			\hbox to \@p@srwidth sp{
				\hss
			}
		\vss
		}
	}\else{
		% draft figure, just reserve the space and print the
		% path name.
		\if@draftbox{		
			% Verbose draft: print file name in box
			\hbox{\frame{\vbox to \@p@srheight sp{
			\vss
			\hbox to \@p@srwidth sp{ \hss \@p@sfile \hss }
			\vss
			}}}
		}\else{
			% Non-verbose draft
			\vbox to \@p@srheight sp{
			\vss
			\hbox to \@p@srwidth sp{\hss}
			\vss
			}
		}\fi	



	}\fi
}}
\psfigRestoreAt
\let\@=\LaTeXAtSign
%%%%%%%%%%%%%%%%%%%%%%%%%%%%%%%%%%%%%%%%%%%%%%%%%%
\def\ColliderTableInsert#1%
{{%
    \parindent = 0pt \leftskip = 0pt \rightskip = 0pt%
    \vskip .4in%
    \nobreak%
    \vskip -.5in%
    \leavevmode%
    \centerline{\psfig{figure=#1,clip=t}}%
    \nobreak%
    \vglue .1in%
    \nobreak%
    \vskip -.3in%
    \nobreak%
}}
%USAGE:   \FigureInsert{RPP$FIGURES:blahblah.ps}
\global\def\FigureInsert#1#2%
% #1 The filename of the PostScript figure to be inserted.
% #2 Either: "left", "center", or "right".
{{%
    \def\CompareStrings##1##2%
    {%
        TT\fi%
        \edef\StringOne{##1}%
        \edef\StringTwo{##2}%
        \ifx\StringOne\StringTwo%
    }%
    \parindent = 0pt \leftskip = 0pt \rightskip = 0pt%
    \vskip .4in%
    \leavevmode%
    \if\CompareStrings{#2}{left}%
        \leftline{\psfig{figure=figures/#1,clip=t}}%
    \else\if\CompareStrings{#2}{center}%
        \centerline{\psfig{figure=figures/#1,clip=t}}%
    \else\if\CompareStrings{#2}{right}%
        \rightline{\psfig{figure=figures/#1,clip=t}}%
    \fi\fi\fi%
    \nobreak%
    \vglue .1in%
    \nobreak%
}}
%USAGE:   \FigureInsert{RPP$FIGURES:blahblah.ps}{xxx}
\global\def\FigureInsertScaled#1#2#3%
% #1 The filename of the PostScript figure to be inserted.
% #2 Either: "left", "center", or "right".
{{%
    \def\CompareStrings##1##2%
    {%
        TT\fi%
        \edef\StringOne{##1}%
        \edef\StringTwo{##2}%
        \ifx\StringOne\StringTwo%
    }%
    \parindent = 0pt \leftskip = 0pt \rightskip = 0pt%
    \vskip .4in%
    \leavevmode%
    \if\CompareStrings{#2}{left}%
        \leftline{\psfig{figure=figures/#1,height=#3,clip=t}}%
    \else\if\CompareStrings{#2}{center}%
        \centerline{\psfig{figure=figures/#1,height=#3,clip=t}}%
    \else\if\CompareStrings{#2}{right}%
        \rightline{\psfig{figure=figures/#1,height=#3,clip=t}}%
    \fi\fi\fi%
    \nobreak%
    \vglue .1in%
    \nobreak%
}}
%USAGE:   \FigureInsertScaled{RPP$FIGURES:blahblah.ps}{xxx}{dimen to scale}
%
%%%%%%%%%%%%%%%%%%%%%%%%%%%%%%%%%%%%%%%%%%%%%%%%%%
\def\insertpsfigure#1#2#3#4{
\hbox to \hsize
    {
        \vbox to #1%HEIGHT OF FIGURE
        {
            \hsize = 0in
            \special
            {
                insert rpp$figures:#2,% NAME OF FIGURE
                top=#3,left=#4% position on page  MUST CHANGE
            }
            \vss
        }
        \hss
    }
}
\def\insertpsfiguremag#1#2#3#4#5{
\hbox to \hsize
    {
        \vbox to #1%HEIGHT OF FIGURE
        {
            \hsize = 0in
            \special
            {
                insert rpp$figures:#2,% NAME OF FIGURE
                top=#3,left=#4,% position on page  MUST CHANGE
                magnification=#5%
            }
            \vss
        }
        \hss
    }
}
%USAGE \insertpsfigure{Height of figure}%
%   {Name of figure}{top placement}{left-right placement}
%USAGE \insertpsfigure{5.5in}{barymagmom.ps}{7.1in}{-1.3in}
%%%%%%%%%%%%%%%%%%%%%%%%%%%%%%%%%%%%%%%%%%%%%%%%%%
\newdimen\beforefigureheight
\newdimen\afterfigureheight
\beforefigureheight=-.5in
\afterfigureheight=-.3in
%
\def\RPPfigure#1#2#3{
\vskip \beforefigureheight
\FigureInsert{#1}{#2}
\vskip \afterfigureheight
\FigureCaption{#3}
\WWWfigure{#1}
}
%USAGE   \RPPfigure{xxx.ps}{center}{caption goes here}
\def\RPPfigurescaled#1#2#3#4{
\vskip \beforefigureheight
\FigureInsertScaled{#1}{#2}{#3}
\vskip \afterfigureheight
\FigureCaption{#4}
\WWWfigure{#1}
}
%USAGE   \RPPfigurescaled{xxx.ps}{center}{9.4in}{caption goes here}
%FOR FIGURES WITHOUT CHAPTER NUMBERS LIKE IN THE TEXT use:
\def\RPPtextfigure#1#2#3{
\vskip \beforefigureheight
\FigureInsert{#1}{#2}
\vskip \afterfigureheight
\FigureCaption{#3}
\WWWtextfigure{#1}
}
%USAGE   \RPPtextfigure{xxx.ps}{center}{caption goes here}
%%%%%%%%%%%%%%%%%%%%%%%%%%%%%%%%%%%%%%%%%%%%%%%%%%
%%%%%%%%%%%%%%%%%%%%%%%%%%%%%%%%%%%%%%%%%%%%%%%%%%
\newcount\Firstpage
%indexmac.tex
%LABELING PAGENUMBERS FOR INDEX
\newif\ifpageindexopen       \newread\pageindexread \newwrite\pageindexwrite
%
\ifx\WHATEVERIWANT\undefined
\else
\def\rppordb{}
\fi
%%tmp \ifnum\BigBookOrDataBooklet=1
%%tmp \immediate\openout\pageindexwrite=\jobname\rppordb.ind
%%tmp \else
%%tmp \immediate\openout\pageindexwrite=\jobname.ind
%%tmp \fi
%
\newcount\lastpage      \lastpage=0\relax
\def\Page#1{%
\write\pageindexwrite{\string\xdef\string#1{\the\pageno}}%
}
%
%%%%%%%%%%%%%%%%%%%%%%%%%%%%%%%%%%%%%%%%%
\newtoks\FigureCaptiontok
\global\def\FigureCaption#1{
\Caption
#1
\endCaption%
\global\FigureCaptiontok={#1}%
}
%%%%%%%%%%%%%%%%%%%%%%%%%%%%%%%%%%%%%%%%%%%%%%%%%%
\newtoks\ABlanktok
\ABlanktok={ }
%CREATING LIST OF FIGURES FOR WORLD-WIDE WEB
\newif\ifwwwfigureopen       \newread\wwwfigureread \newwrite\wwwfigurewrite
%
\ifx\WHATEVERIWANT\undefined
\else
\def\rppordb{}
\fi
%% \ifnum\BigBookOrDataBooklet=1
%% \immediate\openout\wwwfigurewrite=\jobname\rppordb.wwwfig
%% \else
%% \immediate\openout\wwwfigurewrite=\jobname\rppordb.wwwfig
%% \fi
%
\global\def\WWWfigure#1{%
%\message{HERE I AM AND WHAT AM I DOING HERE}%
\immediate\write\wwwfigurewrite{%
	\string\figurename{\string#1}%
}
\immediate\write\wwwfigurewrite{%
	\string\figurenumber{\the\chapternum.\the\fignum}%
}
\immediate\write\wwwfigurewrite{%
        \string\figurecaption{\the\FigureCaptiontok}%
}
\immediate\write\wwwfigurewrite{%
	\the\ABlanktok
}
	}%
\global\def\WWWhead#1{%
\immediate\write\wwwfigurewrite{%
	#1%
	}%
}
\newtoks\widthofcolumntoks
\widthofcolumntoks={\widthofcolumn=}
\global\def\WWWwidthofcolumn#1{%
\immediate\write\wwwfigurewrite{%
	\the\widthofcolumntoks#1%
	}%
}
%%%%%%%%%%%%%%%%%%%%%%%%%%%%%%%%%%%%%%%%%%%%%%%%%%%%%%%%%%%%%%%%%
%FOR FIGURES WITHOUT CHAPTER NUMBERS LIKE IN THE TEXT use:
\global\def\WWWtextfigure#1{%
%\message{HERE I AM AND WHAT AM I DOING HERE}%
\immediate\write\wwwfigurewrite{%
	\string\figurename{\string#1}%
}
\immediate\write\wwwfigurewrite{%
	\string\figurenumber{\the\fignum}%
}
\immediate\write\wwwfigurewrite{%
        \string\figurecaption{\the\FigureCaptiontok}%
}
\immediate\write\wwwfigurewrite{%
	\the\ABlanktok
}
	}%
\global\def\WWWhead#1{%
\immediate\write\wwwfigurewrite{%
	#1%
	}%
}
%%%%%%%%%%%%%%%%%%%%%%%%%%%%%%%%%%%%%%%%%%%%%%%%%%%%%%%%%%%%%%%%%
\def\ABlank{ }
\def\IndexEntry#1%
% #1  The code of the Index/TOC page number entry.
%     It may contain non-alphabetics.
{%
    \write\pageindexwrite%
    {%
       \string\expandafter%
       \string\def\string\csname\ABlank\noexpand#1%
       \string\endcsname\expandafter{\the\pageno}%
    }%
}
%%%%%%%%%%%%%%%%%%%%%%%%%%%%%%%%%%%%%%%%%%%%%%%%%%%%%%%%%%%%%%%%%
%
\def\lastpagenumber{%
\write\pageindexwrite{\string\def\string\lastpage{\the\pageno}}%
}
\def\bumpuppagenumber{%
\pageno=\lastpage \advance\pageno by 1 \Firstpage=\pageno}
\def\donotbumpuppagenumber{\pageno=\lastpage  \Firstpage=\pageno}
%
\let\indexpage=\IndexEntry
%%%%%%%%%%%%%%%%%%%%%%%%%%%%%%%%%%%%%%%%%%%%%%%%%%%%%%%%%%%%%%%%%
%%%%%%%%%%%%%%%%%%%%%%%%%%%%%%%%%%%%%%%%%%%%%%%%%%%%%%%%%%%%%%%%%
%sports2col.tex
%TWO-COLUMN FORMAT
\def\swingit{\ifodd\pageno\hoffset=.8in\else\hoffset=.3in\fi}%
\def\swingit{\ifodd\pageno\hoffset=0in\else\hoffset=0in\fi}%
\def\swingit{\ifodd\pageno\hoffset=.1in\else\hoffset=.1in\fi}%
\def\swingit{\ifodd\pageno\hoffset=.08in\else\hoffset=.08in\fi}%
\def\swingit{\ifodd\pageno\hoffset=.12in\else\hoffset=.12in\fi}%
\def\swingit{\relax}
\newdimen\Fullpagewidth                 \Fullpagewidth=8.75in
\newdimen\Halfpagewidth                 \Halfpagewidth=4.25in
\newdimen\fullhsize
\newcount\columnbreak
\newdimen\VerticalFudge
\VerticalFudge =-.32in
\fullhsize=\Fullpagewidth \hsize=\Halfpagewidth
\def\fullline{\hbox to\fullhsize}

\let\knuthmakeheadline=\makeheadline
\let\knuthmakefootline=\makefootline
\def\dbmakeheadline{\vbox to 0pt{\vskip-22.5pt
%     \fullline{\vbox to8.5pt{}\the\headline}\vss}\nointerlineskip}
     \line{\vbox to10pt{}\the\headline}\vss}\nointerlineskip}
\def\dbmakefootline{\baselineskip=24pt \line{\the\footline}}
\let\lr=L \newbox\leftcolumn 
\def\ScalingPostScript#1{\special{ps: #1 #1 scale}}
\def\dbonecolumn{\hsize=4.25in%
%\message{THIS IS DB ONECOLUMN IN EFFECT}
\output={%
\swingit%
\shipout\vbox{%
	\parindent = 0pt%GW
            \leftskip = 0pt%GW
            \nointerlineskip%GW
            \ScalingPostScript{\RetentionPostScript}%GW
            \nointerlineskip%GW
\makeheadline
\pagebody
\makefootline
\vglue \VerticalFudge
\nointerlineskip\SetOverPageBox{}\copy\OverPageBox}%GW
\advancepageno}
\ifnum\outputpenalty>-20000 \else\dosupereject\fi
}
\def\onecolumn{\hsize=8.75in%
\output={%
\swingit%
\shipout\vbox{%
	\parindent = 0pt%GW
            \leftskip = 0pt%GW
            \nointerlineskip%GW
            \ScalingPostScript{\RetentionPostScript}%GW
            \nointerlineskip%GW
\makeheadline
\pagebody
\makefootline
\vglue \VerticalFudge
\nointerlineskip\SetOverPageBox{}\copy\OverPageBox}%GW
\advancepageno}
\ifnum\outputpenalty>-20000 \else\dosupereject\fi
}
\def\twocol{\output={%
     \if L\lr
   \global\setbox\leftcolumn=\columnbox \global\let\lr=R
 \else \doubleformat \global\let\lr=L\fi
 \ifnum\outputpenalty>-20000 \else\dosupereject\fi}
\def\doubleformat{\shipout\vbox{%
	\parindent = 0pt%GW
            \leftskip = 0pt%GW
            \nointerlineskip%GW
            \ScalingPostScript{\RetentionPostScript}%GW
            \nointerlineskip%GW
\makeheadline%
    \fullline{\box\leftcolumn\hfil\columnbox}
    \makefootline%
\vglue \VerticalFudge
\nointerlineskip\SetOverPageBox{}\copy\OverPageBox}%GW
   \advancepageno}}
\def\columnbox{\leftline{\pagebody}}
\def\makeheadline{\vbox to 0pt{\vskip-22.5pt
%     \fullline{\vbox to8.5pt{}\the\headline}\vss}\nointerlineskip}
     \fullline{\vbox to10pt{}\the\headline}\vss}\nointerlineskip}
\def\makefootline{\baselineskip=24pt \fullline{\the\footline}}
%% More uniform spacing can be obtained by changing \doublecolskip.
%% Reduces the number of black boxes in narrow columns
%
%DEFINE WAYS TO BREAK COLUMNS
%
\columnbreak=0
\ifnum\columnbreak=1
\def\CB{\vfill\eject}\fi
\ifnum\columnbreak=0
\def\CB{\relax}\fi
\def\columnbreaknopar{%
   {\parfillskip=0pt\par}\vfill\eject\noindent\ignorespaces}
\def\columnbreakpar{\vfill\eject\ignorespaces}
\gdef\breakrefitem{\hangafter=0\hangindent=\refindent}
%\gdef\breakrefpage{\gdef\paroreject{\vfill\eject}}
%\gdef\breakrefpage{\gdef\@refbreak{\vfill\eject}}
%\gdef\breakrefpage{\gdef\refskip{\vfill\eject\Refskip}}
%\gdef\breakrefpagewithline{\gdef\paroreject{\midline\vfill\eject}}
\def\midline{\vskip .25in \noindent
\setbox1=\hbox to 8.75in{%
   \hss\vrule width 5.6in height 1.9pt\hss}\wd1=0pt\box1
\vskip .25in \bigskip\bigskip \noindent
\setbox2=\hbox to 8.75in{\hss QUARKMODEL SECTION GOES HERE\hss}\wd2=0pt\box2}
%\gdef\stopbreakrefpage{\gdef\@refbreak{\ifbreakrefs\par\fi}}
%\gdef\stopbreakrefpage{\gdef\refskip{\Refskip}}
\def\@refitem#1{%
   \paroreject \hangafter=0 \hangindent=\refindent \Textindent{#1.}}
\def\refitem#1{%
   \paroreject \hangafter=0 \hangindent=\refindent \Textindent{#1.}}
%\stopbreakrefpage
%
%%%%%%%%%%%%%%%%%%%%%%%%%%%%%%%%%%%%%%%%%%%%%%%%%%%%%%%%%%%%%%%%%
%%%%%%%%%%%%%%%%%%%%%%%%%%%%%%%%%%%%%%%%%%%%%%%%%%%%%%%%%%%%%%%%%
%%%%%%%%%%%%%%%%%%%%%%%%%%%%%%%%%%%%%%%%%%%%%%%%%%%%%%%%%%%%%%%%%
% rppmac.tex
%%%%%%%%%%%%%%%%%%%%%%%%%%%%%%%%%%%%%%%%%%%%%%%%%%
%%%%%%%%%%%%%%%%%%%%%%%%%%%%%%%%%%%%%%%%%%%%%%%%%%%%%%%%%%%%%%%%%%%%%%%%%
%
%SLIGHTLY BIGGER SIZED SCRIPT FONTS ARE USED in TENPOINT
%HOWEVER, SOMETIMES A SMALLERSUBFONT IS DESIRED:
%
\def\smallersubfont{%
  \textfont0=\eightrm \scriptfont0=\sevenrm \scriptscriptfont0=\sevenrm
  \textfont1=\eighti \scriptfont1=\seveni \scriptscriptfont1=\seveni
  \textfont2=\eightsy \scriptfont2=\sevensy \scriptscriptfont2=\sevensy
  \textfont3=\eightex \scriptfont3=\sevenex \scriptscriptfont3=\sevenex}
%
\def\biggersubfont{%
  %
\textfont0=\tenrm \scriptfont0=\eightrm \scriptscriptfont0=\sevenrm
  \textfont1=\teni \scriptfont1=\eighti \scriptscriptfont1=\seveni
  \textfont2=\tensy \scriptfont2=\eightsy \scriptscriptfont2=\sevensy
  \textfont3=\tenex \scriptfont3=\eightex \scriptscriptfont3=\sevenex}
%
%%%%%%%%%%%%%%%%%%%%%%%%%%%%%%%%%%%%%%%%%%%%%%%%%%%%%%%%%%%%%%%%%%%%%%%%%
% GENERAL PARAMETERS  + DEFINITIONS
\raggedright
%NOTE  \raggedright must be invoked before definition of \doublecolskip!! 
%as \raggedright also defines \spaceskip
%% More uniform spacing can be obtained by changing \doublecolskip.
%% Reduces the number of black boxes in narrow columns
\newskip\doublecolskip                          % new \spaceskip
\global\doublecolskip=3.333333pt plus3.333333pt minus1.00006pt %more stretch
   \global\spaceskip=\doublecolskip%           % stretch for small cols.
\parindent=12pt
\tenpoint\singlespace
%%%%%%%%%%%%%%%%%%%%%%%%%%%%%%%%%%%%%%%%%%%%%%%%%%%%%%%%%%%%%%%%%%%%%%%%%
\def\ninepointvspace{
  \normalbaselineskip=9pt
  \setbox\strutbox=\hbox{\vrule height7pt depth2pt width0pt}%
  \normalbaselines}
\newdimen\strutskip
\def\strut {\vrule height 0.7\strutskip
                   depth 0.3\strutskip
                   width 0pt}%
\def\setstrut {%
     \strutskip = \baselineskip
}
\setstrut
%
\def\Folio{\ifnum\pageno<0 \romannumeral-\pageno
           \else \number\pageno \fi }
%%%%%%%%%%%%%%%%%%%%%%%%%%%%%%%%%%%
\def\RPPonly#1{\beginRPPonly #1\endRPPonly}
\def\DBonly#1{\beginDBonly #1\endDBonly}
\def\nocropmarks{%
\footline={\hss\sevenrm\today\quad\TimeOfDay\hss}%NO CROP MARKS: BOTTOM HALF
}
\def\blackbox{\overfullrule=5pt}%use to see black boxes
\def\noblackbox{\overfullrule=0pt}%use to get rid of black boxes
\blackbox
\def\okbreak{\penalty-100\relax}
%
\def\fn#1{{}^{#1}}
%
%%%%%%%%%%%%%%%%%%%%%%%%%%%%%%%%%%%
%GARY'S MACROS FOR CROP MARKS
%%%%%%%%%%%%%%%%%%%%%%%%%%%%%%%%%%%
%\input [bettya.lbl100.sports.macros]comparestrings
    \def\CompareStrings#1#2%
    {%
        TT\fi%
        \edef\StringOne{#1}%
        \edef\StringTwo{#2}%
        \ifx\StringOne\StringTwo%
    }%
%%%%%%%%%%%%%%%%%%%%%%%%%%%%%%%%%%%
%%%%%%%%%%%%%%%%%%%%%%%%%%%%%%%%%%%
%\input [bettya.lbl100.sports.macros]publication_specs

%
%
%   This file is invoked with the \input macro and it establishes
%   typesetting parameters for various publishers.
%
%
%   It \def's the following macros:
%       \PublicationName
%           "RPP"
%           "Particle Physics Booklet"
%       \Publisher
%           "Physical Review D"
%           "Physics Letters B"
%           "Particle Physics Booklet"
%       \InputSize
%           "9.60in"
%           "8.75in"
%           "4.50in"
%           "4.25in"
%       \RetentionPrinter
%           "85"
%           "60"
%           "letter"
%           "WWW-odd"
%           "original"
%       \RetentionPostScript
%           ".8730650" for example
%       \CropMarkChoice
%           "Yes" or "No"
%       \BleederTabChoice
%           "Yes" or "No"
%       \PageNumberingStyle
%           "Consecutive"
%           "Roman Sections"
%
%   It sets the following counter register:
%       \BigBookOrDataBooklet
%
%   It sets the following dimensions:
%       \vsize (it does not set \hsize)
%       \hoffset
%       \voffset
%
%   It sets the following dimension registers:
%       \StartImageHsize
%       \StartImageVsize
%       \StartStockHsize
%       \StartStockVsize
%       \FinalImageHsize
%       \FinalImageVsize
%       \FinalStockHsize
%       \FinalStockVsize
%
%
%    =========================================================================
%
%
%
% String comparison routine from Stephan von Bechtolsheim's "TeX in Practice".
% Invoke it like this:
% \if \CompareStrings{first string}{second string}
%     do true stuff
% \else
%     do false stuff
% \fi
%
%
\def\CompareStrings#1#2%
{%
        TT\fi%
        \edef\StringOne{#1}%
        \edef\StringTwo{#2}%
        \ifx\StringOne\StringTwo%
}
%
%
%
%% {\newlinechar=`\|%
%% \def\obeyspaces{\catcode`\ =\active}%
%% {\obeyspaces\global\let =\space}
%% \obeyspaces%
%% \message{||For what publication should the output be formatted?||}
%% \message{1  Big Book (RPP) published format -- double columns|}
%% \message{2  Big Book (RPP) WWW format single column|}
%% \message{3  Particle Physics Booklet|}
%% \message{|SELECTION (Enter 1, 2, or 3):  }}
%% \read-1 to\PublicationNameSelection
%% \if\CompareStrings{\PublicationNameSelection}{1 }
%%     \def\Publisher{Physical Review D}
%%     \def\PublicationName{RPP}
%%     \def\RPPcolumn{two}
%%     \BigBookOrDataBooklet = 1
%%        \WhichSection=1
%% \else\if\CompareStrings{\PublicationNameSelection}{2 }
    \def\Publisher{Physical Review D}
    \def\PublicationName{RPP}
    \def\RPPcolumn{one}
    \BigBookOrDataBooklet = 1
       \WhichSection=7
%% \else\if\CompareStrings{\PublicationNameSelection}{3 }
%%     \def\Publisher{Particle Physics Booklet}
%%     \def\PublicationName{Particle Physics Booklet}
%%     \def\RPPcolumn{one}
%%     \BigBookOrDataBooklet = 2
%%        \WhichSection=2
%% \fi\fi\fi
%
%
%
\if\CompareStrings{\PublicationName}{RPP}
        \def\InputSize{8.75in}
\else\if\CompareStrings{\PublicationName}{Particle Physics Booklet}
        \def\InputSize{4.25in}
\fi\fi
%
%
%   Decide print retention
%
%
    \if\CompareStrings{\RPPcolumn}{two}
        \def\RetentionPrinter{85}
    \fi
\if\CompareStrings{\RPPcolumn}{one}
    \def\RetentionPrinter{original}
    \fi
    \if\CompareStrings{\PublicationName}{Particle Physics Booklet}
        \def\RetentionPrinter{60}
    \fi
%
%
    \def\CropMarkChoice{No}
    \def\BleederTabChoice{No}
%
%
%   Set the page numbering style.
%
%
    \def\PageNumberingStyle{Consecutive}
%
%
%   Set the horizontal and vertical sizes.
%
%
\newdimen\StartImageHsize
\newdimen\StartImageVsize
\newdimen\StartStockHsize
\newdimen\StartStockVsize
\newdimen\FinalImageHsize
\newdimen\FinalImageVsize
\newdimen\FinalStockHsize
\newdimen\FinalStockVsize
%
\StartImageHsize = \InputSize
%   \RetentionPostScript = \FinalImageHsize /
%                          \RetentionPrinter /
%                          \StartImageHsize
%
%   \StartImageVsize =     \FinalImageVsize /
%                          \RetentionPrinter /
%                          \RetentionPostscript
%   \StartStockHsize =     \FinalStockHsize  / 
%                          \RetentionPrinter /
%                          \RetentionPostscript
%   \StartStockVsize =     \FinalStockVsize  / 
%                          \RetentionPrinter /
%                          \RetentionPostscript
%
\if\CompareStrings{\Publisher}{Physical Review D}
    \FinalImageHsize       =  7.05in
    \FinalImageVsize       = 10.05in
    \FinalStockHsize       =  8.25in
    \FinalStockVsize       = 11.25in
%
    \if\CompareStrings{\InputSize}{9.60in}
        \if\CompareStrings{\RetentionPrinter}{85}
            \def\RetentionPostScript{.863970588}
        \else\if\CompareStrings{\RetentionPrinter}{letter}
            \def\RetentionPostScript{.734375000}
        \else\if\CompareStrings{\RetentionPrinter}{WWW-odd}
            \def\RetentionPostScript{.911458333}
        \else\if\CompareStrings{\RetentionPrinter}{original}
            \def\RetentionPostScript{1.0}
        \fi\fi\fi\fi
        \StartImageVsize = 13.54255319in  % 9.50in
        \StartStockHsize = 11.11702128in  % 9.50in
        \StartStockVsize = 15.15957448in  % 9.50in
        \StartImageVsize = 13.68510638in
        \StartStockHsize = 11.23404255in
        \StartStockVsize = 15.31914894in
    \else\if\CompareStrings{\InputSize}{8.75in}
        \if\CompareStrings{\RetentionPrinter}{85}
            \def\RetentionPostScript{.947899160}
        \else\if\CompareStrings{\RetentionPrinter}{letter}
            \def\RetentionPostScript{.805714286}
        \else\if\CompareStrings{\RetentionPrinter}{WWW-odd}
            \def\RetentionPostScript{1.0}
        \else\if\CompareStrings{\RetentionPrinter}{original}
            \def\RetentionPostScript{1.0}
        \fi\fi\fi\fi
        \StartImageVsize = 12.47340425in
        \StartStockHsize = 10.2393617in
        \StartStockVsize = 13.96276595in
    \fi\fi
%
\else\if\CompareStrings{\Publisher}{Physics Letters B}
    \FinalImageHsize       =  6.60in
    \FinalImageVsize       =  9.50in
    \FinalStockHsize       =  7.50in
    \FinalStockVsize       = 10.30in
%
    \if\CompareStrings{\InputSize}{9.60in}
        \if\CompareStrings{\RetentionPrinter}{85}
            \def\RetentionPostScript{.808823529}
        \else\if\CompareStrings{\RetentionPrinter}{letter}
            \def\RetentionPostScript{.687500000}
        \else\if\CompareStrings{\RetentionPrinter}{WWW-odd}
            \def\RetentionPostScript{.911458333}
        \else\if\CompareStrings{\RetentionPrinter}{original}
            \def\RetentionPostScript{1.0}
        \fi\fi\fi\fi
        \StartImageVsize = 13.8181818in
        \StartStockHsize = 10.9090909in
        \StartStockVsize = 14.9818181in
    \else\if\CompareStrings{\InputSize}{8.75in}
        \if\CompareStrings{\RetentionPrinter}{85}
            \def\RetentionPostScript{.887394958}
        \else\if\CompareStrings{\RetentionPrinter}{letter}
            \def\RetentionPostScript{.754285714}
        \else\if\CompareStrings{\RetentionPrinter}{WWW-odd}
            \def\RetentionPostScript{1.0}
        \else\if\CompareStrings{\RetentionPrinter}{original}
            \def\RetentionPostScript{1.0}
        \fi\fi\fi\fi
        \StartImageVsize = 12.594696in
        \StartStockHsize =  9.943181in
        \StartStockVsize = 13.655303in
    \fi\fi
\else\if\CompareStrings{\Publisher}{Particle Physics Booklet}
    \FinalImageHsize       =  2.60in
    \FinalImageVsize       =  4.70in
    \FinalStockHsize       =  3.00in
    \FinalStockVsize       =  5.00in
%
    \if\CompareStrings{\InputSize}{4.50in}
        \if\CompareStrings{\RetentionPrinter}{60}
            \def\RetentionPostScript{.962962962}
        \else\if\CompareStrings{\RetentionPrinter}{letter}
            \def\RetentionPostScript{.633333333333}
        \else\if\CompareStrings{\RetentionPrinter}{WWW-odd}
            \def\RetentionPostScript{1.27}
        \else\if\CompareStrings{\RetentionPrinter}{original}
            \def\RetentionPostScript{1.0}
        \fi\fi\fi\fi
        \StartImageVsize = 8.134615393in
        \StartStockHsize = 5.192307698in
        \StartStockVsize = 8.653846154in
    \else\if\CompareStrings{\InputSize}{4.25in}
        \if\CompareStrings{\RetentionPrinter}{60}
            \def\RetentionPostScript{1.019607843}
        \else\if\CompareStrings{\RetentionPrinter}{letter}
            \def\RetentionPostScript{.726495726}
        \else\if\CompareStrings{\RetentionPrinter}{WWW-odd}
            \def\RetentionPostScript{1.344705882}
        \else\if\CompareStrings{\RetentionPrinter}{original}
            \def\RetentionPostScript{1.0}
        \fi\fi\fi\fi
        \StartImageVsize =  7.682692308in
        \StartStockHsize =  4.903846154in
        \StartStockVsize =  8.173076923in
    \fi\fi
\fi\fi\fi
%
%
%
\vsize = \StartImageVsize
%
% Determine the dimensions of the image plus any crop marks to determine
% what kind of paper to use.
%
\newdimen \NeededHsize
\newdimen \NeededVsize
\newdimen \CropMarkAddition
%
\if\CompareStrings{\CropMarkChoice}{Yes}
    \CropMarkAddition = 2in
    \NeededHsize = \StartStockHsize
    \NeededVsize = \StartStockVsize
    \advance \NeededHsize by \CropMarkAddition
    \advance \NeededVsize by \CropMarkAddition
    \NeededHsize = \RetentionPostScript\NeededHsize
    \NeededVsize = \RetentionPostScript\NeededVsize
\else
    \NeededHsize = \RetentionPostScript\StartImageHsize
    \NeededVsize = \RetentionPostScript\StartImageVsize
\fi
%
% Set the physical paper size.
%
\newdimen \PaperSizeWidth
\newdimen \PaperSizeHeight
\def\PaperSize{ledger}
\PaperSizeWidth  = 11in
\PaperSizeHeight = 17in
\ifdim \NeededHsize <  8.50in
\ifdim \NeededVsize < 11.00in
    \def\PaperSize{letter}
    \PaperSizeWidth  =  8.5in
    \PaperSizeHeight = 11.0in
\fi\fi
%
% Set the horizontal and vertical offsets.
%
%
\if\CompareStrings{\CropMarkChoice}{Yes}
    \hoffset = .625in % I do not know why this works, but it centers the output.
    \dimen1 = \PaperSizeWidth
    \advance \dimen1 by -\NeededHsize
    \divide  \dimen1 by 2
    \advance \hoffset by \dimen1
%
    \voffset = .625in % This does not quite center the output.
    \dimen1 = \PaperSizeHeight
    \advance \dimen1 by -\NeededVsize
    \divide  \dimen1 by 2
    \advance \voffset by \dimen1
\else
    \hoffset = -.5in
    \voffset = -.5in
\fi
%
%
%
%
%% \if\CompareStrings{\PaperSize}{ledger}
%%     {\newlinechar=`\!%
%%     \def\obeyspaces{\catcode`\ =\active}%
%%     {\obeyspaces\global\let =\space}
%%     \obeyspaces%
%%     \message{!}
%%     \message{       ----------------------------------------------------------!}
%%     \message{       |                                                        |!}
%%     \message{       | BE SURE TO:  USE DVIPS/MODE=tabloid AND                |!}
%%     \message{       |              SEND THIS OUTPUT TO THE LARGE PAPER QUEUE |!}
%%     \message{       |                                                        |!}
%%     \message{       ----------------------------------------------------------!}
%%     \message{!}
%%     }
%% %
%% %   The following statement was used until 9/21/94 to instruct Tomas Rokkicki's
%% %   DVIPS to instruct the QMS 860 to use tabloid paper. It seems that the
%% %   qualifier DVIPS/MODE="tabloid" works as well.
%% %
%% %   \special{papersize=11in,17in}
%% %
%% \fi
%%%%%%%%%%%%%%%%%%%%%%%%%%%%%%%%%%%
\newcount\BleederPointer
\BleederPointer=7
%
%   Set the contents of the page overlay.
%
%
%        \def\BleederTabChoice{No}%  TURNS OFF BleederTab
\newbox\OverPageBox
\def\SetOverPageBox#1%
% #1 '{Initialize}' when the size of the box needs to be determined,
%    but the actual contents of the box are unimportant.
%    Pass as empty {} when the contents are about to be used.
{%
%\message{HERE I AM AND WHAT AM I DOING HERE}%
    \setbox\OverPageBox = \vbox%
    {{%
        \if\CompareStrings{\BleederTabChoice}{Yes}%
            \BleederTab%
            {\BleederPointer}%
            {10}%
            {\StartImageHsize}%
            {\StartImageVsize}%
            {0.2in}%
        \fi%
        \nointerlineskip%
        \if\CompareStrings{\CropMarkChoice}{Yes}%
            {%
                \if\CompareStrings{\RetentionPrinter}{100}%
                    \def\temp{}%
                \else%
                    \def\temp{\PublicationName}%
                \fi%
                \CropMarks%
                {\temp}%
                {\RetentionPrinter}%
                {\StartStockHsize}%
                {\StartStockVsize}%
                {\StartImageHsize}%
                {\StartImageVsize}%
            }%
        \fi%
    }}%
%
%
%   Make the box have only height, no depth.
%
%
    \dimen0 = \ht\OverPageBox%
    \advance\dimen0 by \dp\OverPageBox%
    \ht\OverPageBox = \dimen0%
    \dp\OverPageBox = 0pt%
}
%
%%%%%%%%%%%%%%%%%%%%%%%%%%%%%%%%%%%%%%%%%%%%%%%%%%%%%%%%%%%%%%
%%%%%%%%%%%%%%%%%%%%%%%%%%%%%%%%%%%%%%%%%%%%%%%%%%%%%%%%%%%%%%
%       journal.tex
%%%%%%%%%%%%%%%%%%%%%%%%%%%%%%%%%%%%%%%%%%%%%%%%%%%%%%%%%%%%%%
%journal.tex
%%%%%%%%%%%%%%%%%%%%%%%%%%%%%%%%%%%%%%%%%%%%%%%%%%%%%%%%%%%%%%
%These are the macros for inputting journal, conferences, and books.
%They are used by the Sports Section and the Minireviews
%%%%%%%%%%%%%%%%%%%%%%%%%%%%%%%%%%%%%%%%%%%%%%%%%%%%%%%%%%%%%%
%Reference abbreviations
%Example:  \nim212,319(1983)
%
%Reference abbreviations
%Example:  \nim212,319(1983)
\def\anp#1,#2(#3){{\rm Adv.\ Nucl.\ Phys.\ }{\bf #1}, {\rm#2} {\rm(#3)}}
\def\aip#1,#2(#3){{\rm Am.\ Inst.\ Phys.\ }{\bf #1}, {\rm#2} {\rm(#3)}}
\def\aj#1,#2(#3){{\rm Astrophys.\ J.\ }{\bf #1}, {\rm#2} {\rm(#3)}}
\def\ajs#1,#2(#3){{\rm Astrophys.\ J.\ Supp.\ }{\bf #1}, {\rm#2} {\rm(#3)}}
\def\ajl#1,#2(#3){{\rm Astrophys.\ J.\ Lett.\ }{\bf #1}, {\rm#2} {\rm(#3)}}
\def\ajp#1,#2(#3){{\rm Am.\ J.\ Phys.\ }{\bf #1}, {\rm#2} {\rm(#3)}}
\def\apny#1,#2(#3){{\rm Ann.\ Phys.\ (NY)\ }{\bf #1}, {\rm#2} {\rm(#3)}}
\def\apnyB#1,#2(#3){{\rm Ann.\ Phys.\ (NY)\ }{\bf B#1}, {\rm#2} {\rm(#3)}}
\def\apD#1,#2(#3){{\rm Ann.\ Phys.\ }{\bf D#1}, {\rm#2} {\rm(#3)}}
\def\ap#1,#2(#3){{\rm Ann.\ Phys.\ }{\bf #1}, {\rm#2} {\rm(#3)}}
\def\ass#1,#2(#3){{\rm Ap.\ Space Sci.\ }{\bf #1}, {\rm#2} {\rm(#3)}}
\def\astropp#1,#2(#3)%
    {{\rm Astropart.\ Phys.\ }{\bf #1}, {\rm#2} {\rm(#3)}}
\def\aap#1,#2(#3)%
    {{\rm Astron.\ \& Astrophys.\ }{\bf #1}, {\rm#2} {\rm(#3)}}
\def\araa#1,#2(#3)%
    {{\rm Ann.\ Rev.\ Astron.\ Astrophys.\ }{\bf #1}, {\rm#2} {\rm(#3)}}
\def\arnps#1,#2(#3)%
    {{\rm Ann.\ Rev.\ Nucl.\ and Part.\ Sci.\ }{\bf #1}, {\rm#2} {\rm(#3)}}
\def\arns#1,#2(#3)%
   {{\rm Ann.\ Rev.\ Nucl.\ Sci.\ }{\bf #1}, {\rm#2} {\rm(#3)}}
\def\cqg#1,#2(#3){{\rm Class.\ Quantum Grav.\ }{\bf #1}, {\rm#2} {\rm(#3)}}
\def\cpc#1,#2(#3){{\rm Comp.\ Phys.\ Comm.\ }{\bf #1}, {\rm#2} {\rm(#3)}}
\def\cjp#1,#2(#3){{\rm Can.\ J.\ Phys.\ }{\bf #1}, {\rm#2} {\rm(#3)}}
\def\cmp#1,#2(#3){{\rm Commun.\ Math.\ Phys.\ }{\bf #1}, {\rm#2} {\rm(#3)}}
\def\cnpp#1,#2(#3)%
   {{\rm Comm.\ Nucl.\ Part.\ Phys.\ }{\bf #1}, {\rm#2} {\rm(#3)}}
\def\cnppA#1,#2(#3)%
   {{\rm Comm.\ Nucl.\ Part.\ Phys.\ }{\bf A#1}, {\rm#2} {\rm(#3)}}
\def\el#1,#2(#3){{\rm Europhys.\ Lett.\ }{\bf #1}, {\rm#2} {\rm(#3)}}
\def\epjC#1,#2(#3){{\rm Eur.\ Phys.\ J.\ }{\bf C#1}, {\rm#2} {\rm(#3)}}
\def\grg#1,#2(#3){{\rm Gen.\ Rel.\ Grav.\ }{\bf #1}, {\rm#2} {\rm(#3)}}
\def\hpa#1,#2(#3){{\rm Helv.\ Phys.\ Acta }{\bf #1}, {\rm#2} {\rm(#3)}}
\def\ieeetNS#1,#2(#3)%
    {{\rm IEEE Trans.\ }{\bf NS#1}, {\rm#2} {\rm(#3)}}
\def\IEEE #1,#2(#3)%
    {{\rm IEEE }{\bf #1}, {\rm#2} {\rm(#3)}}
\def\ijar#1,#2(#3)%
  {{\rm Int.\ J.\ of Applied Rad.\ } {\bf #1}, {\rm#2} {\rm(#3)}}
\def\ijari#1,#2(#3)%
  {{\rm Int.\ J.\ of Applied Rad.\ and Isotopes\ } {\bf #1}, {\rm#2} {\rm(#3)}}
\def\jcp#1,#2(#3){{\rm J.\ Chem.\ Phys.\ }{\bf #1}, {\rm#2} {\rm(#3)}}
\def\jgr#1,#2(#3){{\rm J.\ Geophys.\ Res.\ }{\bf #1}, {\rm#2} {\rm(#3)}}
\def\jetp#1,#2(#3){{\rm Sov.\ Phys.\ JETP\ }{\bf #1}, {\rm#2} {\rm(#3)}}
\def\jetpl#1,#2(#3)%
   {{\rm Sov.\ Phys.\ JETP Lett.\ }{\bf #1}, {\rm#2} {\rm(#3)}}
\def\jpA#1,#2(#3){{\rm J.\ Phys.\ }{\bf A#1}, {\rm#2} {\rm(#3)}}
\def\jpG#1,#2(#3){{\rm J.\ Phys.\ }{\bf G#1}, {\rm#2} {\rm(#3)}}
\def\jpamg#1,#2(#3)%
    {{\rm J.\ Phys.\ A: Math.\ and Gen.\ }{\bf #1}, {\rm#2} {\rm(#3)}}
\def\jpcrd#1,#2(#3)%
    {{\rm J.\ Phys.\ Chem.\ Ref.\ Data\ } {\bf #1}, {\rm#2} {\rm(#3)}}
\def\jpsj#1,#2(#3){{\rm J.\ Phys.\ Soc.\ Jpn.\ }{\bf G#1}, {\rm#2} {\rm(#3)}}
\def\lnc#1,#2(#3){{\rm Lett.\ Nuovo Cimento\ } {\bf #1}, {\rm#2} {\rm(#3)}}
\def\nature#1,#2(#3){{\rm Nature} {\bf #1}, {\rm#2} {\rm(#3)}}
\def\nc#1,#2(#3){{\rm Nuovo Cimento} {\bf #1}, {\rm#2} {\rm(#3)}}
\def\nim#1,#2(#3)%
   {{\rm Nucl.\ Instrum.\ Methods\ }{\bf #1}, {\rm#2} {\rm(#3)}}
\def\nimA#1,#2(#3)%
    {{\rm Nucl.\ Instrum.\ Methods\ }{\bf A#1}, {\rm#2} {\rm(#3)}}
\def\nimB#1,#2(#3)%
    {{\rm Nucl.\ Instrum.\ Methods\ }{\bf B#1}, {\rm#2} {\rm(#3)}}
\def\np#1,#2(#3){{\rm Nucl.\ Phys.\ }{\bf #1}, {\rm#2} {\rm(#3)}}
\def\npps#1,#2(#3){{\rm Nucl.\ Phys.\ (Proc.\ Supp.) }{\bf #1}, {\rm#2} {\rm(#3)}}
\def\mnras#1,#2(#3){{\rm MNRAS\ }{\bf #1}, {\rm#2} {\rm(#3)}}
\def\medp#1,#2(#3){{\rm Med.\ Phys.\ }{\bf #1}, {\rm#2} {\rm(#3)}}
\def\mplA#1,#2(#3){{\rm Mod.\ Phys.\ Lett.\ }{\bf A#1}, {\rm#2} {\rm(#3)}}
\def\npA#1,#2(#3){{\rm Nucl.\ Phys.\ }{\bf A#1}, {\rm#2} {\rm(#3)}}
\def\npB#1,#2(#3){{\rm Nucl.\ Phys.\ }{\bf B#1}, {\rm#2} {\rm(#3)}}
\def\npBps#1,#2(#3){{\rm Nucl.\ Phys.\ (Proc.\ Supp.) }{\bf B#1},
{\rm#2} {\rm(#3)}}
\def\pasp#1,#2(#3){{\rm Pub.\ Astron.\ Soc.\ Pac.\ }{\bf #1}, {\rm#2} {\rm(#3)}}
\def\pl#1,#2(#3){{\rm Phys.\ Lett.\ }{\bf #1}, {\rm#2} {\rm(#3)}}
\def\fp#1,#2(#3){{\rm Fortsch.\ Phys.\ }{\bf #1}, {\rm#2} {\rm(#3)}}
\def\ijmpA#1,#2(#3)%
   {{\rm Int.\ J.\ Mod.\ Phys.\ }{\bf A#1}, {\rm#2} {\rm(#3)}}
\def\ijmpE#1,#2(#3)%
   {{\rm Int.\ J.\ Mod.\ Phys.\ }{\bf E#1}, {\rm#2} {\rm(#3)}}
\def\plA#1,#2(#3){{\rm Phys.\ Lett.\ }{\bf A#1}, {\rm#2} {\rm(#3)}}
\def\plB#1,#2(#3){{\rm Phys.\ Lett.\ }{\bf B#1}, {\rm#2} {\rm(#3)}}
\def\pnasus#1,#2(#3)%
   {{\it Proc.\ Natl.\ Acad.\ Sci.\ \rm (US)}{B#1}, {\rm#2} {\rm(#3)}}
\def\ppsA#1,#2(#3){{\rm Proc.\ Phys.\ Soc.\ }{\bf A#1}, {\rm#2} {\rm(#3)}}
\def\ppsB#1,#2(#3){{\rm Proc.\ Phys.\ Soc.\ }{\bf B#1}, {\rm#2} {\rm(#3)}}
\def\pr#1,#2(#3){{\rm Phys.\ Rev.\ }{\bf #1}, {\rm#2} {\rm(#3)}}
\def\prA#1,#2(#3){{\rm Phys.\ Rev.\ }{\bf A#1}, {\rm#2} {\rm(#3)}}
\def\prB#1,#2(#3){{\rm Phys.\ Rev.\ }{\bf B#1}, {\rm#2} {\rm(#3)}}
\def\prC#1,#2(#3){{\rm Phys.\ Rev.\ }{\bf C#1}, {\rm#2} {\rm(#3)}}
\def\prD#1,#2(#3){{\rm Phys.\ Rev.\ }{\bf D#1}, {\rm#2} {\rm(#3)}}
\def\prept#1,#2(#3){{\rm Phys.\ Reports\ } {\bf #1}, {\rm#2} {\rm(#3)}}
\def\prslA#1,#2(#3)%
   {{\rm Proc.\ Royal Soc.\ London }{\bf A#1}, {\rm#2} {\rm(#3)}}
\def\prl#1,#2(#3){{\rm Phys.\ Rev.\ Lett.\ }{\bf #1}, {\rm#2} {\rm(#3)}}
\def\ps#1,#2(#3){{\rm Phys.\ Scripta\ }{\bf #1}, {\rm#2} {\rm(#3)}}
\def\ptp#1,#2(#3){{\rm Prog.\ Theor.\ Phys.\ }{\bf #1}, {\rm#2} {\rm(#3)}}
\def\ppnp#1,#2(#3)%
	{{\rm Prog.\ in Part.\ Nucl.\ Phys.\ }{\bf #1}, {\rm#2} {\rm(#3)}}
\def\ptps#1,#2(#3)%
   {{\rm Prog.\ Theor.\ Phys.\ Supp.\ }{\bf #1}, {\rm#2} {\rm(#3)}}
\def\pw#1,#2(#3){{\rm Part.\ World\ }{\bf #1}, {\rm#2} {\rm(#3)}}
\def\pzetf#1,#2(#3)%
   {{\rm Pisma Zh.\ Eksp.\ Teor.\ Fiz.\ }{\bf #1}, {\rm#2} {\rm(#3)}}
\def\rgss#1,#2(#3){{\rm Revs.\ Geophysics \& Space Sci.\ }{\bf #1},
        {\rm#2} {\rm(#3)}}
\def\rmp#1,#2(#3){{\rm Rev.\ Mod.\ Phys.\ }{\bf #1}, {\rm#2} {\rm(#3)}}
\def\rnc#1,#2(#3){{\rm Riv.\ Nuovo Cimento\ } {\bf #1}, {\rm#2} {\rm(#3)}}
\def\rpp#1,#2(#3)%
    {{\rm Rept.\ on Prog.\ in Phys.\ }{\bf #1}, {\rm#2} {\rm(#3)}}
\def\science#1,#2(#3){{\rm Science\ } {\bf #1}, {\rm#2} {\rm(#3)}}
\def\sjnp#1,#2(#3)%
   {{\rm Sov.\ J.\ Nucl.\ Phys.\ }{\bf #1}, {\rm#2} {\rm(#3)}}
\def\panp#1,#2(#3)%
   {{\rm Phys.\ Atom.\ Nucl.\ }{\bf #1}, {\rm#2} {\rm(#3)}}
\def\spu#1,#2(#3){{\rm Sov.\ Phys.\ Usp.\ }{\bf #1}, {\rm#2} {\rm(#3)}}
\def\surveyHEP#1,#2(#3)%
    {{\rm Surv.\ High Energy Physics\ } {\bf #1}, {\rm#2} {\rm(#3)}}
\def\yf#1,#2(#3){{\rm Yad.\ Fiz.\ }{\bf #1}, {\rm#2} {\rm(#3)}}
\def\zetf#1,#2(#3)%
   {{\rm Zh.\ Eksp.\ Teor.\ Fiz.\ }{\bf #1}, {\rm#2} {\rm(#3)}}
\def\zp#1,#2(#3){{\rm Z.~Phys.\ }{\bf #1}, {\rm#2} {\rm(#3)}}
\def\zpA#1,#2(#3){{\rm Z.~Phys.\ }{\bf A#1}, {\rm#2} {\rm(#3)}}
\def\zpC#1,#2(#3){{\rm Z.~Phys.\ }{\bf C#1}, {\rm#2} {\rm(#3)}}
%
%%%%%%%%%%%%%%%%%%%%%%%%%%%%%%%%%%%%%%%%%%%%%%%%
%
%book.tex
\def\ExpTechHEP{{\it Experimental Techniques in High Energy
Physics}\rm, T.~Ferbel (ed.) (Addison-Wesley, Menlo Park, CA, 1987)}
%
\def\MethExpPhys#1#2{{\it Methods
 of Experimental Physics}\rm, L.C.L.~Yuan and
C.-S.~Wu, editors, Academic Press, 1961, Vol.~#1, p.~#2}
%
\def\MethTheorPhys#1{{\it Methods of Theoretical Physics}, McGraw-Hill,
New York, 1953, p.~#1}
%
\def\xsecReacHEP{%
{\it Total Cross Sections for Reactions of High Energy Particles},
Landolt-B\"ornstein, New Series Vol.~{\bf I/12~a} and {\bf I/12~b},
ed.~H.~Schopper (1988)}
%
\def\xsecReacHEPgray{%
{\it Total Cross Sections for Reactions of High Energy Particles},
Landolt-B\"ornstein, New Series Vol.~{\bf I/12~a} and {\bf I/12~b},
ed.~H.~Schopper (1988).
Gray curve shows Regge fit from \Tbl{hadronic96}
}
%
\def\xsecHadronicCaption{%
\noindent%
Hadronic total and elastic cross sections vs. laboratory beam momentum
and total center-of-mass energy.
Data courtesy A.~Baldini,
 V.~Flaminio, W.G.~Moorhead, and D.R.O.~Morrison, CERN;
and COMPAS Group, IHEP, Serpukhov, Russia.
See \xsecReacHEP.\par}
\def\xsecHadronicCaptiongray{%
\noindent%
Hadronic total and elastic cross sections vs. laboratory beam momentum
and total center-of-mass energy.
Data courtesy A.~Baldini,
 V.~Flaminio, W.G.~Moorhead, and D.R.O.~Morrison, CERN;
and COMPAS Group, IHEP, Serpukhov, Russia.
See \xsecReacHEPgray.
\par}
%
%\xsecReacHEP
%%%%%%%%%%%%%%%%%%%%%%%%%%%%%%%%%%%%%%%%%%%%%%%%
%conf.tex
\def\LeptonPhotonseventyseven#1{%
{\it Proceedings of the 1977 International
Symposium on Lepton and Photon Interactions at High Energies}
(DESY, Hamburg, 1977), p.~#1}
%
\def\LeptonPhotoneightyseven#1{%
{\it Proceedings of the 1987 International Symposium on
Lepton and Photon Interactions at High Energies}, Hamburg,
July 27--31, 1987, edited by
W.~Bartel and R.~R\"uckl (North Holland, Amsterdam, 1988), p.~#1}
%
\def\HighSensitivityBeauty#1{%
{\it Proceedings of the Workshop on High Sensitivity Beauty Physics
at Fermilab}, Fermilab, November 11--14, 1987, edited by A.J.~Slaughter,
N.~Lockyer, and M.~Schmidt (Fermilab, Batavia, IL, 1988), p.~#1}
%
\def\ScotlandHEP#1#2{%
{\it Proceedings of the XXVII International
Conference on High-Energy Physics},
Glasgow, Scotland, July 20--27, 1994, edited by P.J. Bussey
and I.G. Knowles
(Institute of Physics, Bristol, 1995), Vol.~#1, p.~#2}
%
\def\SDCCalorimetryeightynine#1{%
{\it Proceedings of the Workshop on Calorimetry for the
Supercollider},
 Tuscaloosa, AL, March 13--17, 1989, edited by R.~Donaldson and
M.G.D.~Gilchriese (World Scientific, Teaneck, NJ, 1989), p.~#1}
%
\def\Snowmasseightyeight#1{%
{\it Proceedings of the 1988 Summer Study on High Energy
Physics in the 1990's},
 Snowmass, CO, June 27 -- July 15, 1990, edited by F.J.~Gilman and S.~Jensen,
(World Scientific, Teaneck, NJ, 1989) p.~#1}
%
\def\Snowmasseightyeightnopage{%
{\it Proceedings of the 1988 Summer Study on High Energy
Physics in the 1990's},
 Snowmass, CO, June 27 -- July 15, 1990, edited by F.J.~Gilman and S.~Jensen,
(World Scientific, Teaneck, NJ, 1989)}
%
\def\Ringbergeightynine#1{%
{\it Proceedings of the Workshop on Electroweak Radiative Corrections
for $e^+ e^-$ Collisions},
Ringberg, Germany, April 3--7, 1989, edited by J.H.~Kuhn,
(Springer-Verlag, Berlin, Germany, 1989) p.~#1}
%
\def\EurophysicsHEPeightyseven#1{%
{\it Proceedings of the International Europhysics Conference on
High Energy Physics},
Uppsala, Sweden, June 25 -- July 1, 1987, edited by O.~Botner,
(European Physical Society, Petit-Lancy, Switzerland, 1987) p.~#1}
%
\def\WarsawEPPeightyseven#1{%
{\it Proceedings of the 10$\,^{th}$ Warsaw Symposium on Elementary Particle
Physics},
Kazimierz, Poland, May 25--30, 1987, edited by Z.~Ajduk,
(Warsaw Univ., Warsaw, Poland, 1987) p.~#1}
%
\def\BerkeleyHEPeightysix#1{%
{\it Proceedings of the 23$^{rd}$ International Conference on
High Energy Physics},
Berkeley, CA, July 16--23, 1986, edited by S.C.~Loken,
(World Scientific, Singapore, 1987) p.~#1}
%
\def\ExpAreaseightyseven#1{%
{\it Proceedings of the Workshop on Experiments, Detectors, and Experimental
Areas for the Supercollider},
Berkeley, CA, July 7--17, 1987, 
edited by R.~Donaldson and
M.G.D.~Gilchriese (World Scientific, Singapore, 1988), p.~#1}
%
\def\CosmicRayseventyone#1#2{%
{\it Proceedings of the International Conference on Cosmic
Rays},
Hobart, Australia, August 16--25, 1971, 
Vol.~{\bf#1}, p.~#2}
%%%%%%%%%%%%%%%%%%%%%%%%%%%%%%%%%%%%%%%%%%%%%%%%%%%%%%%%%%%%%%
% whichsection.tex
%%%%%%%%%%%%%%%%%%%%%%%%%%%%%%%%%%%%%%%%
\lefteqnsidedimen=22pt %default equation side indent---sports, text, etc 
	\lefteqnside=\lefteqnsidedimen
\newdimen\Textpagelength                \Textpagelength=11.6in
\newdimen\Textplusheadpagelength        \Textplusheadpagelength=12.0in
\Fullpagewidth=8.75in
\Halfpagewidth=4.25in
%\newdimen\biggervsize
\newbox\indexGreek
\newbox\indexOmit
\newbox\wwwfootcitation
\newbox\indexfootline
	\def\IsThisTheFirstpage{\ifnum\pageno=\Firstpage%
                \global\advance\vsize by .3in
                \else\relax\fi
	}
%%%%%%%%%%%%%%%%%%%%%%%%%%%%%%%%%%%%%%%%
%%%%%%%%%%%%%%%%%%%%%%%%%%%%%%%%%%%%%%%%
%\WhichSection=7.tex
        \sectionskip=\bigskipamount
%
        \ifnum\WhichSection=7\relax
\gdef\runningdate{\bgroup\sevenrm\today\quad\TimeOfDay\egroup}
\else
\gdef\runningdate{\relax}
\fi
\advance\voffset by .8in
%HEADLINES:SPORTS
        \ifnum\WhichSection=7\relax
{\newlinechar=`\|%
\def\obeyspaces{\catcode`\ =\active}%
{\obeyspaces\global\let =\space}
\obeyspaces%
\message{2  8 1/2 by 11 paper (DRAFT MODE)|}
\message{HI THERE -- THIS is 7}}
%FOOTLINES:SPORTS
\footline{\IsThisTheFirstpage}
%HSIZE AND VSIZE:SPORTS
       %\biggervsize=\Textpagelength
       %\advance\biggervsize by .3in
%%        \hsize=4.25in\vsize=5.3in
        \hsize=4.25in\vsize=7.3in
%%PZ  \advance\vsize by 1in\advance\hoffset by .7in
 \advance\vsize by 1.5in\advance\hoffset by .7in
%%PZ \advance\voffset by -.7in
 \advance\voffset by -.4in
             \let\twocol\relax
   \let\makeheadline=\dbmakeheadline
        \let\makefootline=\dbmakefootline
           \def\printtheheading{\centerline{\copy\HEADFIRST}\vskip .1in}
        \headline={\ifodd\pageno\hfil\copy\RUNHEADhbox\quad\elevenssbf \Folio%
                \else\elevenssbf\Folio\quad\copy\RUNHEADhbox\hfill\fi}
        \footline={\hfill\runningdate\hfill}
\setbox\wwwfootcitation=\vtop {%
   \vglue .1in%
   \hbox to  6in{%
\hss{\sevenrm CITATION: D.E. Groom {\sevenit et al.}, 
European Physical Journal {\sevenbf C15}, 1 (2000)}\hss}
   \vglue .005in%
   \hbox to  6in{%
\hss{\sevenrm  %1997 edition 
available on
the PDG WWW pages (URL: {\ninett http://pdg.lbl.gov/})
\qquad\runningdate}\hss}
   \vss%
             \vss}%
%\footline={\ifnum\pageno=1\firstfoot\else\restoffoot\fi}
\gdef\firstfoot{\centerline{\hss\copy\wwwfootcitation\hss}}
\gdef\restoffoot{\centerline{\hss\runningdate\hss}}
%%PZ \footline={\ifnum\pageno=1\firstfoot\else\restoffoot\fi}
\footline={\restoffoot}
%THIS AND THAT:SPORTS
        \Linewidth=.00003pt
        \Linewidth=0pt
        \parskip=\smallskipamount
        \sectionminspace=1.2in
        \tenpoint
\BleederPointer=7
\else
\fi
%%%%%%%%%%%%%%%%%%%%%%%%%%%%%%%%%%%%%%%%
%\WhichSection=1.tex
	\sectionskip=\bigskipamount
%
%HEADLINES:SPORTS
	\ifnum\WhichSection=1\relax
{\newlinechar=`\|%
\def\obeyspaces{\catcode`\ =\active}%
{\obeyspaces\global\let =\space}
\obeyspaces%
\message{1  11x17 paper|}
\message{HI THERE -- THIS is 1}}
%FOOTLINES:SPORTS
\footline{\IsThisTheFirstpage}
%HSIZE AND VSIZE:SPORTS
       %\biggervsize=\Textpagelength
       %\advance\biggervsize by .3in
	\fullhsize=\Fullpagewidth\hsize=\Halfpagewidth
	\vsize=\Textpagelength
%THIS AND THAT:SPORTS
	\Linewidth=.00003pt 
	\Linewidth=0pt 
	\parskip=\smallskipamount
	\sectionminspace=1.2in
	\tenpoint
\BleederPointer=7
\else
\fi
%%%%%%%%%%%%%%%%%%%%%%%%%%%%%%%%%%%%%%%%%%%%%%%%%%%%%%%%%
%%%%%%%%%%%%%%%%%%%%%%%%%%%%%%%%%%%%%%%%%%%%%%%%%%%%%%%%%
%\WhichSection=2.tex
%HSIZE AND VSIZE:DATABOOK
	\ifnum\WhichSection=2\relax
	\VerticalFudge =-.32in
	\VerticalFudge =-.23in
        \hsize=4.25in\vsize=7.3in
\let\boldhead=\boldheaddb
\dbonecolumn
%	\advance\voffset by 1.5in
%	\advance\hoffset by 1in
%HEADLINES:DATABOOK
	\let\makeheadline=\dbmakeheadline
	\let\makefootline=\dbmakefootline
      	   \def\printtheheading{\centerline{\copy\HEADFIRST}\vskip .1in}
	\headline={\ifodd\pageno\hfil\copy\RUNHEADhbox\quad\elevenssbf\Folio%
		\else\elevenssbf\Folio\quad\copy\RUNHEADhbox\hfill\fi}
%FOOTLINES:DATABOOK
%	\footline={%
\footline{}
%THIS AND THAT:DATABOOK
	\tenpoint
	\Linewidth=.00003pt 
	\Linewidth=0pt 
	\parskip=1pt plus 1pt
	\sectionminspace=1in
	\global\sectionskip=\smallskipamount
	\tenpoint
	\abovedisplayskip=\medskipamount
	\belowdisplayskip=\medskipamount
\def\runningheadfont{\tenpoint\it}
\advance\voffset by .8in
\else
\fi
%%%%%%%%%%%%%%%%%%%%%%%%%%%%%%%%%%%%%%%%%%%%%%%%%
%%%%%%%%%%%%%%%%%%%%%%%%%%%%%%%%%%%%%%%%%%%%%%%%%
%%%%%%%%%%%%%%%%%%%%%%%%%%%%%%%%%%%%%%%%%%%%%%%%%
%%%%%%%%%%%%%%%%%%%%%%%%%%%%%%%%%%%%%%%%%%%%%%%%%%
%
%\WhichSection=1\relax % FOR SPORTS
%\WhichSection=2\relax % FOR DATABOOK
%\WhichSection=3\relax % FOR MISC 
	%USED in comp.tex  
%\WhichSection=4\relax % FOR INDEX
	%USED in index.tex and orangeindex.tex
%\WhichSection=5\relax % FOR TABLE OF CONTENTS, AUTHORS
%\WhichSection=6\relax % FOR TEXT, HISTORY, DATABASES
%
%
%
%%%%%%%%%%%%%%%%%%%%%%%%%%%%%%%%%%%%%%%%%%%%%%%%%%%%%%%%%%%%%%
\superrefsfalse%  SQUARE BRACKETS FOR REFERENCES
\refReset\relax% RESET references numbers for each section
\eqReset\relax% RESET equation numbers for each section
\refindent=20pt
%setuphead.tex
%CHECK WHAT voffset is supposed to be from 1992 macros for each thing
%
%%%%%%%%%%%%%%%%%%%%%%%%%%%%%%%%%%%%%%%%%%%%%%%%%%%%%%%%%%%%%%
%%%%%%%%%%%%%%%%%%%%%%%%%%%%%%%%%%%%%%%%%%%%%%%%%%%%%%%%%%%%%%%%%%%%%%%%%%%%%%
% This modification of @GetRefText is for texsis release 2.18 (and later?)   %
%%%%%%%%%%%%%%%%%%%%%%%%%%%%%%%%%%%%%%%%%%%%%%%%%%%%%%%%%%%%%%%%%%%%%%%%%%%%%%
%
%
\newdimen\reftopglue
\reftopglue=0pt
\ATunlock       % make @ a letter for "hidden" macro names
\newif\ifEjectHere \EjectHerefalse
\ifEjectHere
\message{HERE I AM}
\fi
%  \@GetRefText gets the reference text as everything up to the next
%  \endreference and writes it to the .ref file, at least for citation
%  styles which save the list to the end.   #1 is the label.
%  #2 is lookahead past possible line end before \seeCR

\def\@GetRefText#1#2{%                  % get and write reference text
  \ifnum\CiteType=4\else                % footnotes rather than ref list?
    \ifnullname                         % Is name null (set by \@tagref)?
      \p@nctwrite{; }%                  %  punctuation for continuation
      \@refwrite{\@comment ... Reference text for %%
                "#1" defined on page \number\pageno.}% 
      \@refwrite{\@refbreak}%           % and a possible line break
    \else                               % No: just begin another...
      \ifnum\refnum>1\p@nctwrite{. }\fi %!% trailing period, unless first one
\ifEjectHere
  \@refwrite{\string\vfill\string\eject\string\vglue\string\reftopglue}
  \EjectHerefalse
\fi
      \@refwrite{\@comment }%           %
      \@refwrite{\@comment (\the\refnum) Reference text for %%
                "#1" defined on page \number\pageno.}%  
      \@refwrite{\string\@refitem{\the\refnum}{#1}}% write label info
    \fi
  \fi
  %%
  \begingroup                           % start group to write text
    \def\endreference{\NX\endreference}% disable these in references
    \def\reference{\NX\reference}\def\ref{\NX\ref}%   
    \seeCR\newlinechar=`\^^M            % so ^^M seen by \@copyref
    \@copyref#2}                        % copy text line to .ref file

\ATlock         % make @ not a letter (hides internal macros).

%%%%%%%%%%%%%%%%%%%%%%%%%%%%%%%%%%%%%%%%%%%%%%%%%%%%%%%%%%%%%%
%%%%%%%%%%%%%%%%%%%%%%%%%%%%%%%%%%%%%%%%%%%%%%%%%%%%%%%%%%%%%%
%
%\newif\ifPubSpecs
%\PubSpecsfalse
%==================================================*

%- \inputaux{file} looks for an aux file called file and if it exists
%  reads it in.  This file should have tag and label definitions from a
%  run of another job (and now containing \newlabel's and such)

\newread\inputauxin                     % input for \inputaux file

\def\inputaux#1{%  reads in the auxilliary file from a previous run
%qqq \openin\auxfilein=\jobname.aux     % open old .aux file for input
   \openin\inputauxin=#1                % open an aux file for input
   \ifeof\inputauxin\closein\inputauxin % if EOF it's empty.
   \else\closein\inputauxin             % else close it and... 
     \begingroup                        % fix up special characters...
        \def\@tag##1##2{\endgroup       % simpler way to \tag here
           \edef\@@temp{##2}%           % put ``value'' in \@@temp
           \testtag{##1}\XA\xdef\csname\tok\endcsname{\@@temp}}%
       \unSpecial\ATunlock              % special characters not so special
%qqq   \input\jobname.aux \relax        % ...and read in the file
       \input #1 \relax                 % ...and read in the file
     \endgroup                          % back to special characters
   \fi}                                 % else ignore it

%%%%%%%%%%%%%%%%%%%%%%%%%%%%%%%%%%%%%%%%%%%%%%%%%%%%%%%%%%%%%%
\def\trippleheading#1#2#3{\chapter{#1}\label{Chap.\jobname}%
           \centerline{\boldhead\hfill\the\chapternum.~#1\hfill}\vskip .1in%
           \centerline{\boldhead\hfill #2\hfill}\vskip .1in%
           \centerline{\boldhead\hfill #3\hfill}\vskip .2in}
%
%%+++++++++++++++++++++++++++++++++++++++++++++++++++++++++++++++++++++%%
%
\def\oEightColorFig{}
\def\colorFigAv{}
\def\colorFigAvLong{}
\def\twentyTenColorFig{}
\def\PHI{\Phi}
%%
\newlabel{Chap.collidersrpp}{{26}{1}}
\newlabel{Chap.bigbangnucrpp}{{20}{220}}
\newlabel{Chap.hubblerpp}{{21}{224}}
\newlabel{Chap.microwaverpp}{{23}{238}}
\newlabel{Chap.bigbangrpp}{{19}{1}}
\newlabel{Sec.MuonEnergy}{{27.6}{25}}
\newlabel{Chap.strucfunrpp}{{16}{1}}
\newlabel{Chap.stanmodelrpp}{{10}{1}}
\newlabel{Chap.qcdrpp}{{9}{1}}
\newlabel{Tb.AverageHadronMult}{{40.1}{3}}
\newlabel{Chap.crosssecrpp}{{39}{1}}
\newlabel{Eq.cumul }{{31.6}{2}}
\newlabel{Tb.probone}{{31.1}{10}}
\newlabel{Sec.Normal}{{31.4.3}{6}}
\newlabel{Chap.su3rpp}{{36}{1}}
\newlabel{Sec.epluseminusannihilation}{{39.2}{1}}
\newlabel{Chap.fragrpp}{{17}{1}}

\def\Big{}
\def\Big{}
\def\Bigl{}
\def\Bigr{}
\def\Biggl{}
\def\Biggr{}
